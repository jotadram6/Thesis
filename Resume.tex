\begingroup
\let\clearpage\relax
\let\cleardoublepage\relax
\let\cleardoublepage\relax

\chapter*{Abstract}

During 2012, the Large Hadron Collider (LHC) has delivered proton-proton collisions at 8 TeV center of mass energy to ATLAS and CMS experiments. These two experiments have been designed to discover the Higgs boson and to search for new particles predicted by several theoretical models, as supersymmetry. The Higgs boson has been discovered by ATLAS and CMS experiments on July, 4th of 2012, starting a new era of discoveries in particle physics domain. %In the next run of the LHC, that had begun at a center of mass of 13 TeV in June 2015, the expectation is to confirm the existence of new particles from theory models beyond the Standard Model (SM) of particle physics.

With the confirmation of the existence of the Higgs boson, searches for new physics involving this boson are of major interest. In particular, data can be used to look for new massive particles that decay into the Higgs boson accompanied with other particles of the standard model. One expected signature is a Higgs boson produced with a top quark, the two heaviest particles in the standard model. The standard model predicts a cross section of top-Higgs production, then any enhancement of their associated production will be a clear signature of physics beyond the standard model.

In the first part of my work I describe the theoretical and experimental foundations of the standard model, as well as the experimental device. In the same theoretical chapter I also discuss the formulation of an extension model of the standard model. In addition, I describe a feasibility study of a search of one of the particles predicted by such model.

The second part contains the realization of the search for a top partner, \Tp, within the CMS experiment. This top partner is a new particle very similar to the standard model top-quark, but much heavier, that decays preferentially into a top quark and a Higgs boson. The analysis looks for this particle in the full hadronic final state, where the Higgs boson decays into two b-quarks and the top quark decays into three standard model quarks, a b and two light quarks. In this channel, I reconstruct its mass from the identification of all its decay products. As a result of the analysis, I show the limits on the \Tp~production cross section from the number of observed events in the specific signature.

\begin{otherlanguage}{francais}
\chapter*{R\'{e}sum\'{e}}

Le LHC (Large Hadron Collider) a produit en 2012 des collisions proton-proton à une énergie de 8 TeV  dans le centre de masse pour les expériences ATLAS et CMS. Ces deux expériences ont été conçues pour découvrir le boson de Higgs et pour rechercher des nouvelles particules prédites par des modèles théoriques. Le boson de Higgs a été découvert le 4 juillet 2012 par les expériences ATLAS et CMS. Cette découverte marque le début  d'une nouvelle période dans le domaine. 

Avec la confirmation de l'existence du boson de Higgs, les recherches de nouvelle physique liées à ce boson sont devenues prioritaires. Par exemple, on peut chercher dans les données une nouvelle particule massive qui peut se désintégrer dans un boson de Higgs associé à d'autres particules du modèle standard. Une signature attendue est un boson de Higgs avec un quark top, les deux particules les plus lourdes du modèle standard. Le modèle standard prédit une section efficace pour la production du Higgs avec un quark top, ainsi une mesure de cette section efficace montrant une valeur plus importante prouverait l'existence de physique au-delà du modèle standard.

Dans la première partie de ce manuscrit, je présente les bases théoriques et expérimentales du modèle standard, ainsi que le dispositif expérimental. Dans le même chapitre théorique je discute une extension dans le cadre d'un modèle effectif englobant le modèle standard. De plus, je détaille une étude de faisabilité d'une recherche d'une des nouvelles particules prédites par ce modèle, un quark vectoriel.

Dans la deuxième partie, la recherche dans CMS de ce quark vectoriel \Tp,  partenaire du quark top, est décrite. Ce partenaire du top est une nouvelle particule très similaire au quark top du SM, mais beaucoup plus lourde. On considère le cas où ce nouveau quark se désintègre préférentiellement dans un quark top et un boson de Higgs. J'ai fait cette recherche dans le canal hadronique où le Higgs se désintègre en deux quarks b et le quark top se désintègre en trois quarks, un quark b et deux quarks légers. J'ai reconstruit la masse du T à partir de l'identification des tous ses produits de désintégration. Comme résultat, je montre les limites observées sur la section efficace de production du \Tp~obtenues à partir de cette analyse.
\end{otherlanguage}  

\endgroup