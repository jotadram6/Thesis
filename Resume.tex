\begingroup
\let\clearpage\relax
\let\cleardoublepage\relax
\let\cleardoublepage\relax

\chapter*{Abstract}

The Large Hadron Collider (LHC) has provided during 2012 with proton-proton collisions at 8 TeV center of mass energy the Compact Muon Solenoid (CMS) experiment. These experiments have been designed to discover the Higgs boson and to search for new particles predicted by several theoretical models, as supersymmetry. The Higgs boson has been discovered by ATLAS and CMS experiments on July 4th of 2012, starting a new era of discoveries in particle physics domain. In the next run of the LHC, that had begun at a center of mass of 13 TeV in June 2015, the expectation is to confirm the existence of new particles from theory models beyond the Standard Model (SM) of particle physics.

With the confirmation of the existence of the Higgs boson, searches for new physics involving this boson had become of major interest. In particular, one can look in data for new massive particles that decay into the Higgs boson accompanied with other particles of the SM. One of the most expected signatures is to look for a Higgs boson produced with a top quark, that are the two heaviest particles in the SM. The SM predicts a cross section of top-Higgs production, then any enhancement of their associated production will be a clear signature of physics beyond the SM.

In the first part of my work I describe the experimental and theoretical foundations of the SM, as well as the experimental device. In the same theoretical chapter I also discuss the formulation of an extension model of the SM. In addition, I describe a feasibility study of a search of one of the particles predicted by such model.

The second part contains the realization of the search for a top partner, $T$, within the CMS experiment. This top partner is a new particle very similar to the SM top-quark, but much heavier, that decays preferentially into a top and a Higgs. The analysis looks for this particle in the full hadronic final state, where the Higgs decays into two b-quarks and the top decays into three SM quarks, a b and two light quarks. In this channel, I reconstruct its mass from the identification of all its decay products. As a result of the analysis, I show the limits on the $T$ production cross section from the number of observed events in the specific signature.

\chapter*{R\'{e}sum\'{e}}

Le LHC (Large Hadron Collider) a produit des collisions proton-proton à une énergie de centre de masse de 8 TeV pendant l'année 2012 pour expériences comme le Compact Muon Solenoid (CMS). Ces deux expériences ont été conçu pour découvrir le boson de Higgs et pour rechercher nouvelles particules prédites par modèles théoriques comme la super-symétrie. Le boson de Higgs a été découvert le 4 juillet 2012 par les expériences ATLAS et CMS. Cette découverte démarre une nouvelle période dans le domaine. Avec les nouvelles collisions du LHC, qui ont commencé en juin 2015 à 13 TeV, on espère confirmer l’existence des particules prédits par modèles théoriques au-delà du Modèle Standard (SM) de la physique de particules.

Avec la confirmation de l'existence du boson de Higgs, les recherches de nouvelle physique liées à ce boson ont devenu prioritaires. Particulièrement, on peut chercher dans les données une nouvelle particule massive qui peut se désintégrer dans un boson de Higgs et des autres particules du SM. Une signature spécialement attendue est un boson de Higgs avec un quark top, les deux particules les plus lourdes du SM. Si bien le SM prédit une section efficace pour la production du Higgs avec un quark top, une mesure de cette section efficace montrant une valeur plus importante prouverait l'existence de physique au-delà du SM.

Dans la première partie de ce manuscrit, je présent les bases théoriques et expérimentales du SM, et du dispositif expérimental. Dans le même chapitre théorique je discute une extension généraliste du SM. De plus, je décrit une étude de faisabilité d'une recherche d'une des nouvelles particules prédites par ce modèle.

Dans la deuxième partie, la recherche d'un quark vectoriel, le partenaire du quark top le $T$, dans CMS est décrite. Ce partenaire du top est une nouvelle particule très similaire au quark top du SM, mais beaucoup plus lourde. On considère le cas où ce nouveau quark se désintègre préférentiellement dans un quark top et un boson de Higgs. J'ai fait cette recherche dans le canal hadronique où le Higgs se désintègre en deux quarks b et le quark top se désintègre en trois quarks, un quark b et deux quarks légers. J'ai reconstruit la masse du $T$ à partir de l'identification des touts ces produits de désintégration. Comme résultat, je montre les limites atteintes par cette recherche relatives à la section efficace de production du $T$. 
  

\endgroup