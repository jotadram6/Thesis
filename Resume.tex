\begingroup
\let\clearpage\relax
\let\cleardoublepage\relax
\let\cleardoublepage\relax

\chapter*{Abstract}

The Large Hadron Collider (LHC) has provided during 2012 with proton-proton collisions at 8 TeV center of mass energy the Compact Muon Solenoid (CMS) experiment. These experiments have been designed to discover the Higgs boson and to search for new particles predicted by several theoretical models, as supersymmetry. The Higgs boson has been discovered by ATLAS and CMS experiments on July 4th of 2012, starting a new era of discoveries in particle physics domain. In the next run of the LHC, that had begun at a center of mass of 13 TeV in June 2015, the expectation is to confirm the existence of new particles from theory models beyond the Standard Model (SM) of particle physics.

With the confirmation of the existence of the Higgs boson, searches for new physics involving this boson had become of major interest. In particular, one can look in data for new massive particles that decay into the Higgs boson accompanied with other particles of the SM. One of the most expected signatures is to look for a Higgs boson produced with a top quark, that are the two heaviest particles in the SM. The SM predicts a cross section of top-Higgs production, then any enhancement of their associated production will be a clear signature of physics beyond the SM.

In the first part of my work I describe the experimental and theoretical foundations of the SM, as well as the experimental device. In addition, I also discuss the formulation of an extension model of the SM and the feasibility of a search of one of the particles predicted by such a model.

The second part contains the realization of the search for a top partner, $T$, within the CMS experiment. This top partner is a new particle very similar to the SM top-quark, but much heavier, that decays preferentially into a top and a Higgs. The analysis looks for this particle in the full hadronic final state, where the Higgs decays into two b-quarks and the top decays into three SM quarks, a b and two light quarks. In this channel, I reconstruct its mass from the identification of all its decay products. As a result of the analysis, I show the limits on the $T$ production from the number of observed events in our specific signature. 

\chapter*{R\'{e}sum\'{e}}



\endgroup