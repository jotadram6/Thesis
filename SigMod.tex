
\chapter{Signal modeling}
\label{chap:sigmod}

In the first chapter of this work~\ref{chap:SM}, the different possibilities of VLQ models implementations have been discussed. Nevertheless, for the analysis presented in chapter~\ref{chap:search} we have done MC simulations of a vector like T from an specific model of VLQ's, different models can be also studied, as shown in table~\ref{tab:VLQRepre}. However, their predictions are similar their kinematics can be in principle different for the same production channel. The purpose of this appendix is to look in detail, at generator level, the kinematics of T and the associated jet in single production for the different models. with such study, we try to assess the possibilities of the presented analysis to see a similar object to the one considered.

\section{Model Independence}
\label{sec:modindp}

There are three different possibilities to have a vector like T, as a singlet or as a member of a standard doublet ($Y=7/6$) or non-standard doublet ($Y=1/6$). On the following we will refer to the singlet simply as T, to the standard doublet as TB and to the non-standard doublet as XT. The vector like T can be mixed to third or light generations of SM quarks exclusively, or can be mixed to the three generations. For the single production mode only the singlet representation can be mixed exclusively to third generation. The doublets should be mixed to the three generations or exclusively to light generations. These mixings are controlled by one single parameter for the MC generation, $R_{L}$. $R_{L}=0$ stands for exclusive mixing to third generation, $R_{L}=\infty$ for exclusive mixing to light generations and $R_{L}=0.5$ for shared mixings to all three SM quark generations. 

On the plots that will be presented, the convention for the legend is \textit{Representation\_$R_{L}$ value\_Production channel}. For example, XT\_RLInf\_Tj stands for the XT doublet representation with $R_{L}=\infty$ in the single production of a vector like T associated to a jet. In figure~\ref{fig:accomjet} it can be seen the distributions for the \pt and $\eta$ of the jet produced in association with the T in the single production mode for all possible models and mixings. The comparison is performed with MC samples for a mass of the T of 700 \GeVcc, at parton level. No significant differences between models or mixings is observed. In figure~\ref{fig:Tppt} it can be found the comparison between different models of the \pt of T. 

\begin{figure}[!hbtp]
  \begin{center}
    \includegraphics[width=0.45\textwidth]{figs/Ana/eta6thjetmodels.png}
    \includegraphics[width=0.45\textwidth]{figs/Ana/pt6thjetmodels.png}
    \caption{$\eta$ and $p_{T}$ of the jet produced with the T for different representations and couplings with SM quarks}
    \label{fig:accomjet}
  \end{center}
\end{figure}

\begin{figure}[!hbtp]
  \begin{center}
    \includegraphics[width=0.5\textwidth]{figs/Ana/ptTpmodels.png}
    \caption{$p_{T}$ of the T for different representations and couplings with SM quarks}
    \label{fig:Tppt}
  \end{center}
\end{figure}

In addition, as in the selection there is a cut on the $\Delta R$ between T and the associated jet, the distribution for this variable for different models can be found in figure~\ref{fig:DRmodels}. In this variable also no significant differences are seen. however the differences between the different models are small, they should be diminished when hadronizing and showering MC samples.

\begin{figure}[!hbtp]
  \begin{center}
    \includegraphics[width=0.5\textwidth]{figs/Ana/DRmodels.png}
    \caption{$\Delta R (Tj)$ for different representations and couplings with SM quarks}
    \label{fig:DRmodels}
  \end{center}
\end{figure}

We consider then as maximum a 5\% systematic uncertainty from differences shown between possible models, to take into account the theoretical predictions of kinematics of T and the associated jet in the single production mode.

\section{Tjj compared to Tj}

As an additional study, the cross sections used as theoretical prediction where obtained considering only the main production process of T with just the associated jet. However, if the vector like T exists in nature it should be produced with extra jets added to the main process. We then consider how kinematics of T and the associated jet changes when including an extra jet, and also we calculated the increment of the production cross section when adding an additional jet. For this purpose, an additional MC sample was produced for the same mass (M=700 \GeVcc) where Tj and Tjj process where produced simultaneously and then compared to our main sample, with only Tj sample.

The Tj production is done by an initial state of two quarks, as shown in figure~\ref{fig:ProdDiagSingle}. In contrast, Tjj production come from quark-quark and quark-gluon initial states. Then, a significant contribution to the total production cross section is expected. We found that this increment is of 39\%, using MG to calculate this cross section. An schematic view of the additional quark-quark processes can be seen in figure~\ref{fig:qqTjj}, and for quark-gluon processes in figure~\ref{fig:qqTjj}.

\begin{figure}[!hbtp]
  \begin{center}
    \includegraphics[scale=0.35]{figs/Ana/Tjj_qq_Tgq_1.jpg}
    \includegraphics[scale=0.35]{figs/Ana/Tjj_qq_Tgq_2.jpg}
    \caption{Schematic feynman diagrams of the processes on the Tjj production with quark-quark initial state. All the solid lines are quarks and the curly line is a Z/W boson.}
    \label{fig:qqTjj}
  \end{center}
\end{figure}

\begin{figure}[!hbtp]
  \begin{center}
    \includegraphics[scale=0.7]{figs/Ana/Tjj_qg_Tqq_1.jpg}
    \includegraphics[scale=0.35]{figs/Ana/Tjj_qg_Tqq_2.jpg}
    \includegraphics[scale=0.35]{figs/Ana/Tjj_qg_Tqq_3.jpg}
    \caption{Schematic feynman diagrams of the processes on the Tjj production with quark-gluon initial state. All the solid lines are quarks and the curly line is a Z/W boson.}
    \label{fig:qgTjj}
  \end{center}
\end{figure}

In figure~\ref{fig:6thJ_Tjj} it can be found the comparison of kinematics of the leading jet produced with T for Tj and Tj+Tjj productions. In addition, in figure~\ref{fig:T_Tjj} it can be seen the kinematics of T for the same cases. No significant differences were found. The small tail of low \pt T, is coming from events where the extra jet was produced from a T that radiated a gluon. 

\begin{figure}[!hbtp]
  \begin{center}
    \includegraphics[width=0.45\textwidth]{figs/Ana/pt6thJ_Tjj.png}
    \includegraphics[width=0.45\textwidth]{figs/Ana/eta6thJ_Tjj.png}
    \caption{$p_{T}$ and $\eta$ of the leading jet produced with the T for inclusive Tjj production and exclusive Tj}
    \label{fig:6thJ_Tjj}
  \end{center}
\end{figure}

\begin{figure}[!hbtp]
  \begin{center}
    \includegraphics[width=0.5\textwidth]{figs/Ana/ptT_Tjj.png}
    \includegraphics[width=0.5\textwidth]{figs/Ana/etaT_Tjj.png}
    \caption{$p_{T}$ and $\eta$ of the T for inclusive Tjj production and exclusive Tj}
    \label{fig:T_Tjj}
  \end{center}
\end{figure}

Furthermore, the distribution for $\Delta R (Tj)$ is presented in figure~\ref{fig:DR_Tjj}, where the jet is the leading one. For the inclusive Tjj there is a difference with Tj at low $\Delta R$ that represents 5\% of the whole distribution. This difference is coming from the events were the leading jet is the extra jet produced with the Tj main process.

\begin{figure}[!hbtp]
  \begin{center}
    \includegraphics[width=0.5\textwidth]{figs/Ana/DR_Tjj.png}
    \caption{$\Delta R (Tj)$ of the T for inclusive Tjj production and exclusive Tj}
    \label{fig:DR_Tjj}
  \end{center}
\end{figure}

We then expect a good sensitivity of the analysis selection to Tjj inclusive process. Moreover, taking into consideration hadronization and detector effects, selection efficiencies should be very close to the ones quoted in table~\ref{tab:cutflow}.