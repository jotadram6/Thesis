\chapter*{Conclusions}

This thesis studies followed two axis: Monte-Carlo simulations in the context of the CMS collaboration and a search for the vector like quark \Tp~in proton-proton collisions from LHC run 1.

The Monte-Carlo studies performed are grouped in two main activities. First of all, the maintenance and improvement of CMS production of simulations using MadGraph package. This task was also accompanied by constant support to CMS users of MadGraph. As a result, several new pieces of code were introduced to the central CMS software, CMSSW, and other codes were optimized and updated to new versions of MadGraph or CMSSW. Additionally, valuable support was given to several users for the production and validation of Monte-Carlo samples of their interest. The second main activity was the physics validation of recent versions of MadGraph and new tools arriving to the market for Monte-Carlo simulations. This validation process probed the reliability of this package, an important step for later CMS usage. From this validation activities various releases of MadGraph were validated and replaced old releases in CMS simulation production process. Additionally, the recent developed tool MadSpin was added to the production of \ttbar~samples produced with MadGraph, increasing the capability of CMS to produce this process in terms of less time and cpu resources consumption, in addition of having samples with a better description of \ttbar~production.

The second axis studies followed in two directions. In first instance, a phenomenological feasibility study for a search in CMS data collected in run 1 was performed. Such study showed the observability of a single produced \Tp~in the full hadronic channel, where the \Tp~decays into a top quark and a Higgs boson, and both decaying in the hadronic modes. This study showed how a mass reconstruction was possible in the full hadronic channel with two main tools: a procedure to identify the from the decay of the \Tp~to optimize the selection of the 5 jets coming from the \Tp, and jet b-tagging as a powerful tool to discriminate signal events from backgrounds. On top of this, several variables were identified to further discriminate signal events. As a result, a selection was designed that gave a modest significance when $S/B$ and $S/\sqrt{S+B}$ discriminators.

Based on the phenomenological study a search for a single produced \Tp~was performed in CMS run 1 data. This search improved the identification method of the 5 jets coming from the \Tp. An exclusive efficiency of 70\% was found of the identification process of the decay products from the \Tp. This identification procedure was based in the requirement of at least 3 b-tagged jets, allowing a high discrimination of signal events with respect to QCD background events. The identification process also reconstructed the top quark and Higgs boson coming from the \Tp. Using these objects, a final selection was performed. Finally, a novel data-based background estimation procedure was put in place. This procedure developed a method to estimate the shape of the invariant mass of 5 jets. A separate method was also developed to estimate the normalization of backgrounds, this is the number of expected events from backgrounds after full selection. With the usage of these methods, sensible exclusion limits were achieved. With respect to theoretical predictions of the production cross section of the \Tp as function of its mass, \Tp~masses were excluded between 813 and 862 \GeVcc~at 95\% CL.

Finally, the results from the analysis were put in perspective with respect to similar analyses done by ATLAS and CMS collaborations. It was shown that the analysis presented in this thesis adds an important contribution to the current knowledge of vector like quarks, in terms of exclusion limits and techniques for searching them in the single production mode. Also is important to stress that the analysis explores the full hadronic final state, a difficult channel not widely explored as leptonic ones. Currently, the LHC has restarted proton-proton collisions at 13 TeV center of mass energy, exploring territories never reached before. From these collisions, new searches for \Tp~vector like quarks are expected, with an increased sensitivity to higher masses. It is expected then that for the first time \Tp~masses higher then 1 TeV/$\text{c}^{2}$ will be tested.