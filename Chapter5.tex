\chapter[Single VLQ search]{Search for a single produced T' decaying into top and Higgs in the full hadronic final state}
\label{chap:search}

In the present chapter we describe in full detail the search performed using 2012 data collected by CMS for a T' in the full hadronic final state. The theoretical formalism for such object has been described on chapter~\ref{chap:VLQ}.

\section{Analysis Strategy}
\label{sec:stra}

The strategy of the analysis is based on keeping large signal efficiency while keeping under control the background. Main background of all hadronic final states is multijet production. This background should not present any resonance in the 5-jets invariant mass variable but purely a continuum. In order to keep high signal efficiency while constraining background, strategy to optimize the selection is based on high signal efficiency criteria (around 80-90\%) and on multiplication of them. 

\section{Datasets}
\label{sec:data}

The analysis is based on the MultiJet primary dataset processed with the 22Jan2013 reconstruction.

\begin{table*}[htbH]
\begin{center}
\resizebox{\textwidth}{!}{
\begin{tabular}{|c|c|}
\hline 
Dataset name & Int. Luminosity ($\text{pb}^{-1}$) \\
\hline
/MultiJet/Run2012A-22Jan2013-v1/AOD & 889.4 \\
/MultiJet1Parked/Run2012B-05Nov2012-v2/AOD & 4427.9 \\
/MultiJet1Parked/Run2012C-part1-05Nov2012-v2/AOD & 494.6 \\
/MultiJet1Parked/Run2012C-part2-05Nov2012-v2/AOD & 6654.9 \\
/MultiJet1Parked/Run2012D-part1-10Dec2012-v1/AOD & 5764.9 \\
/MultiJet1Parked/Run2012D-part2-17Jan2013-v1/AOD & 924.2 \\
/MultiJet1Parked/Run2012D-part2-PixelRecover-17Jan2013-v1 & 538.4 \\
\hline
\end{tabular}
}
\caption{List of Multijet Primary Dataset used in the analysis and the corresponding integrated luminosity calculating using the golden JSON\label{tab:datasets}}
\end{center}
\end{table*}\clearpage

\begin{table*}[htbH]
\begin{center}
\resizebox{\textwidth}{!}{
\begin{tabular}{|c|c|c|}
\hline 
Samples & Cross-Section (pb) & Number of events\\
\hline
QCD\_Pt-120to170\_TuneZ2star\_8TeV\_pythia6 & 16\(\times 10^4\) & 5.9M\\
QCD\_Pt-170to300\_TuneZ2star\_8TeV\_pythia6 & 34\(\times 10^3\) & 5.8M\\
QCD\_Pt-300to470\_TuneZ2star\_8TeV\_pythia6 & 18\(\times 10^2\) & 5.9M\\ 
QCD\_Pt-470to600\_TuneZ2star\_8TeV\_pythia6 & 110 & 3.9M\\
QCD\_Pt-600to800\_TuneZ2star\_8TeV\_pythia6 & 27 & 3.9M\\
QCD\_Pt-800to1000\_TuneZ2star\_8TeV\_pythia6 & 3.5 & 3.9M\\
QCD\_HT-500To1000\_TuneZ2star\_8TeV-madgraph-pythia6 & 84\(\times 10^2\) & 30M\\ 
QCD\_HT-1000ToInf\_TuneZ2star\_8TeV-madgraph-pythia6 & 2\(\times 10^2\) & 14M\\ 
DYToCC\_M\_50\_TuneZ2star\_8TeV\_pythia6 & 31\(\times 10^2\) & 2M\\
DYToBB\_M\_50\_TuneZ2star\_8TeV\_pythia6 & 38\(\times 10^2\) & 2M\\
TTJets\_MSDecays\_central\_TuneZ2star\_8TeV-madgraph-tauola & 245.8 [NNLO] & 60M\\
TT\_CT10\_TuneZ2star\_8TeV-powheg-tauola & 245.8 [NNLO] & 22M\\
T\_tW-channel-DR\_TuneZ2star\_8TeV-powheg-tauola & 11.1 [NNLO] &497k\\
T\_s-channel\_TuneZ2star\_8TeV-powheg-tauola & 1.76 [NNLO] & 260k\\
T\_t-channel\_TuneZ2star\_8TeV-powheg-tauola & 30.7 [NNLO] & 3.7M\\
Tbar\_tW-channel-DR\_TuneZ2star\_8TeV-powheg-tauola & 11.1 [NNLO] & 492k\\
Tbar\_s-channel\_TuneZ2star\_8TeV-powheg-tauola & 3.79 [NNLO] & 140k\\
Tbar\_t-channel\_TuneZ2star\_8TeV-powheg-tauola & 56.4 [NNLO] & 1.9M\\
WZ\_TuneZ2star\_8TeV\_pythia6\_tauola & 33.6 [NLO] & 10M\\
ZZ\_TuneZ2star\_8TeV\_pythia6\_tauola & 7.6 [NLO] & 9.8M\\
WW\_TuneZ2star\_8TeV\_pythia6\_tauola & 56 [NLO] & 10M\\
TTH\_Inclusive\_M-125\_8TeV\_pythia6 & 0.13 [NLO] & 100K\\
\hline
\end{tabular}
}
\caption{List of Monte-Carlo background samples used in the analysis, their corresponding cross-section and their number of events.\label{tab:MCbkg}}
\end{center}
\end{table*}\clearpage

\begin{table*}[htbH]
\begin{center}
\resizebox{\textwidth}{!}{
\begin{tabular}{|c|c|c|}
\hline 
Sample & T' Mass & Cross-Section \\
            & (GeV$/c^{2}$) &  (fb) \\
\hline
TprimeJetToTH\_M-600\_TuneZ2star\_8TeV-madgraph\_tauola & 600 & 215.4\\
TprimeJetToTH\_M-650\_TuneZ2star\_8TeV-madgraph\_tauola & 650 & 177.8\\
TprimeJetToTH\_M-700\_TuneZ2star\_8TeV-madgraph\_tauola & 700 & 143.7\\
TprimeJetToTH\_M-750\_TuneZ2star\_8TeV-madgraph\_tauola & 750 & 118.6\\
TprimeJetToTH\_M-800\_TuneZ2star\_8TeV-madgraph\_tauola & 800 & 100\\
TprimeJetToTH\_M-850\_TuneZ2star\_8TeV-madgraph\_tauola & 850 & 84.3\\
TprimeJetToTH\_M-900\_TuneZ2star\_8TeV-madgraph\_tauola & 900 & 72.6\\
TprimeJetToTH\_M-950\_TuneZ2star\_8TeV-madgraph\_tauola & 950 & 62.6\\
TprimeJetToTH\_M-1000\_TuneZ2star\_8TeV-madgraph\_tauola & 1000 & 53.9\\
\hline
\end{tabular}
}
\caption{List of Monte-Carlo background signal used in the analysis, their corresponding cross-section and mass of the T'.\label{tab:MCsig}}
\end{center}
\end{table*}\clearpage
%\begin{TOINCLUDE}Table with Multijet primary datasets and integrated luminosity. Tables for MC samples, backgrounds and signal mass points.\end{TOINCLUDE}

\section{Event selection}
\label{sec:sel}

This section describes first the event processing details and then the
event selection applied. At the end we present the efficiencies
measurement performed and/or scale factors applied.

\subsection{Event processing}

Description of processing details of events.

\subsection{Basic selection}

Description of basic selection and justification of basic selection cuts.

\begin{figure}[!Hhtbp]
  \begin{center}
    \includegraphics[width=0.3\textwidth]{figs/CMSlogo.png}
    \caption{Distribution of number of vertices reconstructed in the events before selection (except patuple creation) for data and for the sum of Monte-Carlo samples on which a weight have been applied.  At this stage, $H_T$ selection is not yet applied and as the QCD samples does not include events with $H_T$ lower than 500 GeV/c}
    \label{fig:Nvtcs}
  \end{center}
\end{figure}\clearpage

\begin{figure}[!Hhtbp]
  \begin{center}
    \includegraphics[width=0.4\textwidth]{figs/CMSlogo.png}
    \includegraphics[width=0.4\textwidth]{figs/CMSlogo.png}
    \includegraphics[width=0.4\textwidth]{figs/CMSlogo.png}
    \includegraphics[width=0.4\textwidth]{figs/CMSlogo.png}
    \includegraphics[width=0.4\textwidth]{figs/CMSlogo.png}
    \includegraphics[width=0.4\textwidth]{figs/CMSlogo.png}
    \caption{Distribution of transverse momentum of the 6
      leading jets in the events. Data is compared to the sum of the
      MC samples normalized to luminosity after basic selection. The gray band represents the statistical
      uncertainties from the sum of the MC background. Reasonable agreement is observed, multijet process is the dominant process at this stage.}
    \label{fig:6jpt}
  \end{center}
\end{figure}\clearpage

\begin{figure}[!Hhtbp]
  \begin{center}
    \includegraphics[width=0.4\textwidth]{figs/CMSlogo.png}
    \includegraphics[width=0.4\textwidth]{figs/CMSlogo.png}
    \includegraphics[width=0.4\textwidth]{figs/CMSlogo.png}
    \includegraphics[width=0.4\textwidth]{figs/CMSlogo.png}
    \includegraphics[width=0.4\textwidth]{figs/CMSlogo.png}
    \includegraphics[width=0.4\textwidth]{figs/CMSlogo.png}
    \caption{Distribution of $\eta$ of the 6 leading jets in
      the events. Data is compared to the sum of the MC samples
      normalized to luminosity after
      basic selection. The gray band represents the statistical
      uncertainties from the sum of the MC background. Reasonable agreement is observed, multijet process is the dominant process at this stage.}
    \label{fig:6jeta}
  \end{center}
\end{figure}\clearpage

\begin{figure}[!Hhtbp]
  \begin{center}
    \includegraphics[width=0.3\textwidth]{figs/CMSlogo.png}
    \caption{Distribution of $H_{T}$ variable for data and the sum of
      the Monte Carlo samples normalized to luminosity after basic selection. The signal sample
      is simply overlay on top of the stack of the MC samples. It has a T mass of 700 GeV/$c^{2}$. The gray band represents the statistical uncertainties from the sum of the MC background. Reasonable agreement is observed, multijet process is the dominant process at this stage.}
    \label{fig:HT}
  \end{center}
\end{figure}\clearpage

\begin{figure}[!Hhtbp]
  \begin{center}
    \includegraphics[width=0.3\textwidth]{figs/CMSlogo.png}
    \caption{B-tag CSVM jet multiplicity for data and Monte Carlo
      samples after all previous criteria [left] and after requiring
      at least 3 b-tagged jets [right]. The signal has 3 b-tag while
      most of the background have mainly 2 b-tag. The disagreement
      observed can be understood as no scale factors
      are applied on the MC in left plots. In right plot, scale factor
      of 3 b-tag jets is applied.  This criteria is the
      last criteria definition the basic selection for the
      analysis. The sum of MC is normalized to the integrated
      luminosity in the data.}
    \label{fig:Nb}
  \end{center}
\end{figure}\clearpage
%\begin{TOINCLUDE}Plots of pt and eta of six leading jets, number of vertices, HT and number of CSVM b-tagged jets\end{TOINCLUDE}

\subsection{T' reconstruction with a $\chi^{2}$ sorting algorithm}
\label{sec:chi2}

\begin{figure}[!Hhtbp]
  \begin{center}
    \includegraphics[width=0.3\textwidth]{figs/CMSlogo.png}
    \caption{Reconstruction efficiency taking into account only events where jets could be matched to partons [left] and to total number of events [right]}
    \label{fig:RecEff}
  \end{center}
\end{figure}\clearpage

\begin{figure}[!Hhtbp]
  \begin{center}
    \includegraphics[width=0.3\textwidth]{figs/CMSlogo.png}
    \includegraphics[width=0.3\textwidth]{figs/CMSlogo.png}
    \includegraphics[width=0.3\textwidth]{figs/CMSlogo.png}
    \caption{Reconstructed top, W and H for the T mass point of 700 GeV/c}
    \label{fig:WHt}
  \end{center}
\end{figure}\clearpage

\begin{figure}[!Hhtbp]
  \begin{center}
    \includegraphics[width=0.3\textwidth]{figs/CMSlogo.png}
    \includegraphics[width=0.3\textwidth]{figs/CMSlogo.png}
    \caption{Reconstructed T mass for all mass points.}
    \label{fig:RecT}
  \end{center}
\end{figure}\clearpage

\begin{table*}[htbH]
\begin{center}
\begin{tabular}{|c|c|c|c|c|}
\hline 
\multicolumn{2}{|c}{Generated} & \multicolumn{3}{|c|}{Reconstructed} \\
Mass (GeV/$c^{2}$) & Width (GeV/$c^{2}$) & Mass (GeV/$c^{2}$) & Width (GeV/$c^{2}$) & $\chi^{2} /$ndf\\
\hline
600 & 0.62 &$616.68\pm14.17$ & $30.73\pm10.52$ & 15.14/17\\
650 & 0.80 &$649.48\pm13.55$ & $37.32\pm10.67$ & 37.82/17\\
700 & 1.02 &$697.54\pm14.25$ & $42.98\pm12.28$ & 4.39/17\\
750 & 1.27 &$740.23\pm15.53$ & $44.62\pm13.68$ & 14.48/17\\
800 & 1.56 &$787.13\pm16.20$ & $49.95\pm12.18$ & 58.67/27\\
850 & 1.89 &$835.79\pm16.30$ & $48.94\pm12.18$ & 36.35/27\\
900 & 2.26 &$882.308\pm18.34$ & $49.91\pm13.84$ & 34.97/27\\
950 & 2.67 &$933.30\pm23.30$ & $54.78\pm18.24$ & 16.00/27\\
1000 & 3.13 &$976.01\pm29.76$ & $58.27\pm24.04$ & 21.17/27\\
\hline
\end{tabular}
\caption{Reconstructed mass and width for T candidate after full analysis selection from a gaussian fit for each signal mass generated. \label{tab:SignalWidths}}
\end{center}
\end{table*}

\begin{figure}[!Hhtbp]
  \begin{center}
    \includegraphics[width=0.3\textwidth]{figs/CMSlogo.png}
    \caption{Normalized distribution of $\chi^{2}$ variable for signal compared to multijet ($H_{T}$ [500, 1000] only) and $t\bar{t}$. The signal sample used has a T' mass of 700 GeV/$c^{2}$. Signal events are peaking at 0 with a falling distribution while background
    have larger tails.}
    \label{fig:chi2}
  \end{center}
\end{figure}\clearpage
%\begin{TOINCLUDE}Table with inclusive and exclusive reconstruction efficiency. Plots of mass of reconstructed T, W, H and top right after reconstruction for all mass points. Table with gaussian fit results.\end{TOINCLUDE}

\subsection{Selection based on reconstructed objects}

Description of the selection in general and detailed description of each variable used.

\begin{figure}[!Hhtbp]
  \begin{center}
    \includegraphics[width=0.3\textwidth]{figs/CMSlogo.png}
    \caption{$\Delta R$ of the 2 b-tag jets used to reconstruct the Higgs candidate after basic selection. The signal which is simply overlaid prefers low $\Delta R$ while backgrounds have larger distribution at higher value. The gray band represents the statistical uncertainties from the sum of the MC background.}
    \label{fig:DRbb}
  \end{center}
\end{figure}\clearpage

\begin{figure}[!Hhtbp]
  \begin{center}
    \includegraphics[width=0.3\textwidth]{figs/CMSlogo.png}
    \caption{Distribution for $\Delta R (W_{cand} H_{cand})$ for data and the sum of Monte Carlo samples. All others criteria are applied up to this one.}
    \label{fig:DRWH}
  \end{center}
\end{figure}\clearpage

\begin{figure}[!Hhtbp]
  \begin{center}
    \includegraphics[width=0.3\textwidth]{figs/CMSlogo.png}
    \caption{Distributions for $ \Delta R (T' j^{6})$  for data and the sum of Monte Carlo samples. All others criteria are applied up to this one. The low statistics in the multijet (QCD) MC sample is visible at this stage. The gray band represents the statistical uncertainties from the sum of the MC background and it is dominated by QCD samples.}
    \label{fig:jet6}
  \end{center}
\end{figure}\clearpage

\begin{figure}[!Hhtbp]
  \begin{center}
    \includegraphics[width=0.3\textwidth]{figs/CMSlogo.png}
    \caption{Distribution of Relative $H_{T}$ for data and the sum of the Monte Carlo samples. Signal can be separated from background in this variable. All others criteria are applied up to this one. The low statistics in the multijet (QCD) MC sample is visible at this stage. The gray band represents the statistical uncertainties from the sum of the MC background and it is dominated by QCD samples.}
    \label{fig:RelHtMass}
  \end{center}
\end{figure}\clearpage

\begin{figure}[!Hhtbp]
  \begin{center}
    \includegraphics[width=0.3\textwidth]{figs/CMSlogo.png}
    \caption{Distribution of Top-Higgs mass asymmetry for data and the sum of the Monte Carlo samples. Signal can be separated from background in this variable. All others criteria are applied before this one. The low statistics in the multijet (QCD) MC sample is visible at this stage. The gray band represents the statistical uncertainties from the sum of the MC background and it is dominated by QCD samples.}
    \label{fig:mthasym}
  \end{center}
\end{figure}\clearpage

\begin{figure}[!Hhtbp]
  \begin{center}
    \includegraphics[width=0.3\textwidth]{figs/CMSlogo.png}
    \caption{Distribution of $(M(top^{2nd})+M(W^{2}))/M(H)$ 
      for data and the sum of the Monte Carlo samples. Signal
      can be separated from background in this variable. All
      others criteria are applied before this one. The low statistics
      in the multijet (QCD) MC sample is visible at this
      stage. The gray band represents the statistical uncertainties
      from the sum of the MC background and it is dominated by QCD
      samples.}
    \label{fig:m2thp}
  \end{center}
\end{figure}\clearpage

\begin{table}[htbH]
\label{tab:Estimators}
\begin{center}
\begin{tabular}{|c|c|c|}
%\hline 
xxxxxxx & xxxxxxx & xxxxxxx
%\hline
\end{tabular}
\caption{$S/B$ and $S/\sqrt{S+B}$ from MC samples for each step of the selection.}
\end{center}
\end{table}\clearpage

%\begin{TOINCLUDE}$N-1$ plots for object selection variables. Table with $S/B$ and $S/\sqrt{S+B}$ from MC for each step of the selection. \end{TOINCLUDE}

\subsection{Efficiencies}
\label{sec:eff}

\subsubsection{Trigger}
\label{sec:trigger}

\begin{figure}[!Hhtbp]
  \begin{center}
    \includegraphics[width=0.3\textwidth]{figs/CMSlogo.png}
    \caption{Efficiency in data and the MC signal samples for events passing trigger bit HLT\_Dijet80\_Dijet60\_Dijet20 with respect to trigger bit HLT\_HT400 after standard selection up to DEFINE SELECTION [included]. In the middle stage of the selection, we observe discrepancies between 10\% and 6\% at higher $p_{T}$. This efficiency is parametrized as function of the 6$^{th}$ jet $p_{T}$.}
    \label{fig:TrigEff}
  \end{center}
\end{figure}\clearpage

\begin{figure}[!Hhtbp]
  \begin{center}
    \includegraphics[width=0.3\textwidth]{figs/CMSlogo.png}
    \caption{Efficiency in data and the MC signal samples for events passing trigger bit HLT\_Dijet80\_Dijet60\_Dijet20 with respect to trigger bit HLT\_HT400 after standard selection up to the Higgs-candidate mass selection (one cut before the final). This efficiency is parametrized as function of the 6$^{th}$ jet $p_{T}$. The dispersion observed is mainly 6\%.}
    \label{fig:TrigEffPostDalitz}
  \end{center}
\end{figure}\clearpage
%\begin{TOINCLUDE}Trigger efficiency plots\end{TOINCLUDE}

\subsubsection{Selection}
\label{sec:seleff}

\begin{figure}[!Hhtbp]
  \begin{center}
    \includegraphics[width=0.3\textwidth]{figs/CMSlogo.png}
    \caption{Selection efficiency for all signal mass points.}
    \label{fig:SelEff}
  \end{center}
\end{figure}\clearpage

\begin{table*}[htbH]
\begin{center}
\resizebox{\textwidth}{!}{
\begin{tabular}{|c|c|c|c|c|c|}
\hline 
Selection & Cut & Signal (M=700 GeV/$c^{2}$) & Multijet & $t\bar{t}$ + single top & Diboson \\
\hline
%Trigger & & & & \\
\multirow{4}{*}{\rotatebox{90}{Basic}} & Trigger and $p_{T}$,$\eta$ selection & 52.66 & 27.40 & 36.54 & 16.12 \\
&$j^{1}>150$~GeV/c & 47.65 & 24.05 & 24.50 & 11.55 \\
&$H_{T}>630$~GeV/c & 44.29 & 20.57 & 20.57 & 8.98 \\
&$n_{b}^{CSVM}>=3$ & 13.74 & 0.16 & 1.73 & 0.11 \\
\hline
\multirow{10}{*}{\rotatebox{90}{Analysis}} & $\chi^{2}<800$ & 13.38 & 0.14 & 1.71 & 0.11  \\
&$\Delta R(bb) <1.2$ & 11.06 & 0.06 & 0.67 & 0.06 \\
&$1.6 < \Delta R (W_{cand} H_{cand}) < 4.0$ & 10.44 & 0.05 & 0.56 & 0.05 \\
&$\frac{M(top^{2nd}_{cand})+M(W^{2nd}_{cand})}{M(H_{cand})}>7.5$ & 5.82 & 0.03 & 0.17 & 0.02 \\
&$ \Delta R (T' j^{6})>4.5$ & 3.62 & $3\times 10^{-3}$ & 0.03 &  $3\times 10^{-3}$ \\
&$105~\text{GeV}/c^{2} <M(H_{cand})<145~\text{GeV}/c^{2}$ & 2.97 & $1\times 10^{-3}$ & 0.01 &  $4\times 10^{-4}$ \\
%&$\frac{\pt(H_{cand})+\pt(top_{cand})}{H_T} > 0.7 $ & 2.08 & $3\times 10^{-4}$ & $4\times 10^{-3}$ & $4\times 10^{-4}$ \\
%&$-0.02 <\frac{M(H_{cand})-M(top^{1st}_{cand})}{M(H_{cand})+M(top^{1st}_{cand})}<  0.38$ & 1.97 & $3\times 10^{-4}$ & $3\times 10^{-3}$ & $4\times 10^{-4}$ \\
%&$p_{T}(H_{cand})>180$\GeVc \&\& $p_{T}(top_{cand})>180$\GeVc  & 8.01 & 0.01 & 0.34 & 0.02 \\
%&$2.2 < \Delta R (W_{cand} H_{cand}) < 3.5$ & 7.42 & $9\times 10^{-3}$ & 0.27 & 0.01 \\
%&$\frac{\pt(H_{cand})+\pt(top_{cand})}{H_T} > 0.65 $ & 6.51 & $6\times 10^{-3}$ & 0.19 & 0.01 \\
%&$0.3<\frac{M(H_{cand})+M(top_{cand})}{M(H_{cand}+top_{cand})}<0.5$ & 5.87 & $3\times 10^{-3}$ & 0.15 & $7\times 10^{-3}$ \\
%&$|\eta(j^{6})|>1$ & 4.86 & $8\times 10^{-4}$ & 0.07 &  $3\times 10^{-3}$ \\
%&$\frac{M(top^{2nd}_{cand})}{M(H_{cand})}<0.4\times\frac{M(W^{2nd}_{cand})}{M(H_{cand})}-0.2$ & 4.12 & $7\times 10^{-4}$ & 0.03 &  $1\times 10^{-3}$ \\
%&$ \Delta R (T' j^{6})>4.6$ & 2.93 & $2\times 10^{-4}$ & 0.01 &  $1\times 10^{-3}$ \\
%&$100~\GeVcc <m(H_{cand})<150~\GeVcc$ & 2.67 & $2\times 10^{-4}$ & $7\times 10^{-3}$ &  $3\times 10^{-4}$ \\
\hline
\end{tabular}
}
\caption{Cumulative efficiencies, in \%, for signal and main background as a function of cuts.\label{tab:cutflow}}
\end{center}
\end{table*}\clearpage
%\begin{TOINCLUDE}Table with efficiencies of full selection. Plot with selection efficiency for all mass points.\end{TOINCLUDE}

\section{Background estimation from data}
\label{sec:bkg}

Description of difficulties related to use MC for estimation of backgrounds.

\subsection{Known difficulties and tried methods}
\label{sec:tried}

Brief description of tried methods. One paragraph for matrix method one paragraph for tight-loose method.

\begin{figure}[!Hhtbp]
  \begin{center}
    \includegraphics[width=0.3\textwidth]{figs/CMSlogo.png}
    \caption{$p_{T}(H)$ and $p_{T}(top)$ correlation for backgrounds [left] and signal [right].}
    \label{fig:HptTpt}
  \end{center}
\end{figure}\clearpage

%\begin{TOINCLUDE}2D plot of pT(H) and pT(top) to show correlation and table with ttbar+QCD and signal content in each region from ABCD method based on pT(H) and pT(top) \end{TOINCLUDE}

\subsection{Method}
\label{sec:bkgmet}

Description of control sample.

\subsection{Validation}
\label{sec:val}

Description of validation procedure.

Description on test to identify possible bias from chosen WP.

Exclusive WP additional test.

\begin{figure}[!Hhtbp]
  \begin{center}
    \includegraphics[width=0.3\textwidth]{figs/CMSlogo.png}
    \caption{Left: Distribution of the agreement between control sample and analysis sample in data at early stage of the selection as a function of the $\chi^2$ selection. The y-axis represents the chi2/ndf from a shape comparison made in data between control sample and analysis sample. A clear minimum is observed at 620. Right: Corresponding p-value of the $\chi^{2}$ test, showing the compatibility with null hypothesis in the comparison between control and signal samples.}
    \label{fig:optchi2}
  \end{center}
\end{figure}\clearpage

\begin{figure}[!Hhtbp]
  \begin{center}
    \includegraphics[width=0.3\textwidth]{figs/CMSlogo.png}
    \caption{Distribution of the 5-jets invariant mass in the control sample for $t\bar{t}$ for exclusive regions of $\chi^{2}$. Combinations with small values of $\chi^{2}$ tend to populate $M5J$ between $[600,1000]$}
    \label{fig:Chi2Regions}
  \end{center}
\end{figure}\clearpage

\begin{figure}[!Hhtbp]
  \begin{center}
    \includegraphics[width=0.3\textwidth]{figs/CMSlogo.png}
    \caption{Number of Combinations entering the control sample for $t\bar{t}$ and QCD HT 500-1000 for $\chi^{2}<140$. The majority of events entering the control sample only enter with one combination.}
    \label{fig:NComb}
  \end{center}
\end{figure}\clearpage

\begin{figure}[!Hhtbp]
  \begin{center}
    \includegraphics[width=0.3\textwidth]{figs/CMSlogo.png}
    \caption{Signal contamination in the control region comparing 5-jets invariant mass between data and signal.}
    \label{fig:SigContamination}
  \end{center}
\end{figure}\clearpage

\begin{figure}[!Hhtbp]
  \begin{center}
    \includegraphics[width=0.3\textwidth]{figs/CMSlogo.png}
    \caption{5-jets invariant mass in the control sample for different b-tagging working point for $t\bar{t}$ Monte Carlo samples within 4 stages of selection: A [top left], B [top right], C [bottom left] and D [bottom right]. The 3 working points are given in different color. Within statistical error, the 3 shapes are in agreement at all stages. A deep can be noticed around 700 GeV/$c^{2}$. $t\bar{t}$ MC as all the other Monte-Carlo samples are purely used for illustration. The shape is not visible when the same exercise is made on the sum of MC background or in the data.}
    \label{fig:StageWPttbar}
  \end{center}
\end{figure}\clearpage

\begin{figure}[!Hhtbp]
  \begin{center}
    \includegraphics[width=0.3\textwidth]{figs/CMSlogo.png}
    \caption{5-jets invariant mass in the control sample for different b-tagging working point for QCD Monte Carlo samples within 4 stages of selection: A [top left], B [top right], C [bottom left] and D [bottom right]. The 3 working points are given in different color. Quickly a lack of statistics is visible.}
    \label{fig:StageWPQCD}
  \end{center}
\end{figure}\clearpage

\begin{figure}[!Hhtbp]
  \begin{center}
    \includegraphics[width=0.3\textwidth]{figs/CMSlogo.png}
    \caption{5-jets invariant mass in the control sample for different b-tagging working point for the weighted sum of background Monte Carlo samples within 4 stages of selection: A [top left], B [top right], C [bottom left] and D [bottom right]. The 3 working points are given in different color. Within statistical error, the 3 shapes are in agreement at all stages.}
    \label{fig:StageWPSum}
  \end{center}
\end{figure}\clearpage

\begin{figure}[!Hhtbp]
  \begin{center}
    \includegraphics[width=0.3\textwidth]{figs/CMSlogo.png}
    \caption{5-jets invariant mass in the control sample for different b-tagging working point in data within 3 stages of selection: A [top left], B [top right] and C [bottom]. The 3 working points are given in different color. Within statistical error, the 3 shapes are in agreement at all stages.}
    \label{fig:StageWPData}
  \end{center}
\end{figure}\clearpage

\begin{figure}[!Hhtbp]
  \begin{center}
    \includegraphics[width=0.3\textwidth]{figs/CMSlogo.png}
    \caption{5-jets invariant mass in the control sample for different b-tagging working point, in exclusive regions, for $t\bar{t}$ Monte Carlo samples within 4 stages of selection: A [top left], B [top right], C [bottom left] and D [bottom right]. The 3 working points are given in different color. Within statistical error, the 3 shapes are in agreement at all stages. A deep can be noticed around 700 GeV/$c^{2}$. $t\bar{t}$ MC as all the other Monte-Carlo samples are purely used for illustration. The shape is not visible when the same exercise is made on the sum of MC background or in the data.}
    \label{fig:StageExWPttbar}
  \end{center}
\end{figure}\clearpage

\begin{figure}[!Hhtbp]
  \begin{center}
    \includegraphics[width=0.3\textwidth]{figs/CMSlogo.png}
    \caption{5-jets invariant mass in the control sample for different b-tagging working point, in exclusive regions, for QCD Monte Carlo samples within 4 stages of selection: A [top left], B [top right], C [bottom left] and D [bottom right]. The 3 working points are given in different color. Quickly a lack of statistics is visible.}
    \label{fig:StageExWPQCD}
  \end{center}
\end{figure}\clearpage

\begin{figure}[!Hhtbp]
  \begin{center}
    \includegraphics[width=0.3\textwidth]{figs/CMSlogo.png}
    \caption{5-jets invariant mass in the control sample for different b-tagging working point, in exclusive regions, for the weighted sum of background Monte Carlo samples within 4 stages of selection: A [top left], B [top right], C [bottom left] and D [bottom right]. The 3 working points are given in different color. Within statistical error, the 3 shapes are in agreement at all stages.}
    \label{fig:StageExWPSum}
  \end{center}
\end{figure}\clearpage

\begin{figure}[!Hhtbp]
  \begin{center}
    \includegraphics[width=0.3\textwidth]{figs/CMSlogo.png}
    \caption{5-jets invariant mass in the control sample for different b-tagging working point, in exclusive regions, in data within 3 stages of selection: A [top left], B [top right] and C [bottom]. The 3 working points are given in different color. Within statistical error, the 3 shapes are in agreement at all stages.}
    \label{fig:StageExWPData}
  \end{center}
\end{figure}\clearpage
%\begin{TOINCLUDE}Validation plots: MC, Data. Chi2 plots for minimization. Number of combinations in the control sample plot. Plot of M5J for different values of chi2. Table with signal contamination percentage in the control sample cut per cut and for all mass points.\end{TOINCLUDE}

\section{Systematics}
\label{sec:sys}

Description by paragraphs of each systematic source for background estimation and signal.

\begin{table*}[htbH]
\begin{center}
\begin{tabular}{|c|c|c|c|c|}
\hline 
Sample Name & \multicolumn{2}{c|}{b or c quark} & \multicolumn{2}{c|}{Light flavours} \\
\hline
 & up & down & up & down \\
\hline
$Tj\rightarrow tHj$ 600 GeV/$c^{2}$ & 7.70\% & 7.33\% & 0.84\% & 0.84\% \\
$Tj\rightarrow tHj$ 650 GeV/$c^{2}$ & 7.94\% & 7.54\% & 0.65\% & 0.64\% \\
$Tj\rightarrow tHj$ 700 GeV/$c^{2}$ & 7.75\% & 7.36\% & 0.56\% & 0.56\% \\
$Tj\rightarrow tHj$ 750 GeV/$c^{2}$ & 7.82\% & 7.43\% & 0.57\% & 0.57\% \\
$Tj\rightarrow tHj$ 800 GeV/$c^{2}$ & 7.89\% & 7.50\% & 0.65\% & 0.65\% \\
$Tj\rightarrow tHj$ 850 GeV/$c^{2}$ & 8.14\% & 7.72\% & 0.67\% & 0.66\% \\
$Tj\rightarrow tHj$ 900 GeV/$c^{2}$ & 8.34\% & 7.89\% & 0.68\% & 0.68\% \\
$Tj\rightarrow tHj$ 950 GeV/$c^{2}$ & 8.58\% & 8.10\% & 0.65\% & 0.65\% \\
$Tj\rightarrow tHj$ 1000 GeV/$c^{2}$ & 8.66\% & 8.17\% & 0.66\% & 0.66\% \\
\hline
\end{tabular}
\caption{B-tagging uncertainties for signal samples\label{tab:SFSys}}
\end{center}
\end{table*}\clearpage

\begin{table*}[htbH]
\begin{center}
\begin{tabular}{|c|c|c|c|c|}
\hline 
Sample Name & \multicolumn{2}{c|}{JER} & \multicolumn{2}{c|}{JES} \\
\hline
 & up & down & up & down \\
\hline
$Tj\rightarrow tHj$ 600 GeV/$c^{2}$ & 0.28\% & 0.19\% & 11.48\% & 12.32\% \\
$Tj\rightarrow tHj$ 650 GeV/$c^{2}$ & 0.55\% & 1.71\% & 10.01\% & 11.32\% \\
$Tj\rightarrow tHj$ 700 GeV/$c^{2}$ & 0.63\% & 0.05\% & 6.32\% & 9.33\% \\
$Tj\rightarrow tHj$ 750 GeV/$c^{2}$ & 0.29\% & 0.14\% & 5.37\% & 9.50\% \\
$Tj\rightarrow tHj$ 800 GeV/$c^{2}$ & 0.0\% & 0.64\% & 5.24\% & 8.65\% \\
$Tj\rightarrow tHj$ 850 GeV/$c^{2}$ & 0.29\% & 0.68\% & 5.99\% & 9.40\% \\
$Tj\rightarrow tHj$ 900 GeV/$c^{2}$ & 0.42\% & 0.69\% & 6.04\% & 8.53\% \\
$Tj\rightarrow tHj$ 950 GeV/$c^{2}$ & 0.95\% & 0.33\% & 5.52\% & 8.53\% \\
$Tj\rightarrow tHj$ 1000 GeV/$c^{2}$ & 0.66\% & 0.20\% & 7.70\% & 7.24\% \\
\hline
\end{tabular}
\caption{JEC uncertainties for signal samples\label{tab:JECSys}}
\end{center}
\end{table*}\clearpage

\begin{table*}[htbH]
\begin{center}
\begin{tabular}{|c|c|c|c|c|c|}
\hline 
Sample Name & \multicolumn{2}{c|}{CTEQ6.6} & \multicolumn{2}{c|}{MSTW2008} & NNPDF2.0\\
\hline
 & up & down & up & down & up,down \\
\hline
$Tj\rightarrow tHj$ 600 GeV/$c^{2}$ & 2.11\% & 1.61\% & 2.78\% & 1.94\% & 1.97\% \\
$Tj\rightarrow tHj$ 650 GeV/$c^{2}$ & 2.17\% & 1.61\% & 2.90\% & 2.00\% & 2.24\% \\
$Tj\rightarrow tHj$ 700 GeV/$c^{2}$ & 2.19\% & 1.62\% & 2.90\% & 2.00\% & 2.24\% \\
$Tj\rightarrow tHj$ 750 GeV/$c^{2}$ & 2.30\% & 1.68\% & 2.85\% & 1.97\% & 2.41\% \\
$Tj\rightarrow tHj$ 800 GeV/$c^{2}$ & 2.35\% & 1.72\% & 2.94\% & 2.04\% & 2.21\% \\
$Tj\rightarrow tHj$ 850 GeV/$c^{2}$ & 2.45\% & 1.73\% & 2.96\% & 2.07\% & 2.37\% \\
$Tj\rightarrow tHj$ 900 GeV/$c^{2}$ & 2.62\% & 1.81\% & 3.04\% & 2.11\% & 2.70\% \\
$Tj\rightarrow tHj$ 950 GeV/$c^{2}$ & 2.78\% & 1.88\% & 3.11\% & 2.18\% & 2.90\% \\
$Tj\rightarrow tHj$ 1000 GeV/$c^{2}$ & 2.76\% & 1.88\% & 3.09\% & 2.15\% & 2.86\% \\
\hline
\end{tabular}
\caption{PDF+$\alpha_{s}$ uncertainties for signal samples\label{tab:PDFsys}}
\end{center}
\end{table*}\clearpage

\begin{table*}[htbH]
\begin{center}
\begin{tabular}{|c|c|c|c|c|c|c|}
\hline 
Sample Name & \multicolumn{2}{c|}{b-tagging} & \multicolumn{2}{c|}{JEC} & \multicolumn{2}{c|}{PDF+$\alpha_{S}$}\\
\hline
 & up & down & up & down & up & down \\
\hline
$Tj\rightarrow tHj$ 600 GeV/$c^{2}$ & 7.52\% & 7.16\% & 18.2\% & 17.3\% & 3.60\% & 3.60\% \\
$Tj\rightarrow tHj$ 650 GeV/$c^{2}$ & 7.84\% & 7.44\% & 10.9\% & 7.7\% & 3.11\% & 2.08\% \\
$Tj\rightarrow tHj$ 700 GeV/$c^{2}$ & 7.73\% & 7.34\% & 8.7\% & 4.6\% & 3.20\% & 2.34\% \\
$Tj\rightarrow tHj$ 750 GeV/$c^{2}$ & 7.68\% & 7.31\% & 9.8\% & 2.3\% & 3.15\% & 3.15\% \\
$Tj\rightarrow tHj$ 800 GeV/$c^{2}$ & 7.80\% & 7.41\% & 8.2\% & 5.3\% & 3.07\% & 2.42\% \\
$Tj\rightarrow tHj$ 850 GeV/$c^{2}$ & 8.12\% & 7.70\% & 6.8\% & 4.6\% & 3.06\% & 2.44\% \\
$Tj\rightarrow tHj$ 900 GeV/$c^{2}$ & 8.24\% & 7.80\% & 9.4\% & 5.5\% & 3.18\% & 2.45\% \\
$Tj\rightarrow tHj$ 950 GeV/$c^{2}$ & 8.60\% & 8.12\% & 10.0\% & 6.0\% & 3.22\% & 2.70\% \\
$Tj\rightarrow tHj$ 1000 GeV/$c^{2}$ & 8.73\% & 8.23\% & 10.1\% & 9.1\% & 3.18\% & 2.87\% \\
\hline
\end{tabular}
\end{center}
\caption{Summary of uncertainties for signal samples\label{tab:sys}}
\end{table*}\clearpage

\begin{table*}[htbH]
\begin{center}
\begin{tabular}{|c|c|c|}
\hline 
Systematics Name & Signal & Background \\
\hline
Theory & 5\% & --\\
PDF & +2.90\% / -2.24\% & --\\
Luminosity & 2.6\% & --\\
Trigger & 10\% & --\\
B-tag & +7.77\% / - 7.38\% & --\\
JEC & +6.35\% / - 9.33\% & --\\
Background Shape determination & -- & 12\%\\
Background Normalization & -- & 20\%\\
\hline
\end{tabular}
\caption{Summary of uncertainties in case of signal mass point at
  700 GeV/$c^{2}$ and for background.\label{tab:sys700}}
\end{center}
\end{table*}\clearpage
%\begin{TOINCLUDE}Tables for PDF, b-tagging and JEC systematics. Table with full systematics.\end{TOINCLUDE}

\section{Results}
\label{sec:res}

Discussion of results and interpretation.

\begin{table*}[htbH]
\begin{center}
\begin{tabular}{|c|c|c|c|}
\hline 
T' Mass GeV/$c^{2}$ & Signal & Background & Observed Data\\
\hline 
600 & 3.57 & 5.14 & -- \\
650 & 6.03 & 8.69 & -- \\
700 & 8.00 & 12.10 & -- \\
750 & 7.61 & 11.22 & -- \\
800 & 7.32 & 11.36 & -- \\
850 & 7.00 & 9.96 & -- \\
900 & 5.71 & 9.58 & -- \\
950 & 4.33 & 9.44 & -- \\
1000 & 3.36 & 9.44 & -- \\
\hline
\end{tabular}
\caption{Expected number of events for the signal, estimated
  background and observed data after full selection \label{tab:ExpEvts}
}
\end{center}
\end{table*}\clearpage

\begin{figure}[!Hhtbp]
  \begin{center}
    \includegraphics[width=0.3\textwidth]{figs/CMSlogo.png}
    \caption{$M_{jjjjj}$ after full selection, estimated background from data and signal mass point of 700 GeV/$c^{2}$}
    \label{fig:FinalPlot}
  \end{center}
\end{figure}\clearpage

\begin{figure}[!Hhtbp]
  \begin{center}
    \includegraphics[width=0.3\textwidth]{figs/CMSlogo.png}
    \caption{Expected and observed limits}
    \label{fig:Lim}
  \end{center}
\end{figure}\clearpage
%\begin{TOINCLUDE}Expected and observed yields table. Exclusion limits plot. M5J plot for data with estimated background and signal.\end{TOINCLUDE}