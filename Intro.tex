\chapter*{Introduction}

The seek to understand the nature of matter has been a long quest in human history. The ultimate objective to find the mechanisms that govern matter behavior and its composition has directed physics evolution and its discoveries. Concerning the mechanisms, four forces has been found in nature that dictate how different matter components interact between them. On the other hand, in the search for the fundamental components of matter the periodic table has been constructed. But even deeper in matter structure a set of fundamental particles, without inner structure, have been also found. 

One of the most astonishing findings has been the mechanics of the microscopic world. Starting the XX century, physics has described for the first time the mechanics of very small bodies. Such mechanics have been a challenge to human understanding of nature due to their radical difference to macroscopic mechanics. All these discoveries, have been gathered and combined into a modern theory that describe the subatomic world: The Standard Model of particle physics. 

This Standard Model has been confronted to nature during the last century with astonishing results. Up to now, almost all Standard Model predictions have been confirmed by several experiments. These confirmations have been a gigantic physics, mathematics and engineering task.

In particular, the LHC (Large Hadron Collider) has confirmed the last principal missing piece of the model. The LHC is the most powerful particle collider conceived and created by mankind up to our days. However it is a very big and complex machine, it operates with the same fundamental ideas of Rutherford of colliding particles ones against others to look at their inner structure. In 2012, with the finding of the Higgs boson, by CMS and ATLAS experiments, a long search of around half century concluded.

Nonetheless, the Standard Model has limitations. Even though it constitutes the most complete picture of nature we have, it is not able to explain all phenomena in nature. LHC, as CMS and ATLAS experiments, have been also designed to look for extensions of the Standard Model designed to fill its lacunae. 

This doctoral thesis is an effort to look for a prediction of an extension of the Standard Model using CMS data collected from collisions delivered by LHC during 2012. 