\chapter*{Introduction}

The seek to understand the nature of matter has been a long quest in human history. The ultimate objective to find the mechanisms that govern matter behavior and its composition has directed physics evolution and its discoveries. Concerning the mechanisms, four forces has been found in nature that dictate how different matter components interact between them. On the other hand, in the search for the fundamental components of matter the periodic table has been constructed. But even deeper in matter structure a set of fundamental particles, without known inner structure, have been also found. 

One of the most astonishing findings has been the mechanics of the microscopic world. At the beginning of the XX century, physics has described for the first time the mechanics of very small bodies. Such mechanics have been a challenge to human understanding of nature due to their radical difference to macroscopic mechanics. All these discoveries, have been gathered and combined into a modern theory that describe the subatomic world: the standard model of particle physics. 

This standard model has been confronted to nature during the last century with astonishing results. Up to now, almost all standard model predictions have been confirmed by several experiments. These confirmations have been a gigantic physics, mathematics and engineering task.

In particular, the LHC (Large Hadron Collider) has confirmed the last principal missing piece of the model. The LHC is the most powerful particle collider conceived and created by mankind up to our days. However it is a very big and complex machine, it operates with the same fundamental ideas of Rutherford of colliding particles ones against others to look at their inner structure. In 2012, with the finding of the Higgs boson, by ATLAS and CMS experiments, a long search of around half century concluded.

Nonetheless, the standard model has limitations. Even though it constitutes the most complete picture of nature we have, it is not able to explain all phenomena in nature. LHC, as ATLAS and CMS experiments, have been also designed to look for extensions of the standard model designed to fill its lacunae. 

This doctoral thesis is an effort to look for a prediction of an extension of the standard model using CMS data collected from collisions delivered by LHC during 2012. It has been structured in five chapters.

In the first chapter, I describe the standard model and its predictions. I also describe in detail the top quark and the Higgs boson, their properties, how they have been measured and their production according to the standard model. Afterward, I proceed to describe an effective extension of the model and one of the new particles predicted by it. 

In the second chapter, the experimental set up is detailed. In first instance, the Large Hadron Collider is described and the experiments based on it. Then, I pay special attention to the description of CMS experiment. This chapter is closed by the description of the techniques used in CMS to reconstruct particles from proton-proton collision events. 

The third chapter is devoted to cast the phenomenological study I performed during my first year of thesis. I describe in detail the feasibility study done for a possible search in data of a Vector Like top partner. This chapter serve as theoretical motivation to the data analysis describe after in chapter 5.

During the first two years of my thesis I have performed different tasks associated to the Monte-Carlo simulations of proton-proton collisions. This work is drawn up in the fourth chapter. I discuss there, what are Monte-Carlo simulations, how they are used in particle physics and the different packages in the market to produce them. I also show the specific task I have performed for CMS collaboration on the matter.

The fifth chapter is the main product of my work during my thesis. It contains the data analysis I performed using CMS data for proton-proton collisions provided by the LHC at 8 TeV. I present in detail all the techniques used in the analysis and its results.