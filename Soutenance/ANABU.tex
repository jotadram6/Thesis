\subsection{Analysis}

\begin{frame}{Datasets - Backgrounds}
\vspace{-.2cm}

%\begin{table*}[htbH]
\begin{center}
\resizebox{\textwidth}{!}{
\begin{tabular}{|c|c|c|}
\hline 
Samples & Cross-Section (pb) & Number of events\\
\hline
QCD\_Pt-120to170\_TuneZ2star\_8TeV\_pythia6 & 16\(\times 10^4\) & 5.9M\\
QCD\_Pt-170to300\_TuneZ2star\_8TeV\_pythia6 & 34\(\times 10^3\) & 5.8M\\
QCD\_Pt-300to470\_TuneZ2star\_8TeV\_pythia6 & 18\(\times 10^2\) & 5.9M\\ 
QCD\_Pt-470to600\_TuneZ2star\_8TeV\_pythia6 & 114 & 3.9M\\
QCD\_Pt-600to800\_TuneZ2star\_8TeV\_pythia6 & 27 & 3.9M\\
QCD\_Pt-800to1000\_TuneZ2star\_8TeV\_pythia6 & 3.5 & 3.9M\\
QCD\_HT-500To1000\_TuneZ2star\_8TeV-madgraph-pythia6 & 84\(\times 10^2\) & 30M\\ 
QCD\_HT-1000ToInf\_TuneZ2star\_8TeV-madgraph-pythia6 & 2\(\times 10^2\) & 14M\\ 
DYToCC\_M\_50\_TuneZ2star\_8TeV\_pythia6 & 31\(\times 10^2\) & 2M\\
DYToBB\_M\_50\_TuneZ2star\_8TeV\_pythia6 & 38\(\times 10^2\) & 2M\\
TTJets\_MSDecays\_central\_TuneZ2star\_8TeV-madgraph-tauola & 247.7 [NNLO] & 62M\\
%TT\_CT10\_TuneZ2star\_8TeV-powheg-tauola & 247.7 [NNLO] & 22M\\
T\_tW-channel-DR\_TuneZ2star\_8TeV-powheg-tauola & 11.1 [NNLO] &497k\\
T\_s-channel\_TuneZ2star\_8TeV-powheg-tauola & 3.79 [NNLO] & 260k\\
T\_t-channel\_TuneZ2star\_8TeV-powheg-tauola & 54.9 [NNLO] & 3.7M\\
Tbar\_tW-channel-DR\_TuneZ2star\_8TeV-powheg-tauola & 11.1 [NNLO] & 493k\\
Tbar\_s-channel\_TuneZ2star\_8TeV-powheg-tauola & 1.76 [NNLO] & 140k\\
Tbar\_t-channel\_TuneZ2star\_8TeV-powheg-tauola & 29.7 [NNLO] & 1.9M\\
WZ\_TuneZ2star\_8TeV\_pythia6\_tauola & 33.6 [NLO] & 10M\\
ZZ\_TuneZ2star\_8TeV\_pythia6\_tauola & 7.6 [NLO] & 9.8M\\
WW\_TuneZ2star\_8TeV\_pythia6\_tauola & 56 [NLO] & 10M\\
TTH\_Inclusive\_M-125\_8TeV\_pythia6 & 0.13 [NLO] & 100K\\
\hline
\end{tabular}
}
%\caption{List of Monte-Carlo background samples used in the analysis, their corresponding cross-section and their number of events.\label{tab:MCbkg}}
\end{center}
%\end{table*}

\end{frame}

\begin{frame}{Datasets - Signal}
\vspace{-.2cm}

%\begin{table*}[htbH]
\begin{center}
\resizebox{\textwidth}{!}{
\begin{tabular}{|c|c|c|c|}
\hline 
Sample & \Tp~Mass & Cross-Section & Number of events\\
            & (GeV$/c^{2}$) &  (fb) & \\
\hline
TprimeJetToTH\_M-600\_TuneZ2star\_8TeV-madgraph\_tauola & 600 & 215.4 & 95K \\
TprimeJetToTH\_M-650\_TuneZ2star\_8TeV-madgraph\_tauola & 650 & 177.8 & 99K \\
TprimeJetToTH\_M-700\_TuneZ2star\_8TeV-madgraph\_tauola & 700 & 143.7 & 99K \\
TprimeJetToTH\_M-750\_TuneZ2star\_8TeV-madgraph\_tauola & 750 & 118.6 & 99K \\
TprimeJetToTH\_M-800\_TuneZ2star\_8TeV-madgraph\_tauola & 800 & 100 & 96K \\
TprimeJetToTH\_M-850\_TuneZ2star\_8TeV-madgraph\_tauola & 850 & 84.3 & 99K \\
TprimeJetToTH\_M-900\_TuneZ2star\_8TeV-madgraph\_tauola & 900 & 72.6 & 99K \\
TprimeJetToTH\_M-950\_TuneZ2star\_8TeV-madgraph\_tauola & 950 & 62.6 & 96K \\
TprimeJetToTH\_M-1000\_TuneZ2star\_8TeV-madgraph\_tauola & 1000 & 53.9 & 99K \\
\hline
\end{tabular}
}
%\caption{List of Monte-Carlo signal samples used in the analysis, their corresponding cross-section and \Tp~mass.\label{tab:MCsig}}
\end{center}
%\end{table*}

\end{frame}

\begin{frame}{PU corrections}
\vspace{-.2cm}

\begin{figure}[!Hhtbp]
  \begin{center}
    \includegraphics[width=0.49\textwidth]{../figs/Ana/DataPU40.png}
    \includegraphics[width=0.49\textwidth]{../figs/Ana/MCPU40.png}\\
    \includegraphics[width=0.5\textwidth]{../figs/Ana/WeightPU40.png}
    \begin{block}{}
      \tiny \centering Pileup for data [up-left], MC S10 [up-right] and ratio between them [bottom].
    \end{block}
    %\caption{Pileup for data [up-left], MC S10 [up-right] and ratio between them [bottom].}
    %\label{fig:PU_distros}
  \end{center}
\end{figure}

\end{frame}

\begin{frame}{Six leading jets \pt}
\vspace{-.2cm}
\begin{figure}[!Hhtbp]
  \begin{center}
    \includegraphics[width=0.35\textwidth, height=0.65\textheight]{../figs/Ana/jet1pt.png}
    \includegraphics[width=0.35\textwidth, height=0.65\textheight]{../figs/Ana/jet2pt.png}
    \includegraphics[width=0.35\textwidth, height=0.65\textheight]{../figs/Ana/jet3pt.png}
    %\caption{Distribution of transverse momentum of the 6 leading jets. The gray band represents the statistical uncertainties from the sum of the MC background. Reasonable agreement is observed, with the multijet process as the dominant process at this stage. Normalization of samples was done to the 19.7~fb$^{-1}$.}
    %\label{fig:6jpt}
  \end{center}
\end{figure}

\vspace{-.2cm}
    \begin{block}{}
      \tiny \centering Distribution of transverse momentum of the 3 leading jets. The gray band represents the statistical uncertainties from the sum of the MC background. Reasonable agreement is observed, with the multijet process as the dominant process at this stage. Normalization of samples was done to the 19.7~fb$^{-1}$.
    \end{block}

\end{frame}

\begin{frame}{}
\vspace{-.2cm}
\begin{figure}[!Hhtbp]
  \begin{center}
    \includegraphics[width=0.35\textwidth, height=0.65\textheight]{../figs/Ana/jet4pt.png}
    \includegraphics[width=0.35\textwidth, height=0.65\textheight]{../figs/Ana/jet5pt.png}
    \includegraphics[width=0.35\textwidth, height=0.65\textheight]{../figs/Ana/jet6pt.png}
    %\caption{Distribution of transverse momentum of the 6 leading jets. The gray band represents the statistical uncertainties from the sum of the MC background. Reasonable agreement is observed, with the multijet process as the dominant process at this stage. Normalization of samples was done to the 19.7~fb$^{-1}$.}
    %\label{fig:6jpt}
  \end{center}
\end{figure}

\vspace{-.2cm}
    \begin{block}{}
      \tiny \centering Distribution of transverse momentum of the 4th, 5th and 6th leading jets. The gray band represents the statistical uncertainties from the sum of the MC background. Reasonable agreement is observed, with the multijet process as the dominant process at this stage. Normalization of samples was done to the 19.7~fb$^{-1}$.
    \end{block}

\end{frame}

\begin{frame}{Six leading jets $\eta$}
\vspace{-.2cm}
\begin{figure}[!Hhtbp]
  \begin{center}
    \includegraphics[width=0.35\textwidth, height=0.65\textheight]{../figs/Ana/jet1eta.png}
    \includegraphics[width=0.35\textwidth, height=0.65\textheight]{../figs/Ana/jet2eta.png}
    \includegraphics[width=0.35\textwidth, height=0.65\textheight]{../figs/Ana/jet3eta.png}
    %\caption{Distribution of $\eta$ of the 6 leading jets. The gray band represents the statistical uncertainties from the sum of the MC background. Reasonable agreement is observed, with the multijet process as the dominant process at this stage. Normalization of samples was done to luminosity.}
    %\label{fig:6jeta}
  \end{center}
\end{figure}

\vspace{-.2cm}
    \begin{block}{}
      \tiny \centering Distribution of $\eta$ of the 3 leading jets. The gray band represents the statistical uncertainties from the sum of the MC background. Reasonable agreement is observed, with the multijet process as the dominant process at this stage. Normalization of samples was done to luminosity.
    \end{block}

\end{frame}

\begin{frame}{}
\vspace{-.2cm}
\begin{figure}[!Hhtbp]
  \begin{center}
    \includegraphics[width=0.35\textwidth, height=0.65\textheight]{../figs/Ana/jet4eta.png}
    \includegraphics[width=0.35\textwidth, height=0.65\textheight]{../figs/Ana/jet5eta.png}
    \includegraphics[width=0.35\textwidth, height=0.65\textheight]{../figs/Ana/jet6eta.png}
    %\caption{Distribution of $\eta$ of the 6 leading jets. The gray band represents the statistical uncertainties from the sum of the MC background. Reasonable agreement is observed, with the multijet process as the dominant process at this stage. Normalization of samples was done to luminosity.}
    %\label{fig:6jeta}
  \end{center}
\end{figure}

\vspace{-.2cm}
    \begin{block}{}
      \tiny \centering Distribution of $\eta$ of the 4th, 5th and 6th leading jets. The gray band represents the statistical uncertainties from the sum of the MC background. Reasonable agreement is observed, with the multijet process as the dominant process at this stage. Normalization of samples was done to luminosity.
    \end{block}

\end{frame}

\begin{frame}{B-tagging working point study}
\vspace{-.2cm}
\begin{table}[htbH]
\begin{center}
\resizebox{\textwidth}{!}{
\begin{tabular}{| c || c | c | c | c | c | c |}
\hline 
\textit{At least} & $\epsilon(S)$ [\%] & $\epsilon(t\bar{t})$ [\%] & $\epsilon(\text{QCD\_HT-500To1000})$ [\%] & $\epsilon(\text{QCD\_HT-1000ToInf})$ [\%] & $\frac{S}{B}\times 10^{3}$ & $\frac{S}{\sqrt{S+B}}\times 10^{2}$ \\
\hline
3 CSVL                       & $65 \pm 0.4$  & $38 \pm 0.04$  & $6 \pm 0.02$    & $7 \pm 0.02$     & $0.4 \pm 0.005$  & $24.8 \pm 0.3$ \\
3 CSVM                       & $31 \pm 0.4$  & $8 \pm 0.02$   & $1 \pm 0.01$    & $0.6 \pm 0.01$   & $1.8 \pm 0.05$   & $38.2 \pm 0.8$ \\
1 CSVL and 2 CSVM            & $55 \pm 0.4$  & $27 \pm 0.03$  & $2 \pm 0.01$    & $2 \pm 0.01$     & $0.8 \pm 0.01$   & $33.7 \pm 0.5$ \\
2 CSVL and 1 CSVM            & $64 \pm 0.4$  & $37 \pm 0.04$  & $5 \pm 0.02$    & $5 \pm 0.02$     & $0.5 \pm 0.007$  & $28.3 \pm 0.3$  \\
2 CSVM and 1 CSVT            & $29 \pm 0.4$  & $8 \pm 0.02$   & $0.5 \pm 0.006$ & $0.5 \pm 0.006$  & $1.9 \pm 0.05$   & $38.4 \pm 0.8$  \\
1 CSVM and 2 CSVT            & $22 \pm 0.4$  & $5 \pm 0.02$   & $0.3 \pm 0.005$ & $0.3 \pm 0.003$  & $2.2 \pm 0.07$   & $35.8 \pm 0.9$  \\
3 CSVT                       & $9 \pm 0.2$   & $1 \pm 0.01$   & $0.1 \pm 0.003$ & $0.09 \pm 0.002$ & $3.1 \pm 0.2$    & $27.3 \pm 1.1$  \\
1 CSVL and 2 CSVT            & $33 \pm 0.4$  & $13 \pm 0.03$  & $0.9 \pm 0.01$  & $0.8 \pm 0.007$  & $1.1 \pm 0.03$   & $31.5 \pm 0.6$  \\
2 CSVL and 1 CSVT            & $57 \pm 0.4$  & $30 \pm 0.03$  & $3 \pm 0.02$    & $3 \pm 0.01$     & $0.7 \pm 0.01$   & $31.5 \pm 0.4$  \\
1 CSVL and 1 CSVM and 1 CSVT &  $51 \pm 0.4$ &  $24 \pm 0.03$ & $2 \pm 0.01$    & $2 \pm 0.01$     & $0.9 \pm 0.02$   & $30.8 \pm 0.4$  \\
\hline
\end{tabular}
}
%\caption{Comparative study of different possible combinations to require at least 3 b-tagged jets with CSVL, CSVM and CSVT working points. Efficiencies of cuts over signal and principal MC background samples are presented, as well as $\frac{S}{B}$ and $\frac{S}{S+B}$. High values of $\frac{S}{S+B}$ point to a good discrimination while keeping the signal efficiency high.\label{tab:BCutStudy}}
\end{center}
\end{table}

\vspace{-.2cm}
    \begin{block}{}
      \tiny \centering Comparative study of different possible combinations to require at least 3 b-tagged jets with CSVL, CSVM and CSVT working points. Efficiencies of cuts over signal and principal MC background samples are presented, as well as $\frac{S}{B}$ and $\frac{S}{S+B}$. High values of $\frac{S}{S+B}$ point to a good discrimination while keeping the signal efficiency high.
    \end{block}

\end{frame}

\begin{frame}{B-tagging efficiencies}
\vspace{-.2cm}
\begin{figure}[!Hhtbp]
  \begin{center}
    \includegraphics[width=0.5\textwidth, height=0.4\textheight]{../figs/Ana/ttbar_beff.png}
    \includegraphics[width=0.5\textwidth, height=0.4\textheight]{../figs/Ana/ttbar_ceff.png}\\
    \includegraphics[width=0.5\textwidth, height=0.4\textheight]{../figs/Ana/ttbar_leff.png}
    %\caption{CSVM b-tagging efficiency for b-jets [left], c-jets [center] and light jets [right] as function of \pt~and $\eta$ for \ttbar.}
    %\label{fig:ttbarBEff}
  \end{center}
\end{figure}

\vspace{-.5cm}
    \begin{block}{}
      \tiny \centering CSVM b-tagging efficiency for b-jets [left], c-jets [right] and light jets [bottom] as function of \pt~and $\eta$ for \ttbar.
    \end{block}

\end{frame}

\begin{frame}{Jet multiplicity after basic selection}
\vspace{-.2cm}
\begin{figure}[!Hhtbp]
  \begin{center}
    \includegraphics[width=0.48\textwidth]{../figs/Ana/Nj_Nm1.png}
    %\caption{Jet multiplicity for MC samples after requiring at least 3 CSVM b-tagged jets. The sum of MC samples is normalized to the integrated luminosity. Signal is overlaid.}
    %\label{fig:Nj}
  \end{center}
\end{figure}

\vspace{-.2cm}
    \begin{block}{}
      \tiny \centering Jet multiplicity for MC samples after requiring at least 3 CSVM b-tagged jets. The sum of MC samples is normalized to the integrated luminosity. Signal is overlaid.
    \end{block}

\end{frame}

\begin{frame}{B-tagging efficiencies}
\vspace{-.2cm}
\begin{figure}[!Hhtbp]
  \begin{center}
    \includegraphics[width=0.5\textwidth, height=0.4\textheight]{../figs/Ana/ttbar_beff.png}
    \includegraphics[width=0.5\textwidth, height=0.4\textheight]{../figs/Ana/ttbar_ceff.png}\\
    \includegraphics[width=0.5\textwidth, height=0.4\textheight]{../figs/Ana/ttbar_leff.png}
    %\caption{CSVM b-tagging efficiency for b-jets [left], c-jets [center] and light jets [right] as function of \pt~and $\eta$ for \ttbar.}
    %\label{fig:ttbarBEff}
  \end{center}
\end{figure}

\vspace{-.5cm}
    \begin{block}{}
      \tiny \centering CSVM b-tagging efficiency for b-jets [left], c-jets [center] and light jets [right] as function of \pt~and $\eta$ for \ttbar.
    \end{block}

\end{frame}

\begin{frame}{Reconstructed resonances by the $\chi^{2}$ sorting algorithm}
\vspace{-.2cm}

\begin{figure}[!Hhtbp]
  \begin{center}
    \includegraphics[width=0.35\textwidth]{../figs/Ana/TopMass_S700.png}
    \includegraphics[width=0.35\textwidth]{../figs/Ana/WMass_S700.png}
    \includegraphics[width=0.35\textwidth]{../figs/Ana/HiggsMass_S700.png}
    %\caption{Reconstructed top, \W~and \Hb~masses for the \Tp~mass point of 700 \GeVcc. The reconstructed masses and widths of the three resonances, $M^{reco}_{H}=124.92\pm0.26$~\GeVcc, $M^{reco}_{W}=85.06\pm0.26$~\GeVcc, $M^{reco}_{t}=179.02\pm0.42$~\GeVcc, $\sigma^{reco}_{H}=13.50\pm0.27$~\GeVcc, and $\sigma^{reco}_{W}=11.03\pm0.28$~\GeVcc~and $\sigma^{reco}_{t}=18.10\pm0.42$~\GeVcc. The corresponding values used for the reconstruction procedure are: $M_{H}=125$~\GeVcc, $M_{W}=84.06$~\GeVcc, $M_{t}=175.16$~\GeVcc, $\sigma_{H}=12.4$~\GeVcc, and $\sigma_{W}=10.12$~\GeVcc~and $\sigma_{t}=17.35$~\GeVcc.}
    %\label{fig:WHt}
  \end{center}
\end{figure}

\vspace{-.2cm}
    \begin{block}{}
      \tiny \centering Reconstructed top, \W~and \Hb~masses for the \Tp~mass point of 700 \GeVcc. The reconstructed masses and widths of the three resonances, $M^{reco}_{H}=124.92\pm0.26$~\GeVcc, $M^{reco}_{W}=85.06\pm0.26$~\GeVcc, $M^{reco}_{t}=179.02\pm0.42$~\GeVcc, $\sigma^{reco}_{H}=13.50\pm0.27$~\GeVcc, and $\sigma^{reco}_{W}=11.03\pm0.28$~\GeVcc~and $\sigma^{reco}_{t}=18.10\pm0.42$~\GeVcc. The corresponding values used for the reconstruction procedure are: $M_{H}=125$~\GeVcc, $M_{W}=84.06$~\GeVcc, $M_{t}=175.16$~\GeVcc, $\sigma_{H}=12.4$~\GeVcc, and $\sigma_{W}=10.12$~\GeVcc~and $\sigma_{t}=17.35$~\GeVcc.
    \end{block}

\end{frame}

\begin{frame}{}
\vspace{-.2cm}

\begin{columns}
\begin{column}{.50\textwidth}

\begin{figure}[!Hhtbp]
  \begin{center}
    %\includegraphics[width=1.0\textwidth]{../figs/Ana/Exclusive_Efficiency_V8.png}
    \includegraphics[width=1.0\textwidth]{../figs/Ana/Inclusive_Efficiency_V8.png}
    %\caption{Reconstruction efficiency by the $\chi^{2}$ algorithm of the Higgs boson, \W~boson, top quark and \Tp, as the ratio of the number of events where the particle was correctly reconstructed to the number of events where jets could be matched to partons [left] and to the total number of events [right]}
    %\label{fig:RecEff}
  \end{center}
\end{figure}

\vspace{-.2cm}
\begin{block}{}
\scriptsize \textbf{Figure}: Reconstruction efficiency by the $\chi^{2}$ algorithm of the Higgs boson, \W~boson, top quark and \Tp, as the ratio of the number of events where the particle was correctly reconstructed to the total number of events.
\end{block}
\end{column}

\begin{column}{.50\textwidth}

\begin{figure}[!Hhtbp]
  \begin{center}
    %\includegraphics[width=1.0\textwidth]{../figs/Ana/HundresdsMassChi2Tp.png}
    \includegraphics[width=1.0\textwidth]{../figs/Ana/FiftiesMassChi2Tp.png}
    %\caption{Reconstructed \Tp~mass for all mass points from the $\chi^{2}$ sorting algorithm after basic selection. Each mass point is normalized to luminosity and its corresponding cross section. A gaussian fit of these distributions will be presented afterward in section~\ref{sec:finalsel}, accompanied with a discussion about the resolution on the reconstruction of the \Tp.}
    %\label{fig:RecT}
  \end{center}
\end{figure}

\vspace{-.2cm}
\begin{block}{}
\scriptsize \textbf{Figure}: Reconstructed \Tp~mass for some mass points from the $\chi^{2}$ sorting algorithm after basic selection. Each mass point is normalized to luminosity and its corresponding cross section.
\end{block}
\end{column}

\end{columns}
\end{frame}

\begin{frame}{Selection based on reconstructed objects}
\vspace{-.2cm}

\begin{figure}[!Hhtbp]
  \begin{center}
    \includegraphics[width=0.45\textwidth]{../figs/Ana/chi2_SoB.png}
    %\caption{Efficiency of selection criterion $\chi^{2}<x$, with $x$ the cut value, as function of the cut value for \Tp~with $M=700$ \GeVcc, \ttbar~and QCD\_HT-500To1000 MC samples. Ratios between efficiency for signal and each background are also displayed.}
    %\label{fig:chi2cut}
  \end{center}
\end{figure}

\vspace{-.2cm}
    \begin{block}{}
      \tiny \centering Efficiency of selection criterion $\chi^{2}<x$, with $x$ the cut value, as function of the cut value for \Tp~with $M=700$ \GeVcc, \ttbar~and QCD\_HT-500To1000 MC samples. Ratios between efficiency for signal and each background are also displayed.
    \end{block}

\end{frame}

\begin{frame}{}
\vspace{-.2cm}

\begin{figure}[!Hhtbp]
  \begin{center}
    \includegraphics[width=0.5\textwidth]{../figs/Ana/DRbbNm1.png}
    \includegraphics[width=0.5\textwidth]{../figs/Ana/DRbb_SoB.png}
    %\caption{$\Delta R$ of the 2 b-tagged jets used to reconstruct the Higgs candidate after $\chi^{2}$ cut. The signal which is simply overlaid prefers low $\Delta R$ while backgrounds have larger distribution at higher value. The gray band represents the statistical uncertainties from the sum of the MC background. Normalization was done to luminosity~[right]. Efficiency of selection criterion $\Delta R(bb)<x$, with $x$ the cut value, as function of cut value for $M=700$ \GeVcc~signal sample, \ttbar~and QCD\_HT-500To1000 MC samples and ratios between efficiency for signal and each background~[left].}
    %\label{fig:DRbb}
  \end{center}
\end{figure}

\vspace{-.2cm}
    \begin{block}{}
      \tiny \centering $\Delta R$ of the 2 b-tagged jets used to reconstruct the Higgs candidate after $\chi^{2}$ cut. The signal which is simply overlaid prefers low $\Delta R$ while backgrounds have larger distribution at higher value. The gray band represents the statistical uncertainties from the sum of the MC background. Normalization was done to luminosity~[right]. Efficiency of selection criterion $\Delta R(bb)<x$, with $x$ the cut value, as function of cut value for $M=700$ \GeVcc~signal sample, \ttbar~and QCD\_HT-500To1000 MC samples and ratios between efficiency for signal and each background~[left].
    \end{block}

\end{frame}

\begin{frame}{}
\vspace{-.2cm}

\begin{figure}[!Hhtbp]
  \begin{center}
    \includegraphics[width=0.45\textwidth]{../figs/Ana/HMNm1.png}
    \includegraphics[width=0.45\textwidth]{../figs/Ana/HM_SoB.png}
    %\caption{Distribution for $M(H_{cand})$ for data and the sum of Monte Carlo samples~[left]. Efficiency of selection criterion $|M(H_{cand})-125|<x$, with $x$ the cut value, as function of cut value for the $M=700$ \GeVcc~signal sample, \ttbar~and QCD\_HT-500To1000 MC samples and ratios between efficiency for signal and each background~[right]. All others selection criteria are applied up to this one.}
    %\label{fig:HiggsMassDMC}
  \end{center}
\end{figure}

\vspace{-.2cm}
    \begin{block}{}
      \tiny \centering Distribution for $M(H_{cand})$ for data and the sum of Monte Carlo samples~[left]. Efficiency of selection criterion $|M(H_{cand})-125|<x$, with $x$ the cut value, as function of cut value for the $M=700$ \GeVcc~signal sample, \ttbar~and QCD\_HT-500To1000 MC samples and ratios between efficiency for signal and each background~[right]. All others selection criteria are applied up to this one.
    \end{block}

\end{frame}


\begin{frame}{}
\vspace{-.2cm}
\scriptsize

\begin{columns}
\begin{column}{.50\textwidth}
\vspace{-.2cm}
\begin{block}{}
\tiny
\textbf{Bin~0:} Higgs candidate reconstructed from two b-quarks coming each one from a top quark.\\
\textbf{Bin~1:} Higgs candidate reconstructed from Higgs boson \bbbar~system.\\
\textbf{Bin~2:} Higgs candidate reconstructed from one quark from the Higgs boson and one b-quark from a top quark.\\
\textbf{Bin~3:} Higgs candidate reconstructed from one quark from the Higgs boson and one quark from a \W~boson.\\
\textbf{Bin~4:} Higgs candidate reconstructed from one b-quark from the top quark and one quark from a \W~boson.\\
\textbf{Bin~5:} Higgs candidate reconstructed from two quarks coming from \W~bosons.\\
\textbf{Bin~6:} Higgs candidate reconstructed from one jet from the Higgs boson and one additional quark (not from Higgs boson, \W~boson nor top quark).\\
\textbf{Bin~7:} Higgs candidate reconstructed from one b-quark from a top quark and one additional quark (not from Higgs boson, \W~boson nor top quark).\\
\textbf{Bin~8:} Higgs candidate reconstructed from one jet from a \W~boson and one additional quark (not from Higgs boson, \W~boson nor top quark).\\
\textbf{Bin~9:} Higgs candidate reconstructed from two additional quarks (not from Higgs boson, \W~boson nor top quark).
\end{block}
\end{column}

\begin{column}{.50\textwidth}
\begin{figure}[!Hhtbp]
  \begin{center}\vspace{-.3cm}
    \includegraphics[width=0.8\textwidth, height=0.47\textheight]{../figs/Ana/HiggsRecoMCTruth_S700.png}\\\vspace{-.3cm}
    \includegraphics[width=0.8\textwidth, height=0.47\textheight]{../figs/Ana/HiggsRecoMCTruth_TTJets.png}
  \end{center}
\end{figure}
\end{column}

\end{columns}

\vspace{-.5cm}
\begin{block}{}
\tiny \textbf{Figure}: Higgs candidate reconstruction MC truth for signal (M=700~\GeVcc)~[left] and \ttbar~[right] MC samples. In signal the Higgs is reconstructed preferentially from two jets coming from the Higgs (bin 1) while for \ttbar~the Higgs candidate is reconstructed preferentially from a b-quark from a top and a quark from a \W~boson (bin 4). The study is performed after $\chi^{2}$ reconstruction, before cut over the $\chi^{2}$ variable.
\end{block}

\end{frame}


\begin{frame}{}
\vspace{-.2cm}

\begin{figure}[!Hhtbp]
  \begin{center}
    \includegraphics[width=0.5\textwidth]{../figs/Ana/M2ndTopNm1.png}
    %\caption{Second top mass distribution for data and the sum of the Monte Carlo samples. Selection criteria are applied up to Higgs mass cut. The second top is reconstructed from the two Higgs jets and the leading jet not used by the $\chi^{2}$ algorithm. The normalization was done to luminosity.}
    %\label{fig:2ndTM}
  \end{center}
\end{figure}

\vspace{-.2cm}
    \begin{block}{}
      \tiny \centering Second top mass distribution for data and the sum of the Monte Carlo samples. Selection criteria are applied up to Higgs mass cut. The second top is reconstructed from the two Higgs jets and the leading jet not used by the $\chi^{2}$ algorithm. The normalization was done to luminosity.
    \end{block}

\end{frame}

\begin{frame}{}
\vspace{-.2cm}

\begin{figure}[!Hhtbp]
  \begin{center}
    \includegraphics[width=0.5\textwidth]{../figs/Ana/M2HPNm1.png}
    \includegraphics[width=0.5\textwidth]{../figs/Ana/M2HP_SoB.png}
    %\caption{Distribution of $(M(top^{2nd})+M(W^{2nd}))/M(H)$ for data and the sum of the Monte Carlo samples. Selection criteria are applied up to Higgs mass cut. The low statistics in the multijet (QCD) MC sample is visible at this stage. The gray band represents the statistical uncertainties from the sum of the MC background and it is dominated by QCD samples~[left]. Efficiency of selection criterion $(M(top^{2nd})+M(W^{2nd}))/M(H)>x$, with $x$ the cut value, as function of cut value for $M=700$ \GeVcc~signal sample, \ttbar~and QCD\_HT-500To1000 MC samples and ratios between efficiency for signal and each background~[right].}
    %\label{fig:m2thp}
  \end{center}
\end{figure}

\vspace{-.5cm}
    \begin{block}{}
      \tiny \centering Distribution of $(M(top^{2nd})+M(W^{2nd}))/M(H)$ for data and the sum of the Monte Carlo samples. Selection criteria are applied up to Higgs mass cut. The low statistics in the multijet (QCD) MC sample is visible at this stage. The gray band represents the statistical uncertainties from the sum of the MC background and it is dominated by QCD samples~[left]. Efficiency of selection criterion $(M(top^{2nd})+M(W^{2nd}))/M(H)>x$, with $x$ the cut value, as function of cut value for $M=700$ \GeVcc~signal sample, \ttbar~and QCD\_HT-500To1000 MC samples and ratios between efficiency for signal and each background~[right].
    \end{block}

\end{frame}

\begin{frame}{}
\vspace{-.2cm}

\begin{figure}[!Hhtbp]
  \begin{center}
    \includegraphics[width=0.5\textwidth]{../figs/Ana/DRTp6JNm1.png}
    \includegraphics[width=0.5\textwidth]{../figs/Ana/DRTp6thJ_SoB.png}
    %\caption{Distributions for $\Delta R (T' j^{6})$  for data and the sum of Monte Carlo samples. All others criteria are applied up to this one. The low statistics in the multijet (QCD) MC sample is visible at this stage. The gray band represents the statistical uncertainties from the sum of the MC background and it is dominated by QCD samples~[left]. Efficiency of selection criterion $\Delta R (T' j^{6})>x$, with $x$ the cut value, as function of cut value for $M=700$ \GeVcc~signal sample, \ttbar~and QCD\_HT-500To1000 MC samples and ratios between efficiency for signal and each background~[right].}
    %\label{fig:jet6}
  \end{center}
\end{figure}

\vspace{-.5cm}
    \begin{block}{}
      \tiny \centering Distributions for $\Delta R (T' j^{6})$  for data and the sum of Monte Carlo samples. All others criteria are applied up to this one. The low statistics in the multijet (QCD) MC sample is visible at this stage. The gray band represents the statistical uncertainties from the sum of the MC background and it is dominated by QCD samples~[left]. Efficiency of selection criterion $\Delta R (T' j^{6})>x$, with $x$ the cut value, as function of cut value for $M=700$ \GeVcc~signal sample, \ttbar~and QCD\_HT-500To1000 MC samples and ratios between efficiency for signal and each background~[right].
    \end{block}

\end{frame}

\begin{frame}{}
\vspace{-.2cm}

\begin{figure}[!Hhtbp]
  \begin{center}
    \includegraphics[width=0.5\textwidth]{../figs/Ana/RelHTNm1.png}
    \includegraphics[width=0.5\textwidth]{../figs/Ana/RelHT_SoB.png}
    %\caption{Distribution of Relative $H_{T}$ for data and the sum of the Monte Carlo samples. All others criteria are applied up to this one. At this stage multijet (QCD) MC sample have very low statistics. The gray band represents the statistical uncertainties from the sum of the MC background and it is dominated by QCD samples [left]. Efficiency of selection criterion Relative $H_{T}>x$, with $x$ the cut value, as function of cut value for $M=700$ \GeVcc~signal sample, \ttbar~and QCD\_HT-500To1000 MC samples and ratios between efficiency for signal and each background [right]. }
    %\label{fig:RelHtMass}
  \end{center}
\end{figure}

\vspace{-.5cm}
    \begin{block}{}
      \tiny \centering Distribution of Relative $H_{T}$ for data and the sum of the Monte Carlo samples. All others criteria are applied up to this one. At this stage multijet (QCD) MC sample have very low statistics. The gray band represents the statistical uncertainties from the sum of the MC background and it is dominated by QCD samples [left]. Efficiency of selection criterion Relative $H_{T}>x$, with $x$ the cut value, as function of cut value for $M=700$ \GeVcc~signal sample, \ttbar~and QCD\_HT-500To1000 MC samples and ratios between efficiency for signal and each background [right].
    \end{block}

\end{frame}

\begin{frame}{}
\vspace{-.2cm}

\begin{table}[htbH]
\begin{center}
\resizebox{\textwidth}{!}{
\begin{tabular}{|c|c|c|c|c|c|}
\hline 
Cut & Signal (M=700 GeV/$c^{2}$) & Multijet & $t\bar{t}$ + single top & Diboson  & Data \\
\hline
Trigger selection & 560.31$\pm$3.13 & 74803879.03$\pm$190145.80 & 601988.30$\pm$512.93 & 11718.97$\pm$47.35 & 451250111$\pm$21242.65 \\
$p_{T}$ and $\eta$ selection on jets & 295.15$\pm$2.27 & 13863750.21$\pm$73389.97 & 219998.68$\pm$275.69 & 1891.41$\pm$19.02 & 12865712$\pm$3586.88 \\
$j_{1}>150$~GeV/c & 268.60$\pm$2.17 & 10501566.69$\pm$58350.08 & 148893.06$\pm$232.38 & 1357.09$\pm$16.11 & 9303286$\pm$3050.13 \\
$H_{T}>550$~GeV/c & 267.01$\pm$2.16 & 10123680.68$\pm$56326.87 & 145792.65$\pm$229.61 & 1307.81$\pm$15.82 & 9001871$\pm$3000.31 \\
$n_{b}^{CSVM}>=3$ & 81.65$\pm$1.19 & 48381.01$\pm$3554.13 & 11920.15$\pm$61.87 & 17.09$\pm$1.74 & 73879$\pm$271.81 \\
$\chi^{2}<8$ & 39.49$\pm$0.83 & 4284.06$\pm$947.99 & 3858.73$\pm$29.84 & 1.65$\pm$0.51 & 10581$\pm$102.86 \\
$\Delta R(bb) <1.2$ & 36.00$\pm$0.79 & 1343.66$\pm$249.14 & 1552.02$\pm$18.01 & 0.78$\pm$0.29 & 3874$\pm$62.24 \\
$105~\text{GeV}/c^{2} <M(H_{cand})<145~\text{GeV}/c^{2}$ & 32.05$\pm$0.75 & 1023.03$\pm$220.82 & 1138.84$\pm$15.37 & 0.40$\pm$0.21 & 2820$\pm$53.10 \\
$\frac{M(top^{2nd}_{cand})+M(W^{2nd}_{cand})}{M(H_{cand})}>6.8$ & 20.19$\pm$0.59 & 400.00$\pm$102.61 & 359.50$\pm$8.78 & 0.19$\pm$0.14 & 1242$\pm$35.24 \\
$ \Delta R (T' j^{6})>4.8$ & 11.27$\pm$0.44 & 17.90$\pm$7.34 & 36.00$\pm$2.16 & 0$\pm$0 & --- \\
$\frac{p_{T}(H_{cand})+p_{T}(top_{cand})}{H_{T}} > 0.67 $ & 10.01$\pm$0.42 & 3.16$\pm$3.07 & 20.95$\pm$1.28 & 0$\pm$0 & --- \\
\hline
\end{tabular}
}
%\caption{Cut flow of expected events from MC samples and observed events in data as a function of cuts applied. After $ \Delta R (T' j^{6})>4.8$ cut there are no longer events in the diboson MC samples.\label{tab:cutflow2}}
\end{center}
\end{table}

\vspace{-.2cm}
    \begin{block}{}
      \tiny \centering Cut flow of expected events from MC samples and observed events in data as a function of cuts applied. After $ \Delta R (T' j^{6})>4.8$ cut there are no longer events in the diboson MC samples.
    \end{block}

\end{frame}

\begin{frame}{}
\vspace{-.2cm}

\begin{table}[htbH]
\begin{center}
\resizebox{\textwidth}{!}{
\begin{tabular}{|c|c|c|c|c|c|c|c|c|}
\hline 
Cut & QCD\_HT-500To1000 & QCD\_HT-1000ToInf & QCD\_Pt-120to170 & QCD\_Pt-170to300 & QCD\_Pt-300to470 & QCD\_Pt-470to600 & QCD\_Pt-600to800 & QCD\_Pt-800to1000 \\
\hline
Trigger selection & $4341732\pm2083.68$ & $3626698\pm1904.39$ & $84995\pm291.54$ & $273227\pm522.71$ & $604608\pm777.57$ & $515380\pm717.90$ & $547153\pm739.70$ & $544121\pm737.65$ \\
$p_{T}$ and $\eta$ selection on jets & 1239652$\pm$1113.40 & 1590777$\pm$1261.26 & 10695$\pm$103.42 & 66092$\pm$257.08 & 214821$\pm$463.49 & 206651$\pm$454.59 & 226714$\pm$476.14 & 227544$\pm$477.02 \\
$j_{1}>150$~GeV/c & 1111168$\pm$1054.12 & 1590771$\pm$1261.26 & 5873$\pm$76.64 & 57578$\pm$239.95 & 213922$\pm$462.52 & 206606$\pm$454.54 & 226707$\pm$476.14 & 227542$\pm$477.01 \\
$H_{T}>550$~GeV/c & 1097972$\pm$1047.84 & 1590771$\pm$1261.26 & 5329$\pm$73.00 & 56575$\pm$237.85 & 213880$\pm$462.47 & 206605$\pm$454.54 & 226707$\pm$476.14 & 227542$\pm$477.01 \\
$n_{b}^{CSVM}>=3$ & 5522$\pm$74.31 & 9114$\pm$95.47 & 18$\pm$4.24 & 303$\pm$17.41 & 1325$\pm$36.40 & 1196$\pm$34.58 & 1314$\pm$36.25 & 1283$\pm$35.82 \\
$\chi^{2}<8$ & 526$\pm$22.93 & 548$\pm$23.41 & 1$\pm$1 & 32$\pm$5.66 & 102$\pm$10.10 & 70$\pm$8.37 & 57$\pm$7.55 & 40$\pm$6.32 \\
$\Delta R(bb) <1.2$ & 195$\pm$13.96 & 286$\pm$16.91 & 0$\pm$0 & 10$\pm$3.16 & 51$\pm$7.14 & 42$\pm$6.48 & 23$\pm$4.80 & 20$\pm$4.47 \\
$105~\text{GeV}/c^{2} <M(H_{cand})<145~\text{GeV}/c^{2}$ & 137$\pm$11.70 & 222$\pm$14.90 & 0$\pm$0 & 8$\pm$2.83 & 42$\pm$6.48 & 34$\pm$5.83 & 16$\pm$4 & 14$\pm$3.74 \\
$\frac{M(top^{2nd}_{cand})+M(W^{2nd}_{cand})}{M(H_{cand})}>6.8$ & 77$\pm$8.77 & 201$\pm$14.18 & 0$\pm$0 & 1$\pm$1 & 28$\pm$5.29 & 31$\pm$5.57 & 14$\pm$3.74 & 13$\pm$3.61 \\
$ \Delta R (T' j^{6})>4.8$ & 6$\pm$2.45 & 5$\pm$2.24 & 0$\pm$0 & 0$\pm$0 & 0$\pm$0 & 0$\pm$0 & 0$\pm$0 & 0$\pm$0 \\
$\frac{p_{T}(H_{cand})+p_{T}(top_{cand})}{H_{T}} > 0.67 $ & 1$\pm$1 & 2$\pm$1.41 & 0$\pm$0 & 0$\pm$0 & 0$\pm$0 & 0$\pm$0 & 0$\pm$0 & 0$\pm$0 \\
\hline
Weight & $28.59\ex{-1}$ & $14.95\ex{-2}$ & 518.02 & 58.06 & $29.24\ex{-1}$ & $28.45\ex{-2}$ & $6.76\ex{-2}$ & $0.89\ex{-2}$ \\
\hline
\end{tabular}
}
%\caption{Cut flow of unweighted events from QCD MC samples as a function of cuts applied. Errors are calculated as $\sqrt{N}$ of the central value. In the last line are presented the weights corresponding to each sample from normalization to luminosity.\label{tab:cutflowQCD}}
\end{center}
\end{table}

\vspace{-.2cm}
    \begin{block}{}
      \tiny \centering Cut flow of unweighted events from QCD MC samples as a function of cuts applied. Errors are calculated as $\sqrt{N}$ of the central value. In the last line are presented the weights corresponding to each sample from normalization to luminosity.
    \end{block}

\end{frame}

\begin{frame}{}
\vspace{-.2cm}

\begin{table}[htbH]
\begin{center}
\resizebox{\textwidth}{!}{
\begin{tabular}{|c|c|c|c|c|c|c|c|}
\hline 
Cut & TTJets & T\_tW-channel & T\_t-channel & T\_s-channel & Tbar\_tW-channel & Tbar\_t-channel & Tbar\_s-channel \\
\hline
Trigger selection & 6970016$\pm$2640.08 & 33485$\pm$182.99 & 49803$\pm$223.17 & 5304$\pm$72.83 & 33001$\pm$181.66 & 23459$\pm$153.16 & 2416$\pm$49.15 \\
$p_{T}$ $\eta$ selection on jets & 2633335$\pm$1622.76 & 9080$\pm$95.29 & 9863$\pm$99.31 & 1172$\pm$34.23 & 8827$\pm$93.95 & 4477$\pm$66.91 & 494$\pm$22.23 \\
$j_{1}>150$~GeV/c & 1772135$\pm$1331.22 & 6696$\pm$81.83 & 7384$\pm$85.93 & 954$\pm$30.89 & 6467$\pm$80.42 & 3257$\pm$57.07 & 390$\pm$19.75 \\
$H_{T}>550$~GeV/c & 1735831$\pm$1317.51 & 6549$\pm$80.93 & 7155$\pm$84.59 & 931$\pm$30.51 & 6321$\pm$79.50 & 3140$\pm$56.04 & 377$\pm$19.42 \\
$n_{b}^{CSVM}>=3$ & 143984$\pm$379.45 & 403$\pm$20.07 & 439$\pm$20.95 & 67$\pm$8.19 & 413$\pm$20.32 & 216$\pm$14.70 & 27$\pm$5.20 \\
$\chi^{2}<8$ & 47840$\pm$218.72 & 66$\pm$8.12 & 47$\pm$6.86 & 10$\pm$3.16 & 81$\pm$9 & 23$\pm$4.80 & 8$\pm$2.83 \\
$\Delta R(bb) <1.2$ & 19350$\pm$139.10 & 21$\pm$4.58 & 18$\pm$4.24 & 4$\pm$2 & 21$\pm$4.58 & 8$\pm$2.83 & 2$\pm$1.41 \\
$105~\text{GeV}/c^{2} <M(H_{cand})<145~\text{GeV}/c^{2}$ & 14201$\pm$119.17 & 14$\pm$3.74 & 14$\pm$3.74 & 3$\pm$1.73 & 16$\pm$4 & 6$\pm$2.45 & 1$\pm$1 \\
$\frac{M(top^{2nd}_{cand})+M(W^{2nd}_{cand})}{M(H_{cand})}>6.8$ & 4467$\pm$66.84 & 3$\pm$1.73 & 9$\pm$3 & 1$\pm$1 & 5$\pm$2.24 & 4$\pm$2 & 0$\pm$0 \\
$ \Delta R (T' j^{6})>4.8$ & 446$\pm$21.12 & 0$\pm$0 & 3$\pm$1.73 & 0$\pm$0 & 0$\pm$0 & 0$\pm$0 & 0$\pm$0 \\
$\frac{p_{T}(H_{cand})+p_{T}(top_{cand})}{H_{T}} > 0.67 $ & 266$\pm$16.31 & 0$\pm$0 & 0$\pm$0 & 0$\pm$0 & 0$\pm$0 & 0$\pm$0 & 0$\pm$0 \\
\hline
Weight & $78.77\ex{-3}$ & $44.01\ex{-2}$ & $28.80\ex{-2}$ & $28.76\ex{-2}$ & $44.38\ex{-2}$ & $30.30\ex{-2}$ & $24.79\ex{-2}$ \\
\hline
\end{tabular}
}
%\caption{Cut flow of unweighted events from \ttbar~and single top MC samples as a function of cuts applied. Errors are calculated as $\sqrt{N}$ of the central value. In the last line are presented the weights corresponding to each sample from normalization to luminosity.\label{tab:cutflowTop}}
\end{center}
\end{table}

\vspace{-.2cm}
    \begin{block}{}
      \tiny \centering Cut flow of unweighted events from \ttbar~and single top MC samples as a function of cuts applied. Errors are calculated as $\sqrt{N}$ of the central value. In the last line are presented the weights corresponding to each sample from normalization to luminosity.
    \end{block}

\end{frame}

\begin{frame}{}
\vspace{-.2cm}

\begin{table}[htbH]
\begin{center}
\resizebox{\textwidth}{!}{
\begin{tabular}{|c|c|c|c|c|}
\hline 
Cut & WZ & ZZ & WW & Signal (M=700 \GeVcc) \\
\hline
Trigger selection & 66144$\pm$257.18 & 65230$\pm$255.40 & 51613$\pm$227.18 & 32081$\pm$179.11 \\
$p_{T}$ and $\eta$ selection on jets & 10744$\pm$103.65 & 9934$\pm$99.67 & 8353$\pm$91.39 & 16899$\pm$130.00 \\
$j_{1}>150$~GeV/c & 7679$\pm$87.63 & 7197$\pm$84.84 & 6027$\pm$77.63 & 15379$\pm$124.01 \\
$H_{T}>550$~GeV/c & 7404$\pm$86.05 & 6965$\pm$83.46 & 5795$\pm$76.12 & 15288$\pm$123.46 \\
$n_{b}^{CSVM}>=3$ &  97$\pm$9.85 & 253$\pm$15.91 & 38$\pm$6.16 & 4675$\pm$68.37 \\
$\chi^{2}<8$ &  9$\pm$3 & 34$\pm$5.83 & 2$\pm$1.41 & 2261$\pm$47.55 \\
$\Delta R(bb) <1.2$ &  4$\pm$2 & 22$\pm$4.69 & 0$\pm$0 & 2061$\pm$45.40 \\
$105~\text{GeV}/c^{2} <M(H_{cand})<145~\text{GeV}/c^{2}$ &  2$\pm$1.41 & 12$\pm$3.46 & 0$\pm$0 & 1835$\pm$42.84 \\
$\frac{M(top^{2nd}_{cand})+M(W^{2nd}_{cand})}{M(H_{cand})}>6.8$ &  1$\pm$1 & 5$\pm$2.24 & 0$\pm$0 & 1156$\pm$34.00 \\
$ \Delta R (T' j^{6})>4.8$ &  0$\pm$0 & 0$\pm$0 & 0$\pm$0 & 645$\pm$25.40 \\
$\frac{p_{T}(H_{cand})+p_{T}(top_{cand})}{H_{T}} > 0.67 $ &  0$\pm$0 & 0$\pm$0 & 0$\pm$0 & 573$\pm$23.94 \\
\hline
Weight & $11.04\ex{-2}$ & $15.29\ex{-3}$ & $66.24\ex{-3}$ & $17.47\ex{-3}$ \\
\hline
\end{tabular}
}
%\caption{Cut flow of unweighted events from diboson and signal (M=700 \GeVcc) MC samples as a function of cuts applied. Errors are calculated as $\sqrt{N}$ of the central value. In the last line are presented the weights corresponding to each sample from normalization to luminosity.\label{tab:cutflowDibosonSignal}}
\end{center}
\end{table}

\vspace{-.2cm}
    \begin{block}{}
      \tiny \centering Cut flow of unweighted events from diboson and signal (M=700 \GeVcc) MC samples as a function of cuts applied. Errors are calculated as $\sqrt{N}$ of the central value. In the last line are presented the weights corresponding to each sample from normalization to luminosity.
    \end{block}

\end{frame}

\begin{frame}{Reconstructed \Tp~after full selection}
\vspace{-.2cm}
\begin{table}[htbH]
\begin{center}
\resizebox{\textwidth}{!}{
\begin{tabular}{|c|c|c|c|c|}
\hline 
\multicolumn{2}{|c}{Generated} & \multicolumn{3}{|c|}{Reconstructed} \\
Mass (GeV/$c^{2}$) & Width (GeV/$c^{2}$) & Mass (GeV/$c^{2}$) & Width (GeV/$c^{2}$) & $\chi^{2} /$ndf\\
\hline
600 & 0.62 &$604.60\pm14.18$ & $32.44\pm10.37$ & 4.99/4\\
650 & 0.80 &$644.56\pm12.64$ & $35.53\pm9.54$ & 8.03/4\\
700 & 1.02 &$691.79\pm13.65$ & $41.16\pm9.75$ & 10.80/7\\
750 & 1.27 &$736.26\pm15.53$ & $45.38\pm11.46$ & 10.24/7\\
800 & 1.56 &$782.77\pm18.17$ & $49.52\pm13.54$ & 24.10/7\\
850 & 1.89 &$832.86\pm18.09$ & $47.89\pm13.44$ & 16.06/7\\
900 & 2.26 &$881.53\pm19.12$ & $45.69\pm14.23$ & 11.50/7\\
950 & 2.67 &$929.02\pm24.97$ & $51.48\pm18.91$ & 14.11/7\\
1000 & 3.13 &$970.48\pm32.15$ & $53.45\pm25.13$ & 10.42/7\\
\hline
\end{tabular}
}
%\caption{Reconstructed mass and width for \Tp~candidate after full analysis selection from a gaussian fit for each signal mass generated. \label{tab:SignalWidths}}
\end{center}
\end{table}

\vspace{-.2cm}
    \begin{block}{}
      \tiny \centering Reconstructed mass and width for \Tp~candidate after full analysis selection from a gaussian fit for each signal mass generated. 
    \end{block}

\end{frame}



\begin{frame}{Selection efficiency}
\vspace{-.2cm}
\begin{table}[htbH]
\begin{center}
\resizebox{\textwidth}{!}{
\begin{tabular}{|c|c|c|c|c|c|}
\hline 
Selection & Cut & Signal (M=700 GeV/$c^{2}$) & Multijet & $t\bar{t}$ + single top & Diboson \\
\hline
\multirow{4}{*}{\rotatebox{90}{Basic}} & Trigger cut and $p_{T}$,$\eta$ selection & $52.68\pm2.11$ & $18.53\pm5\ex{-3}$ & $36.55\pm0.06$ & $16.14\pm0.34$ \\
&$j_{1}>150$~GeV/c & $47.94\pm2.11$ & $14.04\pm4\ex{-3}$ & $24.73\pm0.06$ & $11.58\pm0.30$ \\
&$H_{T}>550$~GeV/c & $47.65\pm2.11$ & $13.53\pm4\ex{-3}$ & $24.22\pm0.06$ & $11.16\pm0.29$ \\
&$n_{b}^{CSVM}>=3$ & $14.57\pm1.49$ & $0.06\pm3\ex{-4}$ & $1.98\pm0.02$ & $0.15\pm0.04$ \\
\hline
\multirow{6}{*}{\rotatebox{90}{Analysis}} & $\chi^{2}<8$ & $7.09\pm1.09$ & $5\ex{-3}\pm8\ex{-5}$ & $0.64\pm0.01$ & $0.01\pm0.01$  \\
&$\Delta R(bb) <1.2$ & $6.47\pm1.04$ & $2\ex{-3}\pm5\ex{-5}$ & $0.26\pm7\ex{-3}$ & $7\ex{-3}\pm7\ex{-3}$ \\
%&$1.6 < \Delta R (W_{cand} H_{cand}) < 4.0$ & 6.36 & $2\ex{-3}$ & 0.23 & $5\ex{-3}$ \\
&$105~\text{GeV}/c^{2} <M(H_{cand})<145~\text{GeV}/c^{2}$ & $5.76\pm0.99$ & $1\ex{-3}\pm4\ex{-5}$ & $0.19\pm6\ex{-3}$ & $3\ex{-3}\pm5\ex{-3}$  \\
&$\frac{M(top^{2nd}_{cand})+M(W^{2nd}_{cand})}{M(H_{cand})}>6.8$ & $3.63\pm0.79$ & $5\ex{-4}\pm3\ex{-5}$ & $0.06\pm3\ex{-3}$ & $2\ex{-3}\pm4\ex{-3}$ \\
&$ \Delta R (T' j^{6})>4.8$ & $2.02\pm0.60$ & $2\ex{-5}\pm6\ex{-6}$ & $6\ex{-3}\pm1\ex{-3}$ & ---  \\
&$\frac{p_{T}(H_{cand})+p_{T}(top_{cand})}{H_{T}} > 0.67 $ & $1.80\pm0.56$ & $4\ex{-6}\pm2\ex{-6}$ & $3\ex{-3}\pm8\ex{-4}$ & --- \\
\hline
\end{tabular}
}
%\caption{Cumulative efficiencies, in \%, for signal and main background as a function of cuts applied. After $ \Delta R (T' j^{6})>4.8$ cut there are no longer events in the diboson MC samples. These efficiencies have been calculated with respect to the number of events that passed the trigger selection.\label{tab:cutflow}}
\end{center}
\end{table}

\vspace{-.2cm}
    \begin{block}{}
      \tiny \centering Cumulative efficiencies, in \%, for signal and main background as a function of cuts applied. After $ \Delta R (T' j^{6})>4.8$ cut there are no longer events in the diboson MC samples. These efficiencies have been calculated with respect to the number of events that passed the trigger selection.
    \end{block}

\end{frame}



\begin{frame}{Trigger efficiency}
\vspace{-.5cm}
\begin{figure}[!Hhtbp]
  \begin{center}
    \includegraphics[width=0.45\textwidth]{../figs/Ana/Trigger_Eff_hundreds_chi2.png}
    \includegraphics[width=0.45\textwidth]{../figs/Ana/Trigger_Eff_fifties_chi2.png}
    %\caption{Efficiency in data and the MC signal samples for events passing trigger HLT\_Dijet80\_Dijet60\_Dijet20 with respect to trigger bit HLT\_HT400 after standard selection up to $\chi^{2}$ cut (included). At this early stage of the selection, discrepancies between 10\% and 6\% at higher $p_{T}$ are observed between data and signal MC samples. Differences between \ttbar~and data are maximum 7\%. This efficiency is parametrized as function of the 6$^{th}$ jet $p_{T}$. Efficiencies for signal MC samples with \Tp~masses equal to 600, 700, 800, 900 and 1000~\GeVcc~are shown with \ttbar~and data [left]. Efficiencies for signal MC samples with \Tp~masses equal to 650, 750, 850 and 950~\GeVcc~are shown with \ttbar~and data [right].}
    %\label{fig:TrigEff}
  \end{center}
\end{figure}

\vspace{-.6cm}
    \begin{block}{}
      \tiny \centering Efficiency in data and the MC signal samples for events passing trigger HLT\_Dijet80\_Dijet60\_Dijet20 with respect to trigger bit HLT\_HT400 after standard selection up to $\chi^{2}$ cut (included). At this early stage of the selection, discrepancies between 10\% and 6\% at higher $p_{T}$ are observed between data and signal MC samples. Differences between \ttbar~and data are maximum 7\%. This efficiency is parametrized as function of the 6$^{th}$ jet $p_{T}$. Efficiencies for signal MC samples with \Tp~masses equal to 600, 700, 800, 900 and 1000~\GeVcc~are shown with \ttbar~and data [left]. Efficiencies for signal MC samples with \Tp~masses equal to 650, 750, 850 and 950~\GeVcc~are shown with \ttbar~and data [right].
    \end{block}

\end{frame}

\begin{frame}{}
\vspace{-.2cm}
\begin{figure}[!Hhtbp]
  \begin{center}
    \includegraphics[width=0.45\textwidth]{../figs/Ana/Trigger_Eff_hundreds_FullSel.png}
    \includegraphics[width=0.45\textwidth]{../figs/Ana/Trigger_Eff_fifties_FullSel.png}
    %\caption{Efficiency in data and the MC signal samples for events passing trigger bit HLT\_Dijet80\_Dijet60\_Dijet20 with respect to trigger bit HLT\_HT400 after full selection. This efficiency is parametrized as function of the 6$^{th}$ jet $p_{T}$. The dispersion observed is about 10\% between data and signal MC samples, while only about 4\%  for \ttbar. Efficiencies for signal MC samples with \Tp~masses equal to 600, 700, 800, 900 and 1000~\GeVcc~are shown with \ttbar~and data [left]. Efficiencies for signal MC samples with \Tp~masses equal to 650, 750, 850 and 950~\GeVcc~are shown with \ttbar~and data [right].}
    %\label{fig:TrigEffPostMH}
  \end{center}
\end{figure}

\vspace{-.2cm}
    \begin{block}{}
      \tiny \centering Efficiency in data and the MC signal samples for events passing trigger bit HLT\_Dijet80\_Dijet60\_Dijet20 with respect to trigger bit HLT\_HT400 after full selection. This efficiency is parametrized as function of the 6$^{th}$ jet $p_{T}$. The dispersion observed is about 10\% between data and signal MC samples, while only about 4\%  for \ttbar. Efficiencies for signal MC samples with \Tp~masses equal to 600, 700, 800, 900 and 1000~\GeVcc~are shown with \ttbar~and data [left]. Efficiencies for signal MC samples with \Tp~masses equal to 650, 750, 850 and 950~\GeVcc~are shown with \ttbar~and data [right].
    \end{block}

\end{frame}


\begin{frame}{QCD~MC - QCD~MC}
\vspace{-.2cm}
\begin{figure}[!Hhtbp]
  \begin{center}
    \includegraphics[width=0.33\textwidth]{../figs/Ana/InclusiveVal_chi2_QCD.png}
    \includegraphics[width=0.33\textwidth]{../figs/Ana/InclusiveVal_DRWH_QCD.png}
    \includegraphics[width=0.33\textwidth]{../figs/Ana/InclusiveVal_M2HP_QCD.png}
    %\caption{Comparison of 5-jets invariant mass in signal sample and control sample. In the control sample, different b-tagging working points are studied. This comparison is done for QCD Monte Carlo samples within 3 stages of selection: A [top left], B [top right] and C [bottom]. The 3 working points are given in different colors. The signal sample is displayed in gray. QCD MC as all the other Monte-Carlo samples are purely used for illustration. All histograms have been normalized to unity. The errors of the QCD samples are underestimated due to their low statistics. To correctly estimate the associated errors, for each MC sample an error of 1.8 events (times the corresponding weight) should be added to each bin with zero entries.}
    %\label{fig:StageWPQCD}
  \end{center}
\end{figure}

\vspace{-.2cm}
    \begin{block}{}\tiny
      \textbf{Figure}: Comparison of 5-jets invariant mass in signal sample and control sample. In the control sample, different b-tagging working points are studied. This comparison is done for QCD Monte Carlo samples within 3 stages of selection: A [top left], B [top right] and C [bottom]. The 3 working points are given in different colors. The signal sample is displayed in gray. QCD MC as all the other Monte-Carlo samples are purely used for illustration. All histograms have been normalized to unity. The errors of the QCD samples are underestimated due to their low statistics. To correctly estimate the associated errors, for each MC sample an error of 1.8 events (times the corresponding weight) should be added to each bin with zero entries.
    \end{block}

\end{frame}

\begin{frame}{MC Sum - MC Sum}
\vspace{-.2cm}
\begin{figure}[!Hhtbp]
  \begin{center}
    \includegraphics[width=0.25\textwidth]{../figs/Ana/InclusiveVal_chi2_MCsum.png}
    \includegraphics[width=0.25\textwidth]{../figs/Ana/InclusiveVal_DRWH_MCsum.png}
    \includegraphics[width=0.25\textwidth]{../figs/Ana/InclusiveVal_M2HP_MCsum.png}
    \includegraphics[width=0.25\textwidth]{../figs/Ana/InclusiveVal_RelHT_MCsum.png}
    %\caption{Comparison of 5-jets invariant mass in signal sample and control sample. In the control sample, different b-tagging working points are studied. This comparison is done for the weighted sum of background Monte Carlo samples within 4 stages of selection: A [top left], B [top right], C [bottom left] and D [bottom right]. The 3 working points are given in different colors. Within statistical error, the 3 shapes for the control sample are in agreement at all stages with the signal sample, in gray. Monte-Carlo samples are purely used for illustration. All histograms have been normalized to unity. The errors of the sum of MC samples are underestimated due to the low statistics of the QCD samples.}
    %\label{fig:StageWPSum}
  \end{center}
\end{figure}

\vspace{-.2cm}
    \begin{block}{}\tiny
      \textbf{Figure}: Comparison of 5-jets invariant mass in signal sample and control sample. In the control sample, different b-tagging working points are studied. This comparison is done for the weighted sum of background Monte Carlo samples within 4 stages of selection: A [top left], B [top right], C [bottom left] and D [bottom right]. The 3 working points are given in different colors. Within statistical error, the 3 shapes for the control sample are in agreement at all stages with the signal sample, in gray. Monte-Carlo samples are purely used for illustration. All histograms have been normalized to unity. The errors of the sum of MC samples are underestimated due to the low statistics of the QCD samples.
    \end{block}

\end{frame}


\begin{frame}{$\chi^{2}$-test Control-Signal samples}
\vspace{-.2cm}
\begin{figure}[!Hhtbp]
  \begin{center}
    \includegraphics[width=0.45\textwidth]{../figs/Ana/Chi2_opt_at_DRbb_OneCombination_Scan2To800Step10_LOGSpacing_Revision.png} %Chi2_opt_at_DRbb_OneCombination_Scan2To800Step10.png}
    \includegraphics[width=0.45\textwidth]{../figs/Ana/Chi2_opt_at_M2HP_OneCombination_Scan2To800Step10_LOGSpacing_Revision.png}
    %\caption{Distribution of the agreement between control sample and signal sample in data at early stage of the selection as a function of the $\chi^2$ cut value. The y-axis represents the $\chi^{2}$/ndf from a shape comparison made in data between control sample and analysis sample. The study is performed after requiring $\Delta R(bb) <1.2$ [left] and up to $\frac{M(top^{2nd}_{cand})+M(W^{2nd}_{cand})}{M(H_{cand})}>6.8$ criterion [right] on top of basic selection and reconstruction. The control sample $M(5j)$ shape tend to agree with signal sample for lower values of $\chi^2$ cut.}
    %\label{fig:optchi2}
  \end{center}
\end{figure}

\vspace{-.2cm}
    \begin{block}{}
      \tiny \centering Distribution of the agreement between control sample and signal sample in data at early stage of the selection as a function of the $\chi^2$ cut value. The y-axis represents the $\chi^{2}$/ndf from a shape comparison made in data between control sample and analysis sample. The study is performed after requiring $\Delta R(bb) <1.2$ [left] and up to $\frac{M(top^{2nd}_{cand})+M(W^{2nd}_{cand})}{M(H_{cand})}>6.8$ criterion [right] on top of basic selection and reconstruction. The control sample $M(5j)$ shape tend to agree with signal sample for lower values of $\chi^2$ cut.
    \end{block}

\end{frame}


\begin{frame}{Exclusive Data - Data}
\vspace{-.2cm}
\begin{figure}[!Hhtbp]
  \begin{center}
    \includegraphics[width=0.33\textwidth]{../figs/Ana/ExclusiveVal_chi2_data.png}
    \includegraphics[width=0.33\textwidth]{../figs/Ana/ExclusiveVal_DRWH_data.png}
    \includegraphics[width=0.33\textwidth]{../figs/Ana/ExclusiveVal_M2HP_data.png}
    %\caption{Comparison of 5-jets invariant mass in signal sample and control sample. In the control sample, different b-tagging working points are studied. This comparison is done, in exclusive regions, for data within 3 stages of selection: A [top left], B [top right] and C [bottom]. The 3 working points are given in different colors. Within statistical error, the 3 shapes for the control sample are in agreement at all stages with the signal sample, in gray. All histograms have been normalized to unity.}
    %\label{fig:StageExWPData}
  \end{center}
\end{figure}

\vspace{-.2cm}
    \begin{block}{}\tiny
      Independence of WP for Control Sample $\to$ exclusive WP: [0.244,0.389), [0.389,0.534), [0.534,0.679)\\
      \textbf{Figure}: Comparison of 5-jets invariant mass in signal sample and control sample. In the control sample, different b-tagging working points are studied. This comparison is done, in exclusive regions, for data within 3 stages of selection: A, B and C. The 3 working points are given in different colors. Within statistical error, the 3 shapes for the control sample are in agreement at all stages with the signal sample, in gray. All histograms have been normalized to unity.
    \end{block}

\end{frame}

\begin{frame}{Exclusive \ttbar~MC - \ttbar~MC}
\vspace{-.2cm}
\begin{figure}[!Hhtbp]
  \begin{center}
    \includegraphics[width=0.25\textwidth]{../figs/Ana/ExclusiveVal_chi2_ttbar.png}
    \includegraphics[width=0.25\textwidth]{../figs/Ana/ExclusiveVal_DRWH_ttbar.png}
    \includegraphics[width=0.25\textwidth]{../figs/Ana/ExclusiveVal_M2HP_ttbar.png}
    \includegraphics[width=0.25\textwidth]{../figs/Ana/ExclusiveVal_RelHT_ttbar.png}
    %\caption{Comparison of 5-jets invariant mass in signal sample and control sample. In the control sample, different b-tagging working points are studied. This comparison is done, in exclusive regions, for $t\bar{t}$ Monte Carlo samples within 4 stages of selection: A [top left], B [top right], C [bottom left] and D [bottom right]. The 3 working points are given in different colors. Within statistical error, the 3 shapes for the control sample are in agreement at all stages with the signal sample, in gray. \ttbar~MC as all the other Monte-Carlo samples are purely used for illustration. All histograms have been normalized to unity.}
    %\label{fig:StageExWPttbar}
  \end{center}
\end{figure}

\vspace{-.2cm}
    \begin{block}{}\tiny
      Independence of WP for Control Sample $\to$ exclusive WP: [0.244,0.389), [0.389,0.534), [0.534,0.679)\\
      \textbf{Figure}: Comparison of 5-jets invariant mass in signal sample and control sample. In the control sample, different b-tagging working points are studied. This comparison is done, in exclusive regions, for $t\bar{t}$ Monte Carlo samples within 4 stages of selection: A, B, C and D. The 3 working points are given in different colors. Within statistical error, the 3 shapes for the control sample are in agreement at all stages with the signal sample, in gray. \ttbar~MC as all the other Monte-Carlo samples are purely used for illustration. All histograms have been normalized to unity.
    \end{block}

\end{frame}

\begin{frame}{Exclusive QCD~MC - QCD~MC}
\vspace{-.2cm}
\begin{figure}[!Hhtbp]
  \begin{center}
    \includegraphics[width=0.33\textwidth]{../figs/Ana/ExclusiveVal_chi2_QCD.png}
    \includegraphics[width=0.33\textwidth]{../figs/Ana/ExclusiveVal_DRWH_QCD.png}
    \includegraphics[width=0.33\textwidth]{../figs/Ana/ExclusiveVal_M2HP_QCD.png}
    %\caption{Comparison of 5-jets invariant mass in signal sample and control sample. In the control sample, different b-tagging working points are studied. This comparison is done, in exclusive regions, for QCD Monte Carlo samples within 3 stages of selection: A [top left], B [top right] and C [bottom]. The 3 working points are given in different colors. The signal sample is displayed in gray. A lack of statistics is visible. All histograms have been normalized to unity. The errors of the QCD samples are underestimated due to their low statistics. }
    %\label{fig:StageExWPQCD}
  \end{center}
\end{figure}

\vspace{-.2cm}
    \begin{block}{}\tiny
      Independence of WP for Control Sample $\to$ exclusive WP: [0.244,0.389), [0.389,0.534), [0.534,0.679)\\
      \textbf{Figure}: Comparison of 5-jets invariant mass in signal sample and control sample. In the control sample, different b-tagging working points are studied. This comparison is done, in exclusive regions, for QCD Monte Carlo samples within 3 stages of selection: A, B and C. The 3 working points are given in different colors. The signal sample is displayed in gray. A lack of statistics is visible. All histograms have been normalized to unity. The errors of the QCD samples are underestimated due to their low statistics.
    \end{block}

\end{frame}

\begin{frame}{Exclusive MC Sum - MC Sum}
\vspace{-.2cm}
\begin{figure}[!Hhtbp]
  \begin{center}
    \includegraphics[width=0.25\textwidth]{../figs/Ana/ExclusiveVal_chi2_MCsum.png}
    \includegraphics[width=0.25\textwidth]{../figs/Ana/ExclusiveVal_DRWH_MCsum.png}
    \includegraphics[width=0.25\textwidth]{../figs/Ana/ExclusiveVal_M2HP_MCsum.png}
    \includegraphics[width=0.25\textwidth]{../figs/Ana/ExclusiveVal_RelHT_MCsum.png}
    %\caption{Comparison of 5-jets invariant mass in signal sample and control sample. In the control sample, different b-tagging working points are studied. This comparison is done, in exclusive regions, for the weighted sum of background Monte Carlo samples within 4 stages of selection: A [top left], B [top right], C [bottom left] and D [bottom right]. The 3 working points are given in different colors. The signal sample is displayed in gray. A lack of statistics is visible. All histograms have been normalized to unity. The errors of the sum of MC samples are underestimated due to the low statistics of the QCD samples.}
    %\label{fig:StageExWPSum}
  \end{center}
\end{figure}

\vspace{-.2cm}
    \begin{block}{}\tiny
      Independence of WP for Control Sample $\to$ exclusive WP: [0.244,0.389), [0.389,0.534), [0.534,0.679)\\
      \textbf{Figure}: Comparison of 5-jets invariant mass in signal sample and control sample. In the control sample, different b-tagging working points are studied. This comparison is done, in exclusive regions, for the weighted sum of background Monte Carlo samples within 4 stages of selection: A, B, C and D. The 3 working points are given in different colors. The signal sample is displayed in gray. A lack of statistics is visible. All histograms have been normalized to unity. The errors of the sum of MC samples are underestimated due to the low statistics of the QCD samples.
    \end{block}

\end{frame}


\begin{frame}{Background normalization estimation from data}
\vspace{-.2cm}
\begin{figure}[!Hhtbp]
  \begin{center}
    \includegraphics[width=0.49\textwidth]{../figs/Ana/HM_ControlSample_SignalSample_chi2Cut.png}
    \includegraphics[width=0.49\textwidth]{../figs/Ana/HM_ControlSample_SignalSample_M2HPCut.png}
    %\caption{Higgs candidate mass distribution for the data in the signal and control sample after $\chi^{2}<8$ cut [left] and after $\frac{M(top^{2nd}_{cand})+M(W^{2nd}_{cand})}{M(H_{cand})}>6.8$ criterion [right].}
    %\label{fig:HiggsMassCSSS}
  \end{center}
\end{figure}

\vspace{-.2cm}
    \begin{block}{}
      \tiny \centering Higgs candidate mass distribution for the data in the signal and control sample after $\chi^{2}<8$ cut [left] and after $\frac{M(top^{2nd}_{cand})+M(W^{2nd}_{cand})}{M(H_{cand})}>6.8$ criterion [right].
    \end{block}

\end{frame}

\begin{frame}{Signal contamination in method for background normalization estimation from data}
\vspace{-.2cm}
\begin{center}
\resizebox{\textwidth}{!}{
\begin{tabular}{|c|c|c|c|c|}
\hline
 Cut & Signal (M=700\GeVcc) $N^{CS}_{in}$ & Signal (M=700\GeVcc) $N^{CS}_{out}$ & Signal (M=700\GeVcc) $N^{SS}_{in}$ & Signal (M=700\GeVcc) $N^{SS}_{out}$ \\
\hline
$\chi^{2}<8$ & 34.91$\pm$0.78 & 5.83$\pm$0.32 & 34.90$\pm$0.78 & 4.59$\pm$0.28 \\
$\Delta R(bb) <1.2$ & 28.35$\pm$0.70 & 4.12$\pm$0.27 & 32.05$\pm$0.75 & 3.95$\pm$0.26 \\
$\frac{M(top^{2nd}_{cand})+M(W^{2nd}_{cand})}{M(H_{cand})}>6.8$ & 17.85$\pm$0.56 & 2.79$\pm$0.22 & 20.19$\pm$0.59 & 2.53$\pm$0.21 \\
$ \Delta R (T' j^{6})>4.8$ & 9.27$\pm$0.40 & 1.15$\pm$0.14 & 11.27$\pm$0.44 & 1.36$\pm$0.15\\
$\frac{p_{T}(H_{cand})+p_{T}(top_{cand})}{H_{T}} > 0.67 $ & 7.77$\pm$0.37 & 1.01$\pm$0.13 & 10.01$\pm$0.42 & 1.24$\pm$0.15 \\
\hline
\end{tabular}
}
%\caption{Signal MC events for 700 \GeVcc~mass point in the control and signal sample at each stage of the selection used for the estimation of background normalization method. The maximal contamination in the control sample, after $\frac{p_{T}(H_{cand})+p_{T}(top_{cand})}{H_{T}} > 0.67 $ criterion, is around 1\%. The maximal contamination in the signal sample, after $\frac{M(top^{2nd}_{cand})+M(W^{2nd}_{cand})}{M(H_{cand})}>6.8$ cut, is around 1\%. The signal contamination in the control sample has been evaluated as $N^{CS}_{in}(Signal)/N^{CS}_{in}(Data)$, and in the signal sample as $N^{SS}_{in}(Signal)/N^{SS}_{in}(Data)$.\label{tab:NormValSignal}}
\end{center}

\vspace{-.2cm}
    \begin{block}{}
      \tiny \centering Signal MC events for 700 \GeVcc~mass point in the control and signal sample at each stage of the selection used for the estimation of background normalization method. The maximal contamination in the control sample, after $\frac{p_{T}(H_{cand})+p_{T}(top_{cand})}{H_{T}} > 0.67 $ criterion, is around 1\%. The maximal contamination in the signal sample, after $\frac{M(top^{2nd}_{cand})+M(W^{2nd}_{cand})}{M(H_{cand})}>6.8$ cut, is around 1\%. The signal contamination in the control sample has been evaluated as $N^{CS}_{in}(Signal)/N^{CS}_{in}(Data)$, and in the signal sample as $N^{SS}_{in}(Signal)/N^{SS}_{in}(Data)$.
    \end{block}

\end{frame}


\begin{frame}{Signal contamination}
\vspace{-.2cm}
\begin{figure}[!Hhtbp]
  \begin{center}
    \includegraphics[width=0.5\textwidth]{../figs/Ana/SignalContam_700_ConSample.png}
    %\caption{Signal contamination in the control sample comparing 5-jets invariant mass between data and signal (M=700 \GeVcc).}
    %\label{fig:SigContamination}
  \end{center}
\end{figure}

\vspace{-.2cm}
    \begin{block}{}
      \tiny \centering Signal contamination in the control sample comparing 5-jets invariant mass between data and signal (M=700 \GeVcc).
    \end{block}

\end{frame}

\begin{frame}{Normalization method}
\vspace{-.2cm}
%\begin{table*}[htbH]
\begin{center}
\resizebox{\textwidth}{!}{
\begin{tabular}{|c|c|c|c|c|c|}
\hline 
 Cut & $N^{CS}_{in}$ & $N^{CS}_{out}$ & $N^{SS}_{in}$ & $N^{SS}_{out}$ & $\chi^{2}$/ndf \\ 
\hline
$\chi^{2}<8$ & $163589\pm404.46$ & $49112\pm221.61$ & $8071\pm89.84$ & $2510\pm50.10$ &  0.99 \\
$\Delta R(bb) <1.2$ & $35266\pm187.79$ & $13247\pm115.10$ & $2820\pm53.10$ & $1054\pm32.47$ & 2.20 \\
%$1.6 < \Delta R (W_{cand} H_{cand}) < 4.0$ & $28166\pm167.83$ & $10563\pm102.78$ & $2456\pm49.56$ & $931\pm30.51$ & $2.67\pm0.04$ & $2.64\pm0.14$ &  $2482.49\pm120.31$\\
$\frac{M(top^{2nd}_{cand})+M(W^{2nd}_{cand})}{M(H_{cand})}>6.8$ & $19269\pm138.81$ & $8001\pm89.45$ & $1242\pm35.24$ & $528\pm22.98$ & 1.36 \\
$ \Delta R (T' j^{6})>4.8$ & $1566\pm39.75$ & $636\pm25.22$ & --- & $40\pm6.32$ & --- \\
$\frac{p_{T}(H_{cand})+p_{T}(top_{cand})}{H_{T}} > 0.67 $ & $519\pm22.78$ & $196\pm14.00$ & --- & $20\pm4.47$ & --- \\
\hline
 Cut & \multicolumn{2}{c|}{$R^{CS}$} & \multicolumn{2}{c|}{$R^{SS}$} & $N^{SS_{BKG}}_{in}$ \\
\hline
$\chi^{2}<8$ & \multicolumn{2}{c|}{$3.33\pm0.02$} & \multicolumn{2}{c|}{$3.22\pm0.10$} & $8360.65\pm225.28$ \\
$\Delta R(bb) <1.2$ & \multicolumn{2}{c|}{$2.66\pm0.04$} & \multicolumn{2}{c|}{$2.68\pm0.13$} & $2805.95\pm125.75$ \\
$\frac{M(top^{2nd}_{cand})+M(W^{2nd}_{cand})}{M(H_{cand})}>6.8$ & \multicolumn{2}{c|}{$2.41\pm0.04$} & \multicolumn{2}{c|}{$2.35\pm0.17$} & $1271.60\pm78.72$ \\
$ \Delta R (T' j^{6})>4.8$ & \multicolumn{2}{c|}{$2.46\pm0.16$} & \multicolumn{2}{c|}{---} & $98.49\pm21.97$ \\
$\frac{p_{T}(H_{cand})+p_{T}(top_{cand})}{H_{T}} > 0.67 $ & \multicolumn{2}{c|}{$2.65\pm0.31$} & \multicolumn{2}{c|}{---} & $52.96\pm17.95$ \\
\hline
\end{tabular}
}
%\caption{Results of validation procedure of estimation of normalization. The validation is started after $\chi^{2}$ cut. All other cuts are applied progressively. The $R^{SS}$ and $N^{SS}_{in}$ have been blinded in the last two lines for the validation procedure, because the amount of signal events in the signal sample is not negligible with respect to background events. In addition, the Higgs candidate mass distribution in the control sample is compared to its distribution in the signal sample via a $\chi^{2}$ test, the results of this test are shown in the table, showing the shape agreement between both samples.\label{tab:NormVal}}
\end{center}
%\end{table*}

\vspace{-.2cm}
    \begin{block}{}
      \tiny \centering Results of validation procedure of estimation of normalization. The validation is started after $\chi^{2}$ cut. All other cuts are applied progressively. The $R^{SS}$ and $N^{SS}_{in}$ have been blinded in the last two lines for the validation procedure, because the amount of signal events in the signal sample is not negligible with respect to background events. In addition, the Higgs candidate mass distribution in the control sample is compared to its distribution in the signal sample via a $\chi^{2}$ test, the results of this test are shown in the table, showing the shape agreement between both samples.
    \end{block}

\end{frame}

\begin{frame}{}
\vspace{-.2cm}
%\begin{table*}[htbH]
\begin{center}
\resizebox{\textwidth}{!}{
\begin{tabular}{|c|c|c|c|c|}
\hline 
 Cut & Signal (M=700\GeVcc) $N^{CS}_{in}$ & Signal (M=700\GeVcc) $N^{CS}_{out}$ & Signal (M=700\GeVcc) $N^{SS}_{in}$ & Signal (M=700\GeVcc) $N^{SS}_{out}$ \\ 
\hline
$\chi^{2}<8$ & 34.91$\pm$0.78 & 5.83$\pm$0.32 & 34.90$\pm$0.78 & 4.59$\pm$0.28 \\
$\Delta R(bb) <1.2$ & 28.35$\pm$0.70 & 4.12$\pm$0.27 & 32.05$\pm$0.75 & 3.95$\pm$0.26 \\
$\frac{M(top^{2nd}_{cand})+M(W^{2nd}_{cand})}{M(H_{cand})}>6.8$ & 17.85$\pm$0.56 & 2.79$\pm$0.22 & 20.19$\pm$0.59 & 2.53$\pm$0.21 \\
$ \Delta R (T' j^{6})>4.8$ & 9.27$\pm$0.40 & 1.15$\pm$0.14 & 11.27$\pm$0.44 & 1.36$\pm$0.15\\
$\frac{p_{T}(H_{cand})+p_{T}(top_{cand})}{H_{T}} > 0.67 $ & 7.77$\pm$0.37 & 1.01$\pm$0.13 & 10.01$\pm$0.42 & 1.24$\pm$0.15 \\
\hline
\end{tabular}
}
%\caption{Signal MC events for 700 \GeVcc~mass point in the control and signal sample at each stage of the selection used for the estimation of background normalization method. The maximal contamination in the control sample, after $\frac{p_{T}(H_{cand})+p_{T}(top_{cand})}{H_{T}} > 0.67 $ criterion, is around 1\%. The maximal contamination in the signal sample, after $\frac{M(top^{2nd}_{cand})+M(W^{2nd}_{cand})}{M(H_{cand})}>6.8$ cut, is around 1\%. The signal contamination in the control sample has been evaluated as $N^{CS}_{in}(Signal)/N^{CS}_{in}(Data)$, and in the signal sample as $N^{SS}_{in}(Signal)/N^{SS}_{in}(Data)$.\label{tab:NormValSignal}}
\end{center}
%\end{table*}

\vspace{-.2cm}
    \begin{block}{}
      \tiny \centering Signal MC events for 700 \GeVcc~mass point in the control and signal sample at each stage of the selection used for the estimation of background normalization method. The maximal contamination in the control sample, after $\frac{p_{T}(H_{cand})+p_{T}(top_{cand})}{H_{T}} > 0.67 $ criterion, is around 1\%. The maximal contamination in the signal sample, after $\frac{M(top^{2nd}_{cand})+M(W^{2nd}_{cand})}{M(H_{cand})}>6.8$ cut, is around 1\%. The signal contamination in the control sample has been evaluated as $N^{CS}_{in}(Signal)/N^{CS}_{in}(Data)$, and in the signal sample as $N^{SS}_{in}(Signal)/N^{SS}_{in}(Data)$
    \end{block}

\end{frame}

\begin{frame}{}
\vspace{-.2cm}
\begin{figure}[!Hhtbp]
  \begin{center}
    \includegraphics[width=0.49\textwidth]{../figs/Ana/HM_ControlSample_SignalSample_chi2Cut.png}
    \includegraphics[width=0.49\textwidth]{../figs/Ana/HM_ControlSample_SignalSample_M2HPCut.png}
    %\caption{Higgs candidate mass distribution for the data in the signal and control sample after $\chi^{2}<8$ cut [left] and after $\frac{M(top^{2nd}_{cand})+M(W^{2nd}_{cand})}{M(H_{cand})}>6.8$ criterion [right].}
    %\label{fig:HiggsMassCSSS}
  \end{center}
\end{figure}

\vspace{-.2cm}
    \begin{block}{}
      \tiny \centering Higgs candidate mass distribution for the data in the signal and control sample after $\chi^{2}<8$ cut [left] and after $\frac{M(top^{2nd}_{cand})+M(W^{2nd}_{cand})}{M(H_{cand})}>6.8$ criterion [right].
    \end{block}

\end{frame}


\begin{frame}{B-tagging systematic}
\vspace{-.2cm}
\begin{center}
\resizebox{0.5\textwidth}{!}{
\begin{tabular}{|c|c|c|c|c|}
\hline 
Sample Name & \multicolumn{2}{c|}{b or c quark} & \multicolumn{2}{c|}{Light flavours} \\
\hline
 & up [\%] & down [\%] & up [\%] & down [\%] \\
\hline
$Tj\rightarrow tHj$ 600 GeV/$c^{2}$ & 6.05 & 5.82 & 0.89 & 0.89 \\
$Tj\rightarrow tHj$ 650 GeV/$c^{2}$ & 6.08 & 5.85 & 0.51 & 0.51 \\
$Tj\rightarrow tHj$ 700 GeV/$c^{2}$ & 6.08 & 5.86 & 0.48 & 0.49 \\
$Tj\rightarrow tHj$ 750 GeV/$c^{2}$ & 6.18 & 5.95 & 0.66 & 0.66 \\
$Tj\rightarrow tHj$ 800 GeV/$c^{2}$ & 6.23 & 5.99 & 0.35 & 0.38 \\
$Tj\rightarrow tHj$ 850 GeV/$c^{2}$ & 6.57 & 6.30 & 0.88 & 0.88 \\
$Tj\rightarrow tHj$ 900 GeV/$c^{2}$ & 6.63 & 6.36 & 0.52 & 0.51 \\
$Tj\rightarrow tHj$ 950 GeV/$c^{2}$ & 6.87 & 6.58 & 0.88 & 0.89 \\
$Tj\rightarrow tHj$ 1000 GeV/$c^{2}$ & 6.80 & 6.51 & 0.55 & 0.54 \\
\hline
\end{tabular}
}
%\caption{B-tagging uncertainties for signal yields.\label{tab:SFSys}}
\end{center}
\begin{figure}[!Hhtbp]
  \begin{center}
    \includegraphics[width=0.45\textwidth]{../figs/Ana/SFSys_V8_1sigma_S.png}
    %\caption{Total b-tagging uncertainties for signal yields.}
    %\label{fig:TotalSFSys}
  \end{center}
\end{figure}


\vspace{-.2cm}
    \begin{block}{}
      \tiny \centering B-tagging uncertainties for signal yields.
    \end{block}

\end{frame}


\begin{frame}{JEC uncertainties}
\vspace{-.2cm}
\begin{center}
\resizebox{0.5\textwidth}{!}{
\begin{tabular}{|c|c|c|c|c|}
\hline 
Sample Name & \multicolumn{2}{c|}{JER} & \multicolumn{2}{c|}{JES} \\
\hline
 & up [\%] & down [\%] & up [\%] & down [\%] \\
\hline
$Tj\rightarrow tHj$ 600 GeV/$c^{2}$ & 2.87 & 1.92 & 7.02 & 5.90 \\
$Tj\rightarrow tHj$ 650 GeV/$c^{2}$ & 0.43 & 1.29 & 4.66 & 14.09 \\
$Tj\rightarrow tHj$ 700 GeV/$c^{2}$ & 2.61 & 1.26 & 1.05 & 5.98 \\
$Tj\rightarrow tHj$ 750 GeV/$c^{2}$ & 1.94 & 1.58 & 1.67 & 5.12 \\
$Tj\rightarrow tHj$ 800 GeV/$c^{2}$ & 1.15 & 1.73 & 0.94 & 7.18 \\
$Tj\rightarrow tHj$ 850 GeV/$c^{2}$ & 0.40 & 2.27 & 4.06 & 0.74 \\
$Tj\rightarrow tHj$ 900 GeV/$c^{2}$ & 1.53 & 1.72 & 3.33 & 2.89 \\
$Tj\rightarrow tHj$ 950 GeV/$c^{2}$ & 0.74 & 0.32 & 0.77 & 5.69 \\
$Tj\rightarrow tHj$ 1000 GeV/$c^{2}$ & 2.31 & 1.99 & 0.67 & 6.04 \\
\hline
\end{tabular}
}
%\caption{JEC uncertainties for signal yields.\label{tab:JECSys}}
\end{center}
\begin{figure}[!Hhtbp]
  \begin{center}
    \includegraphics[width=0.45\textwidth]{../figs/Ana/JESSys_V8_1sigma_S.png}
    %\caption{Total JEC uncertainties for signal yields.}
    %\label{fig:TotalJECSys}
  \end{center}
\end{figure}

\vspace{-.2cm}
    \begin{block}{}
      \tiny \centering JEC uncertainties for signal yields.
    \end{block}

\end{frame}

\begin{frame}{PDF uncertainties}
\vspace{-.2cm}
\begin{center}
\resizebox{0.5\textwidth}{!}{
\begin{tabular}{|c|c|c|c|c|c|}
\hline 
Sample Name & \multicolumn{2}{c|}{CTEQ6.6} & \multicolumn{2}{c|}{MSTW2008} & NNPDF2.0\\
\hline
 & up [\%] & down [\%] & up [\%] & down [\%] & up,down [\%] \\
\hline
$Tj\rightarrow tHj$ 600 GeV/$c^{2}$ & 2.54 & 1.76 & 3.09 & 2.15 & 2.37 \\
$Tj\rightarrow tHj$ 650 GeV/$c^{2}$ & 2.51 & 1.75 & 3.23 & 2.21 & 2.25 \\
$Tj\rightarrow tHj$ 700 GeV/$c^{2}$ & 2.69 & 1.84 & 3.25 & 2.23 & 2.82 \\
$Tj\rightarrow tHj$ 750 GeV/$c^{2}$ & 2.77 & 1.88 & 3.10 & 2.15 & 2.77 \\
$Tj\rightarrow tHj$ 800 GeV/$c^{2}$ & 2.83 & 1.97 & 3.16 & 2.20 & 2.58 \\
$Tj\rightarrow tHj$ 850 GeV/$c^{2}$ & 2.91 & 1.94 & 3.16 & 2.19 & 2.64 \\
$Tj\rightarrow tHj$ 900 GeV/$c^{2}$ & 3.02 & 2.00 & 3.27 & 2.24 & 2.25 \\
$Tj\rightarrow tHj$ 950 GeV/$c^{2}$ & 3.38 & 2.18 & 3.38 & 2.38 & 2.74 \\
$Tj\rightarrow tHj$ 1000 GeV/$c^{2}$ & 3.19 & 2.08 & 3.30 & 2.29 & 3.07 \\
\hline
\end{tabular}
}
%\caption{PDF+$\alpha_{s}$ uncertainties for signal yields.\label{tab:PDFsys}}
\end{center}
\begin{figure}[!Hhtbp]
  \begin{center}
    \includegraphics[width=0.45\textwidth]{../figs/Ana/PDFSys_V8_1sigma_S.png}
    %\caption{Overall PDF+$\alpha_{s}$ uncertainties for signal yields.}
    %\label{fig:TotalPDFSys}
  \end{center}
\end{figure}

\vspace{-.2cm}
    \begin{block}{}
      \tiny \centering PDF+$\alpha_{s}$ uncertainties for signal yields.
    \end{block}

\end{frame}

\begin{frame}{Pileup uncertainties}
\vspace{-.2cm}
%\begin{table*}[htbH]
\begin{center}
\resizebox{0.5\textwidth}{!}{
\begin{tabular}{|c|c|c|}
\hline 
Sample Name & \multicolumn{2}{c|}{Pileup} \\
\hline
 & up [\%] & down [\%] \\
\hline
$Tj\rightarrow tHj$ 600 GeV/$c^{2}$ & 1.36 & 1.01 \\
$Tj\rightarrow tHj$ 650 GeV/$c^{2}$ & 1.54 & 1.78 \\
$Tj\rightarrow tHj$ 700 GeV/$c^{2}$ & 3.05 & 3.37 \\
$Tj\rightarrow tHj$ 750 GeV/$c^{2}$ & 2.67 & 2.76 \\
$Tj\rightarrow tHj$ 800 GeV/$c^{2}$ & 1.18 & 1.26 \\
$Tj\rightarrow tHj$ 850 GeV/$c^{2}$ & 2.25 & 2.18 \\
$Tj\rightarrow tHj$ 900 GeV/$c^{2}$ & 1.60 & 1.66 \\
$Tj\rightarrow tHj$ 950 GeV/$c^{2}$ & 2.23 & 2.33 \\
$Tj\rightarrow tHj$ 1000 GeV/$c^{2}$ & 3.08 & 3.76 \\
\hline
\end{tabular}
%\caption{Pileup uncertainties for signal yields.\label{tab:PUsys}}
}
\end{center}
%\end{table*}
\begin{figure}[!Hhtbp]
  \begin{center}
    \includegraphics[width=0.45\textwidth]{../figs/Ana/PUSys_V8_1sigma_S.png}
    %\caption{Total pileup uncertainties for signal yields.}
    %\label{fig:TotalPUSys}
  \end{center}
\end{figure}

\vspace{-.2cm}
    \begin{block}{}
      \tiny \centering Pileup uncertainties for signal yields.
    \end{block}

\end{frame}

\begin{frame}{Trigger uncertainties}
\vspace{-.2cm}
\begin{center}
\begin{tabular}{|c|c|}
\hline 
Sample Name & Trigger \\
\hline
 & up/down [\%] \\
\hline
$Tj\rightarrow tHj$ 600 GeV/$c^{2}$ & 2.51 \\
$Tj\rightarrow tHj$ 650 GeV/$c^{2}$ & 3.31 \\
$Tj\rightarrow tHj$ 700 GeV/$c^{2}$ & 4.02 \\
$Tj\rightarrow tHj$ 750 GeV/$c^{2}$ & 3.72 \\
$Tj\rightarrow tHj$ 800 GeV/$c^{2}$ & 3.79 \\
$Tj\rightarrow tHj$ 850 GeV/$c^{2}$ & 3.46 \\
$Tj\rightarrow tHj$ 900 GeV/$c^{2}$ & 3.13 \\
$Tj\rightarrow tHj$ 950 GeV/$c^{2}$ & 3.40 \\
$Tj\rightarrow tHj$ 1000 GeV/$c^{2}$ & 3.60 \\
\hline
\end{tabular}
%\caption{Trigger uncertainties for signal yields.\label{tab:Trisys}}
\end{center}

\vspace{-.2cm}
    \begin{block}{}
      \tiny \centering Trigger uncertainties for signal yields.
    \end{block}

\end{frame}

\begin{frame}{Uncertainties summary}
\vspace{-.2cm}
%\begin{table*}[htbH]
\begin{center}
\resizebox{\textwidth}{!}{
\begin{tabular}{|c|c|c|c|c|c|c|c|}
\hline 
Sample Name & \multicolumn{2}{c|}{b-tagging} & \multicolumn{2}{c|}{JEC} & PDF+$\alpha_{S}$ & Pileup & Trigger \\
\hline
 & up [\%] & down [\%] & up [\%] & down [\%] & up/down [\%] & up/down [\%] & up/down [\%] \\
\hline
$Tj\rightarrow tHj$ 600 GeV/$c^{2}$ & 7.89 & 7.89 & 7.89 & 7.89 & 7.89 & 7.89 & 7.89 \\
$Tj\rightarrow tHj$ 650 GeV/$c^{2}$ & 6.16 & 5.93 & 5.72 & 14.17 & 5.72 & 5.72 & 5.72 \\
$Tj\rightarrow tHj$ 700 GeV/$c^{2}$ & 6.16 & 5.94 & 4.72 & 6.18 & 4.72 & 4.72 & 4.72 \\
$Tj\rightarrow tHj$ 750 GeV/$c^{2}$ & 6.26 & 6.03 & 4.46 & 5.35 & 4.46 & 4.46 & 4.46 \\
$Tj\rightarrow tHj$ 800 GeV/$c^{2}$ & 6.31 & 6.08 & 4.56 & 7.34 & 4.56 & 4.56 & 4.56 \\
$Tj\rightarrow tHj$ 850 GeV/$c^{2}$ & 6.64 & 6.38 & 4.35 & 4.26 & 4.26 & 4.26 & 4.26 \\
$Tj\rightarrow tHj$ 900 GeV/$c^{2}$ & 6.71 & 6.44 & 4.30 & 4.30 & 4.30 & 4.30 & 4.30 \\
$Tj\rightarrow tHj$ 950 GeV/$c^{2}$ & 6.95 & 6.65 & 4.56 & 5.90 & 4.56 & 4.56 & 4.56 \\
$Tj\rightarrow tHj$ 1000 GeV/$c^{2}$ & 6.87 & 6.59 & 5.37 & 6.24 & 5.37 & 5.37 & 5.37 \\
\hline
\end{tabular}
}
\end{center}
%\caption{Summary of uncertainties for signal yields. In addition, all mass points have 2.6\% uncertainty from luminosity measurement.\label{tab:sys}}
%\end{table*}

\vspace{-.2cm}
    \begin{block}{}
      \tiny \centering Summary of uncertainties for signal yields. In addition, all mass points have 2.6\% uncertainty from luminosity measurement.
    \end{block}

\end{frame}

\begin{frame}{Background shape uncertainty}
\vspace{-.2cm}
\begin{figure}[!Hhtbp]
  \begin{center}
    \includegraphics[width=0.9\textwidth]{../figs/Ana/BKG_Estim_Shape_Sys.png}
    %\caption{Dispersion of ratio between control sample and signal sample for shape data-driven estimation of background shape. A dispersion of 20\% is observed after $\Delta R (bb)$ cut. This figure corresponds to the projection on the y-axis of the ratio between signal and control samples shown in top right plot of figure~\ref{fig:StageWPData}.}
    \label{fig:ShapeSys}
  \end{center}
\end{figure}

\vspace{-.2cm}
    \begin{block}{}
      \tiny \centering Dispersion of ratio between control sample and signal sample for shape data-driven estimation of background shape. A dispersion of 20\% is observed after $\Delta R (bb)$ cut. This figure corresponds to the projection on the y-axis of the ratio between signal and control samples shown in top right plot of validation figure
    \end{block}

\end{frame}

\begin{frame}{Results}
\vspace{-.2cm}
%\begin{table*}[htbH]
\begin{center}
\begin{tabular}{|c|c|c|c|}
\hline 
\Tp~Mass GeV/$c^{2}$ & Signal & Background & Observed Data\\
\hline 
600 & 4.39$\pm$0.35 & 11.31 & 15 \\
650 & 6.64$\pm$0.38 & 11.83 & 15 \\
700 & 7.84$\pm$0.37 & 13.06 & 22 \\
750 & 7.24$\pm$0.32 & 12.24 & 17 \\
800 & 6.06$\pm$0.28 & 8.54 & 6 \\
850 & 5.65$\pm$0.24 & 9.36 & 4 \\
900 & 4.78$\pm$0.21 & 7.30 & 8 \\
950 & 3.80$\pm$0.17 & 8.43 & 9 \\
1000 & 2.28$\pm$0.12 & 5.24 & 7 \\
\hline
\end{tabular}
%\caption{Expected number of events for the signal, estimated background and observed data after full selection in 1-$\sigma$ integration window. The errors for signal yields represent only the statistical uncertainty. \label{tab:ExpEvts}}
\end{center}
%\end{table*}

\vspace{-.2cm}
    \begin{block}{}
      \tiny \centering Expected number of events for the signal, estimated background and observed data after full selection in 1-$\sigma$ integration window. The errors for signal yields represent only the statistical uncertainty.
    \end{block}

\end{frame}

\begin{frame}{}
\vspace{-.2cm}
\begin{figure}[!Hhtbp]
  \begin{center}
    \includegraphics[width=0.5\textwidth]{../figs/Ana/Final_Plot_NoSubs.png}
    %\caption{$M(5j)$ after full selection for data and the estimated background and signal mass points at 650 \GeVcc~and 850 \GeVcc.}
    %\label{fig:FinalPlot2}
  \end{center}
\end{figure}

\vspace{-.2cm}
    \begin{block}{}
      \tiny \centering $M(5j)$ after full selection for data and the estimated background and signal mass points at 650 \GeVcc~and 850 \GeVcc.
    \end{block}

\end{frame}

\begin{frame}{Single production analyses}
\vspace{-.2cm}
\begin{figure}[!Hhtbp]
  \begin{center}
    \includegraphics[width=0.5\textwidth]{../figs/Ana/fig_15b.png}
    %\caption{Upper limit (at 95\% CL) on the \Tp~single production mode cross section times the branching ratio as a function of its mass.}
    %\label{fig:SingleATLASres}
  \end{center}
\end{figure}

\vspace{-.2cm}
    \begin{block}{}
      \tiny \centering Upper limit (at 95\% CL) on the \Tp~single production mode cross section times the branching ratio as a function of its mass.
    \end{block}

\end{frame}

\begin{frame}{1.5 sigma window}
\vspace{-.2cm}
\begin{center}
\resizebox{\textwidth}{!}{
\begin{tabular}{|c|c|c|c|c|c|c|c|}
\hline 
Sample Name & \multicolumn{2}{c|}{b-tagging} & \multicolumn{2}{c|}{JEC} & PDF+$\alpha_{S}$ & Pileup & Trigger \\
\hline
 & up [\%] & down [\%] & up [\%] & down [\%] & up/down [\%] & up/down [\%] & up/down [\%] \\
\hline
$Tj\rightarrow tHj$ 600 GeV/$c^{2}$ & 7.28 & 7.28 & 10.79 & 11.18 & 7.28 & 7.28 & 7.28 \\
$Tj\rightarrow tHj$ 650 GeV/$c^{2}$ & 6.15 & 5.92 & 8.25 & 10.52 & 5.29 & 5.29 & 5.29 \\
$Tj\rightarrow tHj$ 700 GeV/$c^{2}$ & 6.12 & 5.91 & 4.51 & 5.16 & 4.51 & 4.51 & 4.51 \\
$Tj\rightarrow tHj$ 750 GeV/$c^{2}$ & 6.29 & 6.05 & 4.68 & 8.03 & 4.46 & 4.27 & 4.27 \\
$Tj\rightarrow tHj$ 800 GeV/$c^{2}$ & 6.33 & 6.10 & 5.47 & 7.07 & 4.22 & 4.22 & 4.22 \\
$Tj\rightarrow tHj$ 850 GeV/$c^{2}$ & 6.70 & 6.43 & 4.00 & 4.36 & 4.00 & 4.00 & 4.00 \\
$Tj\rightarrow tHj$ 900 GeV/$c^{2}$ & 6.74 & 6.46 & 3.94 & 3.94 & 3.94 & 3.94 & 3.94 \\
$Tj\rightarrow tHj$ 950 GeV/$c^{2}$ & 6.95 & 6.66 & 4.29 & 6.15 & 4.29 & 4.29 & 4.29 \\
$Tj\rightarrow tHj$ 1000 GeV/$c^{2}$ & 6.86 & 6.58 & 5.04 & 5.59 & 4.81 & 4.81 & 4.81 \\
\hline
\end{tabular}
}
\end{center}

\vspace{-.2cm}
    \begin{block}{}
      \tiny \centering Summary of uncertainties for signal samples with $1.5\sigma$ integration window. In addition, all mass points have 2.6\% uncertainty from luminosity measurement.
    \end{block}

\end{frame}

\begin{frame}{}
\vspace{-.2cm}
\begin{center}
\resizebox{0.5\textwidth}{!}{
\begin{tabular}{|c|c|c|c|}
\hline 
\Tp~Mass GeV/$c^{2}$ & Signal & Background & Observed Data\\
\hline 
600 & 5.16$\pm$0.38 & 16.97 & 22 \\
650 & 7.78$\pm$0.41 & 18.92 & 28 \\
700 & 8.59$\pm$0.39 & 17.38 & 25 \\
750 & 7.93$\pm$0.34 & 14.60 & 18 \\
800 & 7.08$\pm$0.30 & 14.91 & 14 \\
850 & 6.42$\pm$0.26 & 10.69 & 4 \\
900 & 5.69$\pm$0.22 & 11.11 & 11 \\
950 & 4.28$\pm$0.18 & 11.00 & 10 \\
1000 & 2.84$\pm$0.14 & 8.33 & 7 \\
\hline
\end{tabular}
}
\end{center}
%\begin{figure}[!Hhtbp]
  \begin{center}
    \includegraphics[width=0.5\textwidth]{../figs/Ana/Limits_from_CLs_V8_LinearFitWidths_1p5sigma_Revision.png}
    %\caption{Expected and observed limits in terms of \Tp~production cross section as function of $M(5j)$ with 1.5 sigma integration window. The red line represents the theoretical prediction of the cross section~\cite{Buchkremer:2013bha, Cacciapaglia:2011fx}. The 850~\GeVcc~mass point is excluded at 95\% CL. With a linear approximation it is possible to exclude masses between 830 and 870 \GeVcc~at 95\% CL.}
    %\label{fig:Lim1p5}
  \end{center}
%\end{figure}

\vspace{-.2cm}
    \begin{block}{}
      \tiny \centering Expected and observed limits in terms of \Tp~production cross section as function of $M(5j)$ with 1.5 sigma integration window. The red line represents the theoretical prediction of the cross section. The 850~\GeVcc~mass point is excluded at 95\% CL. With a linear approximation it is possible to exclude masses between 830 and 870 \GeVcc~at 95\% CL.
    \end{block}

\end{frame}

\begin{frame}{2 sigma window}
\vspace{-.2cm}
\begin{center}
\resizebox{\textwidth}{!}{
\begin{tabular}{|c|c|c|c|c|c|c|c|}
\hline 
Sample Name & \multicolumn{2}{c|}{b-tagging} & \multicolumn{2}{c|}{JEC} & PDF+$\alpha_{S}$ & Pileup & Trigger \\
\hline
 & up [\%] & down [\%] & up [\%] & down [\%] & up/down [\%] & up/down [\%] & up/down [\%] \\
\hline
$Tj\rightarrow tHj$ 600 GeV/$c^{2}$ & 7.28 & 7.28 & 10.80 & 11.18 & 7.28 & 7.28 & 7.28 \\
$Tj\rightarrow tHj$ 650 GeV/$c^{2}$ & 6.16 & 5.94 & 6.71 & 10.51 & 5.17 & 5.17 & 5.17 \\
$Tj\rightarrow tHj$ 700 GeV/$c^{2}$ & 6.15 & 5.93 & 4.44 & 6.24 & 4.44 & 4.44 & 4.44 \\
$Tj\rightarrow tHj$ 750 GeV/$c^{2}$ & 6.25 & 6.02 & 4.65 & 7.32 & 4.12 & 4.12 & 4.12 \\
$Tj\rightarrow tHj$ 800 GeV/$c^{2}$ & 6.33 & 6.10 & 4.13 & 5.34 & 4.13 & 4.13 & 4.13 \\
$Tj\rightarrow tHj$ 850 GeV/$c^{2}$ & 6.67 & 6.40 & 4.25 & 5.43 & 3.84 & 3.84 & 3.84 \\
$Tj\rightarrow tHj$ 900 GeV/$c^{2}$ & 6.72 & 6.44 & 3.87 & 3.87 & 3.87 & 3.87 & 3.87 \\
$Tj\rightarrow tHj$ 950 GeV/$c^{2}$ & 6.95 & 6.66 & 4.29 & 6.15 & 4.29 & 4.29 & 4.29 \\
$Tj\rightarrow tHj$ 1000 GeV/$c^{2}$ & 6.85 & 6.58 & 8.22 & 4.65 & 4.65 & 4.65 & 4.65 \\
\hline
\end{tabular}
}
\end{center}

\vspace{-.2cm}
    \begin{block}{}
      \tiny \centering Summary of uncertainties for signal samples with $2\sigma$ integration window. In addition, all mass points have 2.6\% uncertainty from luminosity measurement.
    \end{block}

\end{frame}

\begin{frame}{}
\vspace{-.2cm}
\begin{center}
\resizebox{0.5\textwidth}{!}{
\begin{tabular}{|c|c|c|c|}
\hline 
\Tp~Mass GeV/$c^{2}$ & Signal & Background & Observed Data\\
\hline 
600 & 5.16$\pm$0.38 & 16.97 & 22 \\
650 & 8.15$\pm$0.42 & 22.21 & 33 \\
700 & 8.87$\pm$0.39 & 19.95 & 25 \\
750 & 8.52$\pm$0.35 & 19.44 & 24 \\
800 & 7.38$\pm$0.31 & 19.02 & 21 \\
850 & 6.95$\pm$0.27 & 14.40 & 13 \\
900 & 5.91$\pm$0.23 & 14.81 & 11 \\
950 & 4.28$\pm$0.18 & 11.00 & 10 \\
1000 & 3.04$\pm$0.14 & 11.93 & 9 \\
\hline
\end{tabular}
}
\end{center}
%\begin{figure}[!Hhtbp]
  \begin{center}
    \includegraphics[width=0.5\textwidth]{../figs/Ana/Limits_from_CLs_V8_LinearFitWidths_2sigma_Revision.png}
    %\caption{Expected and observed limits in terms of \Tp~production cross section as function of $M(5j)$ with 2 sigma integration window. The red line represents the theoretical prediction of the cross section~\cite{Buchkremer:2013bha, Cacciapaglia:2011fx}. No observed exclusion limits are reached.}
    %\label{fig:Lim2}
  \end{center}
%\end{figure}

\vspace{-.2cm}
    \begin{block}{}
      \tiny \centering Expected and observed limits in terms of \Tp~production cross section as function of $M(5j)$ with 2 sigma integration window. The red line represents the theoretical prediction of the cross section. No observed exclusion limits are reached.
    \end{block}

\end{frame}

%\begin{frame}{}
%\vspace{-.2cm}
%
%
%\vspace{-.2cm}
%    \begin{block}{}
%      \tiny \centering 
%    \end{block}
%
%\end{frame}