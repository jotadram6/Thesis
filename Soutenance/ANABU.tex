\subsection{Analysis}

\begin{frame}{Datasets - Backgrounds}
\vspace{-.2cm}

%\begin{table*}[htbH]
\begin{center}
\resizebox{\textwidth}{!}{
\begin{tabular}{|c|c|c|}
\hline 
Samples & Cross-Section (pb) & Number of events\\
\hline
QCD\_Pt-120to170\_TuneZ2star\_8TeV\_pythia6 & 16\(\times 10^4\) & 5.9M\\
QCD\_Pt-170to300\_TuneZ2star\_8TeV\_pythia6 & 34\(\times 10^3\) & 5.8M\\
QCD\_Pt-300to470\_TuneZ2star\_8TeV\_pythia6 & 18\(\times 10^2\) & 5.9M\\ 
QCD\_Pt-470to600\_TuneZ2star\_8TeV\_pythia6 & 114 & 3.9M\\
QCD\_Pt-600to800\_TuneZ2star\_8TeV\_pythia6 & 27 & 3.9M\\
QCD\_Pt-800to1000\_TuneZ2star\_8TeV\_pythia6 & 3.5 & 3.9M\\
QCD\_HT-500To1000\_TuneZ2star\_8TeV-madgraph-pythia6 & 84\(\times 10^2\) & 30M\\ 
QCD\_HT-1000ToInf\_TuneZ2star\_8TeV-madgraph-pythia6 & 2\(\times 10^2\) & 14M\\ 
DYToCC\_M\_50\_TuneZ2star\_8TeV\_pythia6 & 31\(\times 10^2\) & 2M\\
DYToBB\_M\_50\_TuneZ2star\_8TeV\_pythia6 & 38\(\times 10^2\) & 2M\\
TTJets\_MSDecays\_central\_TuneZ2star\_8TeV-madgraph-tauola & 247.7 [NNLO] & 62M\\
%TT\_CT10\_TuneZ2star\_8TeV-powheg-tauola & 247.7 [NNLO] & 22M\\
T\_tW-channel-DR\_TuneZ2star\_8TeV-powheg-tauola & 11.1 [NNLO] &497k\\
T\_s-channel\_TuneZ2star\_8TeV-powheg-tauola & 3.79 [NNLO] & 260k\\
T\_t-channel\_TuneZ2star\_8TeV-powheg-tauola & 54.9 [NNLO] & 3.7M\\
Tbar\_tW-channel-DR\_TuneZ2star\_8TeV-powheg-tauola & 11.1 [NNLO] & 493k\\
Tbar\_s-channel\_TuneZ2star\_8TeV-powheg-tauola & 1.76 [NNLO] & 140k\\
Tbar\_t-channel\_TuneZ2star\_8TeV-powheg-tauola & 29.7 [NNLO] & 1.9M\\
WZ\_TuneZ2star\_8TeV\_pythia6\_tauola & 33.6 [NLO] & 10M\\
ZZ\_TuneZ2star\_8TeV\_pythia6\_tauola & 7.6 [NLO] & 9.8M\\
WW\_TuneZ2star\_8TeV\_pythia6\_tauola & 56 [NLO] & 10M\\
TTH\_Inclusive\_M-125\_8TeV\_pythia6 & 0.13 [NLO] & 100K\\
\hline
\end{tabular}
}
%\caption{List of Monte-Carlo background samples used in the analysis, their corresponding cross-section and their number of events.\label{tab:MCbkg}}
\end{center}
%\end{table*}

\end{frame}

\begin{frame}{Datasets - Signal}
\vspace{-.2cm}

%\begin{table*}[htbH]
\begin{center}
\resizebox{\textwidth}{!}{
\begin{tabular}{|c|c|c|c|}
\hline 
Sample & \Tp~Mass & Cross-Section & Number of events\\
            & (GeV$/c^{2}$) &  (fb) & \\
\hline
TprimeJetToTH\_M-600\_TuneZ2star\_8TeV-madgraph\_tauola & 600 & 215.4 & 95K \\
TprimeJetToTH\_M-650\_TuneZ2star\_8TeV-madgraph\_tauola & 650 & 177.8 & 99K \\
TprimeJetToTH\_M-700\_TuneZ2star\_8TeV-madgraph\_tauola & 700 & 143.7 & 99K \\
TprimeJetToTH\_M-750\_TuneZ2star\_8TeV-madgraph\_tauola & 750 & 118.6 & 99K \\
TprimeJetToTH\_M-800\_TuneZ2star\_8TeV-madgraph\_tauola & 800 & 100 & 96K \\
TprimeJetToTH\_M-850\_TuneZ2star\_8TeV-madgraph\_tauola & 850 & 84.3 & 99K \\
TprimeJetToTH\_M-900\_TuneZ2star\_8TeV-madgraph\_tauola & 900 & 72.6 & 99K \\
TprimeJetToTH\_M-950\_TuneZ2star\_8TeV-madgraph\_tauola & 950 & 62.6 & 96K \\
TprimeJetToTH\_M-1000\_TuneZ2star\_8TeV-madgraph\_tauola & 1000 & 53.9 & 99K \\
\hline
\end{tabular}
}
%\caption{List of Monte-Carlo signal samples used in the analysis, their corresponding cross-section and \Tp~mass.\label{tab:MCsig}}
\end{center}
%\end{table*}

\end{frame}

\begin{frame}{PU corrections}
\vspace{-.2cm}

\begin{figure}[!Hhtbp]
  \begin{center}
    \includegraphics[width=0.49\textwidth]{../figs/Ana/DataPU40.png}
    \includegraphics[width=0.49\textwidth]{../figs/Ana/MCPU40.png}\\
    \includegraphics[width=0.5\textwidth]{../figs/Ana/WeightPU40.png}
    \begin{block}{}
      \tiny \centering Pileup for data [up-left], MC S10 [up-right] and ratio between them [bottom].
    \end{block}
    %\caption{Pileup for data [up-left], MC S10 [up-right] and ratio between them [bottom].}
    %\label{fig:PU_distros}
  \end{center}
\end{figure}

\end{frame}

\begin{frame}{Six leading jets \pt}
\vspace{-.2cm}
\begin{figure}[!Hhtbp]
  \begin{center}
    \includegraphics[width=0.35\textwidth, height=0.65\textheight]{../figs/Ana/jet1pt.png}
    \includegraphics[width=0.35\textwidth, height=0.65\textheight]{../figs/Ana/jet2pt.png}
    \includegraphics[width=0.35\textwidth, height=0.65\textheight]{../figs/Ana/jet3pt.png}
    %\caption{Distribution of transverse momentum of the 6 leading jets. The gray band represents the statistical uncertainties from the sum of the MC background. Reasonable agreement is observed, with the multijet process as the dominant process at this stage. Normalization of samples was done to the 19.7~fb$^{-1}$.}
    %\label{fig:6jpt}
  \end{center}
\end{figure}

\vspace{-.2cm}
    \begin{block}{}
      \tiny \centering Distribution of transverse momentum of the 3 leading jets. The gray band represents the statistical uncertainties from the sum of the MC background. Reasonable agreement is observed, with the multijet process as the dominant process at this stage. Normalization of samples was done to the 19.7~fb$^{-1}$.
    \end{block}

\end{frame}

\begin{frame}{}
\vspace{-.2cm}
\begin{figure}[!Hhtbp]
  \begin{center}
    \includegraphics[width=0.35\textwidth, height=0.65\textheight]{../figs/Ana/jet4pt.png}
    \includegraphics[width=0.35\textwidth, height=0.65\textheight]{../figs/Ana/jet5pt.png}
    \includegraphics[width=0.35\textwidth, height=0.65\textheight]{../figs/Ana/jet6pt.png}
    %\caption{Distribution of transverse momentum of the 6 leading jets. The gray band represents the statistical uncertainties from the sum of the MC background. Reasonable agreement is observed, with the multijet process as the dominant process at this stage. Normalization of samples was done to the 19.7~fb$^{-1}$.}
    %\label{fig:6jpt}
  \end{center}
\end{figure}

\vspace{-.2cm}
    \begin{block}{}
      \tiny \centering Distribution of transverse momentum of the 4th, 5th and 6th leading jets. The gray band represents the statistical uncertainties from the sum of the MC background. Reasonable agreement is observed, with the multijet process as the dominant process at this stage. Normalization of samples was done to the 19.7~fb$^{-1}$.
    \end{block}

\end{frame}

\begin{frame}{Six leading jets $\eta$}
\vspace{-.2cm}
\begin{figure}[!Hhtbp]
  \begin{center}
    \includegraphics[width=0.35\textwidth, height=0.65\textheight]{../figs/Ana/jet1eta.png}
    \includegraphics[width=0.35\textwidth, height=0.65\textheight]{../figs/Ana/jet2eta.png}
    \includegraphics[width=0.35\textwidth, height=0.65\textheight]{../figs/Ana/jet3eta.png}
    %\caption{Distribution of $\eta$ of the 6 leading jets. The gray band represents the statistical uncertainties from the sum of the MC background. Reasonable agreement is observed, with the multijet process as the dominant process at this stage. Normalization of samples was done to luminosity.}
    %\label{fig:6jeta}
  \end{center}
\end{figure}

\vspace{-.2cm}
    \begin{block}{}
      \tiny \centering Distribution of $\eta$ of the 3 leading jets. The gray band represents the statistical uncertainties from the sum of the MC background. Reasonable agreement is observed, with the multijet process as the dominant process at this stage. Normalization of samples was done to luminosity.
    \end{block}

\end{frame}

\begin{frame}{}
\vspace{-.2cm}
\begin{figure}[!Hhtbp]
  \begin{center}
    \includegraphics[width=0.35\textwidth, height=0.65\textheight]{../figs/Ana/jet4eta.png}
    \includegraphics[width=0.35\textwidth, height=0.65\textheight]{../figs/Ana/jet5eta.png}
    \includegraphics[width=0.35\textwidth, height=0.65\textheight]{../figs/Ana/jet6eta.png}
    %\caption{Distribution of $\eta$ of the 6 leading jets. The gray band represents the statistical uncertainties from the sum of the MC background. Reasonable agreement is observed, with the multijet process as the dominant process at this stage. Normalization of samples was done to luminosity.}
    %\label{fig:6jeta}
  \end{center}
\end{figure}

\vspace{-.2cm}
    \begin{block}{}
      \tiny \centering Distribution of $\eta$ of the 4th, 5th and 6th leading jets. The gray band represents the statistical uncertainties from the sum of the MC background. Reasonable agreement is observed, with the multijet process as the dominant process at this stage. Normalization of samples was done to luminosity.
    \end{block}

\end{frame}

\begin{frame}{B-tagging working point study}
\vspace{-.2cm}
\begin{table}[htbH]
\begin{center}
\resizebox{\textwidth}{!}{
\begin{tabular}{| c || c | c | c | c | c | c |}
\hline 
\textit{At least} & $\epsilon(S)$ [\%] & $\epsilon(t\bar{t})$ [\%] & $\epsilon(\text{QCD\_HT-500To1000})$ [\%] & $\epsilon(\text{QCD\_HT-1000ToInf})$ [\%] & $\frac{S}{B}\times 10^{3}$ & $\frac{S}{\sqrt{S+B}}\times 10^{2}$ \\
\hline
3 CSVL                       & $65 \pm 0.4$  & $38 \pm 0.04$  & $6 \pm 0.02$    & $7 \pm 0.02$     & $0.4 \pm 0.005$  & $24.8 \pm 0.3$ \\
3 CSVM                       & $31 \pm 0.4$  & $8 \pm 0.02$   & $1 \pm 0.01$    & $0.6 \pm 0.01$   & $1.8 \pm 0.05$   & $38.2 \pm 0.8$ \\
1 CSVL and 2 CSVM            & $55 \pm 0.4$  & $27 \pm 0.03$  & $2 \pm 0.01$    & $2 \pm 0.01$     & $0.8 \pm 0.01$   & $33.7 \pm 0.5$ \\
2 CSVL and 1 CSVM            & $64 \pm 0.4$  & $37 \pm 0.04$  & $5 \pm 0.02$    & $5 \pm 0.02$     & $0.5 \pm 0.007$  & $28.3 \pm 0.3$  \\
2 CSVM and 1 CSVT            & $29 \pm 0.4$  & $8 \pm 0.02$   & $0.5 \pm 0.006$ & $0.5 \pm 0.006$  & $1.9 \pm 0.05$   & $38.4 \pm 0.8$  \\
1 CSVM and 2 CSVT            & $22 \pm 0.4$  & $5 \pm 0.02$   & $0.3 \pm 0.005$ & $0.3 \pm 0.003$  & $2.2 \pm 0.07$   & $35.8 \pm 0.9$  \\
3 CSVT                       & $9 \pm 0.2$   & $1 \pm 0.01$   & $0.1 \pm 0.003$ & $0.09 \pm 0.002$ & $3.1 \pm 0.2$    & $27.3 \pm 1.1$  \\
1 CSVL and 2 CSVT            & $33 \pm 0.4$  & $13 \pm 0.03$  & $0.9 \pm 0.01$  & $0.8 \pm 0.007$  & $1.1 \pm 0.03$   & $31.5 \pm 0.6$  \\
2 CSVL and 1 CSVT            & $57 \pm 0.4$  & $30 \pm 0.03$  & $3 \pm 0.02$    & $3 \pm 0.01$     & $0.7 \pm 0.01$   & $31.5 \pm 0.4$  \\
1 CSVL and 1 CSVM and 1 CSVT &  $51 \pm 0.4$ &  $24 \pm 0.03$ & $2 \pm 0.01$    & $2 \pm 0.01$     & $0.9 \pm 0.02$   & $30.8 \pm 0.4$  \\
\hline
\end{tabular}
}
%\caption{Comparative study of different possible combinations to require at least 3 b-tagged jets with CSVL, CSVM and CSVT working points. Efficiencies of cuts over signal and principal MC background samples are presented, as well as $\frac{S}{B}$ and $\frac{S}{S+B}$. High values of $\frac{S}{S+B}$ point to a good discrimination while keeping the signal efficiency high.\label{tab:BCutStudy}}
\end{center}
\end{table}

\vspace{-.2cm}
    \begin{block}{}
      \tiny \centering Comparative study of different possible combinations to require at least 3 b-tagged jets with CSVL, CSVM and CSVT working points. Efficiencies of cuts over signal and principal MC background samples are presented, as well as $\frac{S}{B}$ and $\frac{S}{S+B}$. High values of $\frac{S}{S+B}$ point to a good discrimination while keeping the signal efficiency high.
    \end{block}

\end{frame}

\begin{frame}{B-tagging efficiencies}
\vspace{-.2cm}
\begin{figure}[!Hhtbp]
  \begin{center}
    \includegraphics[width=0.5\textwidth, height=0.4\textheight]{../figs/Ana/ttbar_beff.png}
    \includegraphics[width=0.5\textwidth, height=0.4\textheight]{../figs/Ana/ttbar_ceff.png}\\
    \includegraphics[width=0.5\textwidth, height=0.4\textheight]{../figs/Ana/ttbar_leff.png}
    %\caption{CSVM b-tagging efficiency for b-jets [left], c-jets [center] and light jets [right] as function of \pt~and $\eta$ for \ttbar.}
    %\label{fig:ttbarBEff}
  \end{center}
\end{figure}

\vspace{-.5cm}
    \begin{block}{}
      \tiny \centering CSVM b-tagging efficiency for b-jets [left], c-jets [right] and light jets [bottom] as function of \pt~and $\eta$ for \ttbar.
    \end{block}

\end{frame}

\begin{frame}{Jet multiplicity after basic selection}
\vspace{-.2cm}
\begin{figure}[!Hhtbp]
  \begin{center}
    \includegraphics[width=0.48\textwidth]{../figs/Ana/Nj_Nm1.png}
    %\caption{Jet multiplicity for MC samples after requiring at least 3 CSVM b-tagged jets. The sum of MC samples is normalized to the integrated luminosity. Signal is overlaid.}
    %\label{fig:Nj}
  \end{center}
\end{figure}

\vspace{-.2cm}
    \begin{block}{}
      \tiny \centering Jet multiplicity for MC samples after requiring at least 3 CSVM b-tagged jets. The sum of MC samples is normalized to the integrated luminosity. Signal is overlaid.
    \end{block}

\end{frame}

\begin{frame}{B-tagging efficiencies}
\vspace{-.2cm}
\begin{figure}[!Hhtbp]
  \begin{center}
    \includegraphics[width=0.5\textwidth, height=0.4\textheight]{../figs/Ana/ttbar_beff.png}
    \includegraphics[width=0.5\textwidth, height=0.4\textheight]{../figs/Ana/ttbar_ceff.png}\\
    \includegraphics[width=0.5\textwidth, height=0.4\textheight]{../figs/Ana/ttbar_leff.png}
    %\caption{CSVM b-tagging efficiency for b-jets [left], c-jets [center] and light jets [right] as function of \pt~and $\eta$ for \ttbar.}
    %\label{fig:ttbarBEff}
  \end{center}
\end{figure}

\vspace{-.5cm}
    \begin{block}{}
      \tiny \centering CSVM b-tagging efficiency for b-jets [left], c-jets [center] and light jets [right] as function of \pt~and $\eta$ for \ttbar.
    \end{block}

\end{frame}

\begin{frame}{Reconstructed resonances by the $\chi^{2}$ sorting algorithm}
\vspace{-.2cm}

\begin{figure}[!Hhtbp]
  \begin{center}
    \includegraphics[width=0.35\textwidth]{../figs/Ana/TopMass_S700.png}
    \includegraphics[width=0.35\textwidth]{../figs/Ana/WMass_S700.png}
    \includegraphics[width=0.35\textwidth]{../figs/Ana/HiggsMass_S700.png}
    %\caption{Reconstructed top, \W~and \Hb~masses for the \Tp~mass point of 700 \GeVcc. The reconstructed masses and widths of the three resonances, $M^{reco}_{H}=124.92\pm0.26$~\GeVcc, $M^{reco}_{W}=85.06\pm0.26$~\GeVcc, $M^{reco}_{t}=179.02\pm0.42$~\GeVcc, $\sigma^{reco}_{H}=13.50\pm0.27$~\GeVcc, and $\sigma^{reco}_{W}=11.03\pm0.28$~\GeVcc~and $\sigma^{reco}_{t}=18.10\pm0.42$~\GeVcc. The corresponding values used for the reconstruction procedure are: $M_{H}=125$~\GeVcc, $M_{W}=84.06$~\GeVcc, $M_{t}=175.16$~\GeVcc, $\sigma_{H}=12.4$~\GeVcc, and $\sigma_{W}=10.12$~\GeVcc~and $\sigma_{t}=17.35$~\GeVcc.}
    %\label{fig:WHt}
  \end{center}
\end{figure}

\vspace{-.2cm}
    \begin{block}{}
      \tiny \centering Reconstructed top, \W~and \Hb~masses for the \Tp~mass point of 700 \GeVcc. The reconstructed masses and widths of the three resonances, $M^{reco}_{H}=124.92\pm0.26$~\GeVcc, $M^{reco}_{W}=85.06\pm0.26$~\GeVcc, $M^{reco}_{t}=179.02\pm0.42$~\GeVcc, $\sigma^{reco}_{H}=13.50\pm0.27$~\GeVcc, and $\sigma^{reco}_{W}=11.03\pm0.28$~\GeVcc~and $\sigma^{reco}_{t}=18.10\pm0.42$~\GeVcc. The corresponding values used for the reconstruction procedure are: $M_{H}=125$~\GeVcc, $M_{W}=84.06$~\GeVcc, $M_{t}=175.16$~\GeVcc, $\sigma_{H}=12.4$~\GeVcc, and $\sigma_{W}=10.12$~\GeVcc~and $\sigma_{t}=17.35$~\GeVcc.
    \end{block}

\end{frame}

\begin{frame}{}
\vspace{-.2cm}

\begin{columns}
\begin{column}{.50\textwidth}

\begin{figure}[!Hhtbp]
  \begin{center}
    %\includegraphics[width=1.0\textwidth]{../figs/Ana/Exclusive_Efficiency_V8.png}
    \includegraphics[width=1.0\textwidth]{../figs/Ana/Inclusive_Efficiency_V8.png}
    %\caption{Reconstruction efficiency by the $\chi^{2}$ algorithm of the Higgs boson, \W~boson, top quark and \Tp, as the ratio of the number of events where the particle was correctly reconstructed to the number of events where jets could be matched to partons [left] and to the total number of events [right]}
    %\label{fig:RecEff}
  \end{center}
\end{figure}

\vspace{-.2cm}
\begin{block}{}
\scriptsize \textbf{Figure}: Reconstruction efficiency by the $\chi^{2}$ algorithm of the Higgs boson, \W~boson, top quark and \Tp, as the ratio of the number of events where the particle was correctly reconstructed to the total number of events.
\end{block}
\end{column}

\begin{column}{.50\textwidth}

\begin{figure}[!Hhtbp]
  \begin{center}
    %\includegraphics[width=1.0\textwidth]{../figs/Ana/HundresdsMassChi2Tp.png}
    \includegraphics[width=1.0\textwidth]{../figs/Ana/FiftiesMassChi2Tp.png}
    %\caption{Reconstructed \Tp~mass for all mass points from the $\chi^{2}$ sorting algorithm after basic selection. Each mass point is normalized to luminosity and its corresponding cross section. A gaussian fit of these distributions will be presented afterward in section~\ref{sec:finalsel}, accompanied with a discussion about the resolution on the reconstruction of the \Tp.}
    %\label{fig:RecT}
  \end{center}
\end{figure}

\vspace{-.2cm}
\begin{block}{}
\scriptsize \textbf{Figure}: Reconstructed \Tp~mass for some mass points from the $\chi^{2}$ sorting algorithm after basic selection. Each mass point is normalized to luminosity and its corresponding cross section.
\end{block}
\end{column}

\end{columns}
\end{frame}