\documentclass[usenames,dvipsnames]{beamer}

% French characters
\usepackage[utf8x]{inputenc}
\usepackage[T1]{fontenc}
% French display
\usepackage[english]{babel}
% Math package
\usepackage{amsmath}
\usepackage{amsfonts}
\usepackage{amssymb}
\usepackage{marvosym}
% includegraphics
\usepackage{graphicx}
% units
%\usepackage{hepunits}
% links
\usepackage{hyperref}
% \usepackage[bookmarksopen,bookmarksopenlevel=1]{hyperref}
\usepackage{colortbl}
\usepackage{tikz}
\usepackage{rotating}
% Theme & Color theme
\usepackage{default}
\usetheme{Frankfurt}
\usecolortheme{perso}
% My green
\definecolor{MyColor}{RGB}{43,153,43}
\setbeamercolor{structure}{fg=MyColor}
\setbeamertemplate{title page}[default]
\setbeamercolor{title}{bg=}

\usepackage{multirow}

% \setbeamercolor{titlelike}{fg=MyColor}
% \setbeamercolor{palette primary}{fg=MyColor}
% \setbeamercolor{palette secondary}{fg=MyColor}
% \setbeamercolor{palette tertiary}{bg=MyColor}
% \setbeamercolor{palette quaternary}{bg=MyColor}
% Font
\usepackage{times}
\usefonttheme[onlymath]{serif}
% Suppress the navigation bar
% \beamertemplatenavigationsymbolsempty
\setbeamertemplate{navigation symbols}{}
%Description environment
%\usepackage{enumitem}
% Outerthemes
% \useoutertheme[footline=empty, subsection=false]{miniframes}
% \useoutertheme[subsection=false]{miniframes}
% \useoutertheme[subsection=false]{smoothbars}
% gets rid of bottom navigation bars
\setbeamertemplate{footline}[page number]{}
% 
% \setbeamercolor{frametitle}{parent=palette primary}
% \setbeamercolor{subsection in head/foot}{parent=palette secondary}
% \setbeamercolor{section in head}{parent=palette quaternary}
% \setbeamercolor{section in sidebar}{fg=black}
% \setbeamercolor{section in sidebar shaded}{fg= grey}
% Appendix page numbering
\usepackage{appendixnumberbeamer}
% feynmp for Feynman diagrams
%\usepackage{feynmp}
%\DeclareGraphicsRule{*}{mps}{*}{}
\makeatletter
\def\endfmffile{%
  \fmfcmd{\p@rcent\space the end.^^J%
          end.^^J%
          endinput;}%
  \if@fmfio
    \immediate\closeout\@outfmf
  \fi
  \ifnum\pdfshellescape=\@ne
    \immediate\write18{mpost \thefmffile}%
  \fi}
\makeatother
\usepackage[abs]{overpic}
\def\met{\ensuremath{\not\!\!{E_{T}}}}

\newcommand{\ttbar}{$t\bar{t}$}
\newcommand{\bbbar}{$b\bar{b}$}
\newcommand{\GeVcc}{GeV/$\text{c}^2$}
\newcommand{\pt}{$p_{T}$}
\newcommand{\ptg}[1]{$p_{T}>#1$~GeV/c}
\newcommand{\etal}[1]{$|\eta|<#1$}
\newcommand{\etag}[1]{$|\eta|>#1$}
\newcommand{\HT}{$H_{T}$}
\newcommand{\MET}{$\slashed{E}_{T}$}
\newcommand{\METg}[1]{$\slashed{E}_{T}>#1$~GeV/c}
\newcommand{\HTg}[1]{${H_{T}>#1}$~GeV/c}
\newcommand{\jn}[1]{$j_{#1}$}
\newcommand{\ex}[1]{\times 10^{#1}}
\newcommand{\Z}{$Z^{0}$}
\newcommand{\tq}{$t$}
\newcommand{\W}{$W^{+/-}$}
\newcommand{\Hb}{$H^{0}$}
\newcommand{\Tp}{$T'$}
\newcommand{\Tj}{$T'j$}
\newcommand{\Tjj}{$T'jj$}

%% Title and such
\title{\texorpdfstring{Search for a vector-like quark \Tp~decaying \\into top+Higgs in single production mode \\in full hadronic final state using \\CMS data collected at 8 TeV\\}}
%\title{\texorpdfstring{Recherche d'un quark vectoriel \Tp~\\qui se désintègre en top+Higgs dans \\le mode de production célibataire \\dans le état final hadronique avec \\les données recueillies par l'éxperience CMS à 8 TeV\\}}
% \subtitle{
% }
\author[J.~Ruiz-\'{A}lvarez]{Jos\'{e} D. Ruiz-\'{A}lvarez}
\institute[IPN Lyon]{
% Institut de Physique Nucléaire de Lyon / Université Claude Bernard Lyon 1\\
\includegraphics[height=.6cm]{../Prelude/CNRS.png}\hspace{.5cm}
\includegraphics[height=.6cm]{udl.png}\hspace{.5cm}
\includegraphics[height=.6cm]{../Prelude/UCBL.png}\hspace{.5cm}
\includegraphics[height=.6cm]{EDPhast.jpg}
\\\vspace{.5cm}
\includegraphics[height=.7cm]{../Prelude/IPNL.png}\hspace{.5cm}
\includegraphics[height=.7cm]{../Prelude/CMS.png}
}
\date{21 octobre 2015}

% \mode<presentation> {
%   \usefonttheme{serif}
%   \useoutertheme[footline=authorinstitutetitle,subsection=false]{miniframes}
%   \usecolortheme{crane}
%   \useinnertheme[<shadow>]{rounded}
% }

% \setbeamercolor{boum}{bg=\color{white!40}, fg=black}
% 
% \addtobeamertemplate{block begin}{\pgfsetfillopacity{0.65}}{\pgfsetfillopacity{1}}
% 
% \newenvironment<>{varblock}[2][\textwidth]{%
%   \setlength{\textwidth}{#1}
%   \begin{actionenv}#3%
%     \def\insertblocktitle{#2}%
%     \par%
%     \usebeamertemplate{block begin}}
%   {\par%
%     \usebeamertemplate{block end}%
%   \end{actionenv}}
% 

\usepackage{fancybox}
\usepackage{framed,xcolor}
\colorlet{shadecolor}{white!60}
\usepackage{rotating}
% \usepackage[table]{xcolor} 

% \includeonly{11_StandardModel}
% \includeonly{11_StandardModel,12_LHC_CMS}
% \includeonly{12_LHC_CMS}
% \includeonly{11_StandardModel,12_LHC_CMS,13_Zgamma}
% \includeonly{13_Zgamma}
% \includeonly{14_Hgg,15_BSM}
% \includeonly{14_Hgg,BB_Hgg}
% \includeonly{13_Zgamma,14_Hgg,BB_Hgg}
% \includeonly{15_BSM,BB_BSM}
% \includeonly{BB_Zgamma}


\begin{document}

\begin{frame}[plain]
  \tikz[remember picture,overlay]
    \node (b) at
      (current page.center) [inner sep=0pt,opacity=.4]
%       (0.5\paperwidth,-0.5\paperheight) [inner sep=0pt,opacity=.4]
      {\begin{overpic}[height=.9\paperheight]{a-1_16_1_RhoPhi.png}\end{overpic}};
  \titlepage
\end{frame}

\setcounter{tocdepth}{1}
\begin{frame}{Outline}
\tableofcontents
\end{frame}


%%%%%%%%%%%%%%%%%%%%%%%%%%%%%%%%%%%%%%%%%%%%%%%%%%%%%%%%
%%%%%%%%%%%%%%%%%%%%%%%%%%%%%%%%%%%%%%%%%%%%%%%%%%%%%%%%
\section[SM]{The Standard Model and beyond}
\setcounter{tocdepth}{2}

\begin{frame}
\begin{center}
The Standard Model and beyond
\end{center}
\end{frame}

\subsection{SM}
\begin{frame}{The Standard Model}
\vspace{-.2cm}
\begin{columns}

\begin{column}{.50\textwidth}
\begin{block}{}
\begin{itemize}\scriptsize
%\item Theory developed from quantum theory and special relativity -> Quantum Field Theory (Feynman, Dirac, ...) + Continuous group symmetries (Noether, Yang, ...)
\item Model developed from quantum theory and special relativity $\to$ Quantum Field Theory + Continuous group symmetries
\item Fermions: matter components (electron, muon, quarks...)
\item Bosons: interaction mediators (\W, \Z, ...)
\item Extremely successful model: \\ quarks, CKM, \W~and \Z~masses, ...
\item But it is limited:
  \begin{itemize}\scriptsize
  \item Neutrino masses
  \item Dark matter
  \item \textbf{Hierarchy problem}
  \end{itemize}
\end{itemize}
\end{block}
\end{column}

\begin{column}{.50\textwidth}
\begin{figure}[!Hhtbp]
  \begin{center}
    \includegraphics[width=1.0\textwidth]{../figs/Standard_Model_of_Elementary_Particles.jpg}\\
    \vspace{.6cm}
    \includegraphics[width=0.8\textwidth]{../figs/HierarchyLoop.png}
    %\caption{Particle content of the standard model.}
    %\label{fig:SMContent}
  \end{center}
\end{figure}
\end{column}
\end{columns}
\end{frame}


\begin{frame}{Vector Like Quarks (VLQ)}
\vspace{-.3cm}
\begin{columns}

\begin{column}{.50\textwidth}
\begin{block}{}
\begin{itemize}\scriptsize
\item Motivated by the hierarchy problem $\to$ New states to cancel loop contributions
\item SM + not chiral quarks
\item Singlet, doublet or triplet representations.
  \begin{itemize}\scriptsize
  \item $X$ with 5/3 electric charge
  \item \textbf{\Tp~with 2/3 electric charge, as the top quark.}
  \item $B$ with -1/3 electric charge, as the bottom quark.
  \item $Y$ with -4/3 electric charge.
  \end{itemize}
%\item \textbf{Generically they can be mixed with the three SM-quark generations}
%\item Produced in pairs or in single production mode with a SM-quark
%\item $T'\to bW^{+/-}, tZ^{0}, tH^{0}$
\end{itemize}
\end{block}
\end{column}

\begin{column}{.50\textwidth}
\begin{figure}[!Hhtbp]
  \begin{center}
    \includegraphics[width=0.5\textwidth]{../figs/Gluon_fusion_T_pair.jpg}
    \includegraphics[width=0.5\textwidth]{../figs/Tchannel_T_single.jpg}
  \end{center}
\end{figure}
\vspace{-.2cm}
\begin{block}{}
\begin{itemize}\scriptsize
\item \textbf{Generically they can be mixed with the three SM-quark generations}
\item Produced in pairs or in single production mode with a SM-quark
\item $T'\to bW^{+/-}, tZ^{0}, tH^{0}$
\end{itemize}
\end{block}

\end{column}
\end{columns}

\vspace{-.2cm}
\begin{figure}[!Hhtbp]
  \begin{center}
    \includegraphics[width=0.4\textwidth]{../figs/pheno_prod_single_tp.png}
    \includegraphics[width=0.4\textwidth]{../figs/pheno_prod_pair_tp.png}
  \end{center}
\end{figure}

\end{frame}

\begin{frame}{SM at the LHC}
\vspace{-.3cm}
\begin{columns}

\begin{column}{.50\textwidth}
\begin{block}{Top quark production}
\begin{itemize}\tiny
\item Heaviest quark in the SM 
\item LHC as a top machine $\to$ 6 tops/s (5 from pair, 1 from single)
\item 8 TeV: $\sigma_{t\bar{t}}=247.47\pm12.37$ pb, $\sigma_{t,\; \text{s-channel}}=5.56\pm0.22$ pb, $\sigma_{t,\; \text{t-channel}}=84.34\pm1.69$ pb and $\sigma_{tW}=22.2\pm0.67$ pb
\item $t\to bW^{-}$ with $Br(W^{+/-}\to l\nu)=0.33$ and $Br(W^{+/-}\to q\bar{q}')=0.67$
\item $m_{t}=173.34\pm 0.76$ \GeVcc
\end{itemize}
\end{block}

\vspace{-.4cm}
\begin{figure}[!Hhtbp]
  \begin{center}
    \includegraphics[width=1.0\textwidth]{../figs/toplhcwg_ttxsec_sqrts_may2015.png}
    %\includegraphics[width=0.3\textwidth]{../figs/Gluon_fusion_top_pair.jpg}
  \end{center}
\end{figure}
\end{column}

\begin{column}{.50\textwidth}
\begin{block}{Higgs boson production}
\begin{itemize}\tiny
\item Heaviest boson in the SM 
\item Rare process: 20 pb at 8TeV
\item Many decay channels: $Br(H^{o}\to b\bar{b})=0.57$
\item $m_{H}=125.09\pm 0.24$ \GeVcc~and $\sigma_{H}<20$ MeV
\end{itemize}
\end{block}

\vspace{-.2cm}
\begin{figure}[!Hhtbp]
  \begin{center}
    \includegraphics[width=0.8\textwidth]{../figs/totalXS_LM.png}
  \end{center}
\end{figure}

\end{column}
\end{columns}

\end{frame}



%%%%%%%%%%%%%%%%%%%%%%%%%%%%%%%%%%%%%%%%%%%%%%%%%%%%%%%%
%%%%%%%%%%%%%%%%%%%%%%%%%%%%%%%%%%%%%%%%%%%%%%%%%%%%%%%%
\section[CMS]{The CMS experiment at LHC}
\setcounter{tocdepth}{2}

\begin{frame}

\end{frame}




%%%%%%%%%%%%%%%%%%%%%%%%%%%%%%%%%%%%%%%%%%%%%%%%%%%%%%%%
%%%%%%%%%%%%%%%%%%%%%%%%%%%%%%%%%%%%%%%%%%%%%%%%%%%%%%%%
\section[Pheno]{Feasibility study for a search of a \Tp~at LHC at 8 TeV}
\setcounter{tocdepth}{2}

\begin{frame}

\end{frame}




%%%%%%%%%%%%%%%%%%%%%%%%%%%%%%%%%%%%%%%%%%%%%%%%%%%%%%%%
%%%%%%%%%%%%%%%%%%%%%%%%%%%%%%%%%%%%%%%%%%%%%%%%%%%%%%%%
\section[MC]{Monte-Carlo event simulation}
\setcounter{tocdepth}{2}

\begin{frame}

\end{frame}




%%%%%%%%%%%%%%%%%%%%%%%%%%%%%%%%%%%%%%%%%%%%%%%%%%%%%%%%
%%%%%%%%%%%%%%%%%%%%%%%%%%%%%%%%%%%%%%%%%%%%%%%%%%%%%%%%
\section[Analysis]{Search for a single produced \Tp~decaying into top and Higgs in the full hadronic final state}
\setcounter{tocdepth}{2}

\begin{frame}
\begin{center}
Search for a single produced \Tp~decaying into top and Higgs in the full hadronic final state
\end{center}
\end{frame}


\subsection{Introduction - Analysis Strategy}
\begin{frame}{Introduction - Analysis Strategy}
\vspace{-.2cm}
\begin{columns}

\begin{column}{.50\textwidth}
\begin{block}{}
\begin{itemize}\scriptsize
\item Single produced \Tp~with an associated jet
\item Full hadronic final state: \\ $T'\to t H \to b W^{+} \bar{b} b \to b \bar{b} j j b$
\item Reconstruction of \Tp~mass: $M(5j)$
\item Main challenges:
  \begin{itemize}\scriptsize
  \item Huge backgrounds $\rightarrow$ Mainly QCD and \ttbar
  \item \Tp~reconstruction with high jet multiplicity
  \end{itemize}
\item Fundamental tools for background discrimination:
  \begin{itemize}\scriptsize
  \item B-tagged jets multiplicity
  \item \Tp~reconstruction procedure
  \end{itemize}
\end{itemize}
\end{block}
\end{column}

\begin{column}{.50\textwidth}
\begin{center}
\includegraphics[width=0.9\textwidth]{../figs/Tchannel_T_single.jpg}\\
\includegraphics[width=1.0\textwidth]{../figs/pheno_prod_single_tp.png}
\end{center}
\end{column}
\end{columns}

\end{frame}

\begin{frame}{Datasets}
\vspace{-.2cm}

%\begin{table*}[htbH]
\begin{center}
\resizebox{\textwidth}{!}{
\begin{tabular}{|c|c|}
\hline 
Dataset name & Int. Luminosity ($\text{pb}^{-1}$) \\
\hline
/MultiJet/Run2012A-22Jan2013-v1/AOD & 889.4 \\
/MultiJet1Parked/Run2012B-05Nov2012-v2/AOD & 4429.0 \\
/MultiJet1Parked/Run2012C-part1-05Nov2012-v2/AOD & 494.6 \\
/MultiJet1Parked/Run2012C-part2-05Nov2012-v2/AOD & 6654.0 \\
/MultiJet1Parked/Run2012D-part1-10Dec2012-v1/AOD & 5955.1 \\
/MultiJet1Parked/Run2012D-part2-17Jan2013-v1/AOD & 734.0 \\
/MultiJet1Parked/Run2012D-part2-PixelRecover-17Jan2013-v1 & 538.4 \\
\hline
\multicolumn{1}{|r|}{\textit{Total}} & 19694.5 \\
\hline
\end{tabular}
}
%\caption{List of Multijet Primary Dataset used in the analysis and the corresponding integrated luminosity calculated using the golden JSON (Java Script Object Notation) file. The golden JSON file contains the information about the luminosity sections considered as good for all runs. A good luminosity section is defined as a luminosity section where the detector was fully functioning, this is all subsystems were taking data and without problems.  \label{tab:datasets}}
\end{center}
%\end{table*}

%\begin{table*}[htbH]
\begin{center}
\resizebox{\textwidth}{!}{
\begin{tabular}{|c|c|c|}
\hline 
Samples & Cross-Section (pb) & Number of events\\
\hline
QCD\_Pt-120to170\_TuneZ2star\_8TeV\_pythia6 & 16\(\times 10^4\) & 5.9M\\
QCD\_Pt-170to300\_TuneZ2star\_8TeV\_pythia6 & 34\(\times 10^3\) & 5.8M\\
QCD\_Pt-300to470\_TuneZ2star\_8TeV\_pythia6 & 18\(\times 10^2\) & 5.9M\\ 
QCD\_Pt-470to600\_TuneZ2star\_8TeV\_pythia6 & 114 & 3.9M\\
QCD\_Pt-600to800\_TuneZ2star\_8TeV\_pythia6 & 27 & 3.9M\\
QCD\_Pt-800to1000\_TuneZ2star\_8TeV\_pythia6 & 3.5 & 3.9M\\
QCD\_HT-500To1000\_TuneZ2star\_8TeV-madgraph-pythia6 & 84\(\times 10^2\) & 30M\\ 
QCD\_HT-1000ToInf\_TuneZ2star\_8TeV-madgraph-pythia6 & 2\(\times 10^2\) & 14M\\ 
TTJets\_MSDecays\_central\_TuneZ2star\_8TeV-madgraph-tauola & 247.7 [NNLO] & 62M\\
TprimeJetToTH\_\textbf{M-700}\_TuneZ2star\_8TeV-madgraph\_tauola & 143.7 & 99K \\
\hline
\end{tabular}
}
%\caption{List of Monte-Carlo background samples used in the analysis, their corresponding cross-section and their number of events.\label{tab:MCbkg}}
\end{center}
%\end{table*}

\tiny{Signal samples were done with $T'\to tH$ with $H\to\tau^{+}\tau^{-}$ (6\%) and $H\to b\bar{b}$ (94\%). Correction with weight of 0.61 to obtain correct $Br(H\to b\bar{b})=0.57$. }

\end{frame}

\begin{frame}{Event selection}
\vspace{-.2cm}

\begin{columns}

\begin{column}{.55\textwidth}
\begin{block}{Event processing}
\begin{itemize}\scriptsize
  \item Data processed using ``golden'' JSON file: Consider only validated lumi sections
  \item PAT processing
    \begin{itemize}\tiny
    \item Jets reconstructed with PF algorithm and CHS
    \item Jets: \ptg{20} and $|\eta|<5$
    \item At least one good primary vertex: $\text{n.d.o.f.} \ge 4,\; |z|<24 \;\text{cm},\; |\rho|< 2 \;\text{cm}$
    \item Global tag: Calibration and alignment info for data, MC corrected to get close to data conditions
    \end{itemize}
  \item Pile-up corrections: Simulated PU in MC was corrected to observed PU in Data
\end{itemize}
\end{block}
\end{column}

\begin{column}{.45\textwidth}
\vspace{-.9cm}
\begin{figure}[!Hhtbp]
  \begin{center}
    \includegraphics[width=1.0\textwidth]{../figs/Ana/Nvtcs.png}
  \end{center}
\end{figure}
\vspace{-.75cm}
\begin{block}{}
\tiny Number of vertices distribution for data and MC samples. The comparison has been performed after basic selection except number of b-tagged jets (the basic selection is described in the next slides). The gray band correspond to the statistical error of MC samples sum. Normalization of MC samples was done to the 19.7~fb$^{-1}$.
\end{block}
\end{column}

\end{columns}
\end{frame}

\begin{frame}{Basic selection}
\vspace{-.2cm}

\begin{block}{}
  \begin{itemize}\scriptsize
  \item Trigger L1: at least 4 central jets (\etal{3}) with \ptg{32}~or \ptg{36}~or \ptg{40} \\
                    or at least 2 central jets with \ptg{52}~or \ptg{56}~or \ptg{64} \\
                    or events with \HTg{125}~or \HTg{150}~or \HTg{175}
  \item HLT: 6 central jets with a \ptg{20}, 4 with a \ptg{60} and 2 with a \ptg{80}
  \item \textbf{First offline cut}: 2 jets with \ptg{90}, 2 jets with \ptg{70} and 2 jets with \ptg{30}
  \end{itemize}
\end{block}

\begin{columns}
\begin{column}{.50\textwidth}
\vspace{-.2cm}
\begin{block}{}
  \scriptsize \textbf{Second cut}: At least 5 jets with \ptg{30} and \etal{2.5} and at least one additional jet with \ptg{30} and \etal{5} were required
\end{block}

\vspace{-.2cm}
\begin{block}{}
\scriptsize \textbf{Figure}: Distribution of pseudorapidity of the accompanying jet produced with the \Tp. The distribution is taken from the signal MC sample with M=700 \GeVcc~and it is normalized to unity.
\end{block}
\end{column}

\begin{column}{.50\textwidth}
\vspace{-.2cm}
\begin{figure}[!Hhtbp]
  \begin{center}
    \includegraphics[width=1.0\textwidth]{../figs/Ana/SixthJetMCTruth.png}
    %\caption{Distribution of pseudorapidity of the accompanying jet produced with the \Tp. The distribution is taken from the signal MC sample with M=700 \GeVcc~and it is normalized to unity.}
    %\label{fig:SixthJetTp}
  \end{center}
\end{figure}
\end{column}
\end{columns}

\end{frame}

\begin{frame}{}
\vspace{-.2cm}

\begin{columns}
\begin{column}{.50\textwidth}
\begin{block}{}
\scriptsize \textbf{3rd cut}: the leading jet was required to have a \ptg{150}
\end{block}
\vspace{-.2cm}
\begin{block}{}
\scriptsize \textbf{4th cut}: \HTg{550} was required
\end{block}
\vspace{-.2cm}
\begin{block}{}
\scriptsize \textbf{Figure}: Distribution of the $H_{T}$ variable for data and the sum of the MC samples normalized to luminosity. The signal sample (M=700 \GeVcc) is over-imposed on top of the stack of the MC samples. The gray band represents the statistical uncertainties from the sum of the MC background. Reasonable agreement is observed, with the multijet process as the dominant process at this stage. Normalization of samples was done to luminosity.
\end{block}
\end{column}

\begin{column}{.50\textwidth}
\begin{figure}[!Hhtbp]
  \begin{center}
    \includegraphics[width=1.0\textwidth]{../figs/Ana/HT.png}
    %\caption{Distribution of the $H_{T}$ variable for data and the sum of the MC samples normalized to luminosity. The signal sample (M=700 \GeVcc) is over-imposed on top of the stack of the MC samples. The gray band represents the statistical uncertainties from the sum of the MC background. Reasonable agreement is observed, with the multijet process as the dominant process at this stage. Normalization of samples was done to luminosity.}
    %\label{fig:HT}
  \end{center}
\end{figure}
\end{column}

\end{columns}
\end{frame}

\begin{frame}{}
\vspace{-.2cm}

\begin{columns}
\begin{column}{.50\textwidth}
\begin{block}{}
\scriptsize \textbf{B-tagging}: CSV (Constrained Secondary Vertex) algorithm $\to$ Multivariate technique that give a discriminator indicating how likely a jet is coming from a b-quark
\textbf{Working points}: Loose, Medium and Tight\\\tiny{CSVL$\to$0.244 \\$\epsilon^{CSVL}_{b}=85$\%, $\epsilon^{CSVL}_{c}=45$\%, $\epsilon^{CSVL}_{l}=10$\% \\CSVM$\to$0.679\\ $\epsilon^{CSVM}_{b}=$70\%, $\epsilon^{CSVM}_{c}=$20\%, $\epsilon^{CSVM}_{l}=$1\% \\CSVT$\to$0.898\\ $\epsilon^{CSVT}_{b}=50$\%, $\epsilon^{CSVT}_{c}=7$\%, $\epsilon^{CSVT}_{l}=0.2$\%}
\end{block}
\vspace{-.2cm}
\begin{block}{}
\scriptsize \textbf{5th cut}: at least 3 CSVM b-tagged jets. Only jets with $|\eta|<=2.4$ considered for b-tagging.
\end{block}
\vspace{-.2cm}
\begin{block}{}
\scriptsize \textbf{Figure}: B-tagged CSVM jet multiplicity for data and MC samples before requiring at least 3 CSVM b-tagged jets. The sum of MC samples is normalized to the integrated luminosity.
\end{block}
\end{column}

\begin{column}{.50\textwidth}
\begin{figure}[!Hhtbp]
  \begin{center}
    \includegraphics[width=1.0\textwidth]{../figs/Ana/NCSVM.png}
    %\caption{B-tagged CSVM jet multiplicity for data and MC samples before requiring at least 3 CSVM b-tagged jets. The sum of MC samples is normalized to the integrated luminosity.}
    %\label{fig:Nb}
  \end{center}
\end{figure}
\end{column}

\end{columns}
\end{frame}



\begin{frame}{Corrections to MC for b-tagging}
\vspace{-.2cm}

\begin{columns}
\begin{column}{.50\textwidth}
\begin{block}{}
\scriptsize Scale factors derived from data/MC comparisons $\to$ $SF^{flavor}_{\eta}(p_{T})$\\
As mean values: $SF^{b\; or\; c}\sim 0.94$ and $SF^{light}\sim 1.06$
\end{block}
\vspace{-.2cm}
\begin{block}{}
\scriptsize In order to apply the SF's a weight per event is calculated\\
$w=\frac{P(\text{DATA})}{P(\text{MC})}$ with \\ \tiny{
$P(\text{MC}) = \prod_{i=\text{tagged}} \varepsilon_i \prod_{j=\text{not tagged}} (1-\varepsilon_j)$\\
$P(\text{DATA}) = \prod_{i=\text{tagged}} \text{SF}_i \varepsilon_i \prod_{j=\text{not tagged}} (1-\text{SF}_j \varepsilon_j)$ \\
}
\end{block}
\vspace{-.2cm}
\begin{block}{}
\tiny B-tagging efficiencies defined as\\
$\varepsilon_f(i,j) = \frac{N_f^\text{b-tagged}(i,j)}{N_f^\text{Total}(i,j)}$ where \\
where $ N_f^\text{Total}(i,j) $ and $ N_f^\text{b-tagged}(i,j) $ are the total number and the number of b-tagged jets, respectively, of flavor $ f $ in the $ (p_\text{T},\eta) $ bin $ (i,j) $ for a given MC sample.
\end{block}
\vspace{-.2cm}
\begin{block}{}
\scriptsize \textbf{Figure}: Distribution of the weights from b-tagging scale factors for all MC samples.
\end{block}
\end{column}

\begin{column}{.50\textwidth}
\begin{figure}[!Hhtbp]
  \begin{center}
    \includegraphics[width=1.0\textwidth]{../figs/Ana/SF_weight.png}
    %\caption{Distribution of the weights from b-tagging scale factors for all MC samples.}
    %\label{fig:SFweight}
  \end{center}
\end{figure}
\end{column}

\end{columns}
\end{frame}

\begin{frame}{\Tp~reconstruction with a $\chi^{2}$ sorting algorithm}
\vspace{-.2cm}
\scriptsize

\begin{itemize}
\item $\chi^{2}$ sorting algorithm used to identify the \Tp~decay products and to reconstruct the Higgs and top candidates
\item $\chi^{2}$ variable defined for each jets combination in an event
\item The combination that minimizes this variable gives the best fit of the objects under reconstruction
\end{itemize}

\begin{equation*}
\chi^{2}=\frac{(M_{H}-M_{bb})^{2}}{\sigma_{H}^{2}}+\frac{(M_{W}-M_{jj})^{2}}{\sigma_{W}^{2}}+\frac{(M_{t}-M_{bjj})^{2}}{\sigma_{t}^{2}}
%\label{eq:chi2def}
\end{equation*}

\begin{itemize}
\item $M_{H}=125$~\GeVcc, $M_{W}=84.06$~\GeVcc, $M_{t}=175.16$~\GeVcc, $\sigma_{H}=12.4$~\GeVcc, and $\sigma_{W}=10.12$~\GeVcc~and $\sigma_{t}=17.35$~\GeVcc. From similar MC studies.
\item For the Higgs reconstruction only CSVM b-tagged jets were considered
\item For the \W~reconstruction all jets with a \ptg{30} were considered
\item For the top reconstruction one b-tagged jet and the pair of jets used for the \W~were utilized
\end{itemize}

\end{frame}

\begin{frame}{}
\vspace{-.2cm}

\begin{columns}
\begin{column}{.50\textwidth}

\begin{figure}[!Hhtbp]
  \begin{center}
    \includegraphics[width=1.0\textwidth]{../figs/Ana/Exclusive_Efficiency_V8.png}
    %\includegraphics[width=0.46\textwidth]{figs/Ana/Inclusive_Efficiency_V8.png}
    %\caption{Reconstruction efficiency by the $\chi^{2}$ algorithm of the Higgs boson, \W~boson, top quark and \Tp, as the ratio of the number of events where the particle was correctly reconstructed to the number of events where jets could be matched to partons [left] and to the total number of events [right]}
    %\label{fig:RecEff}
  \end{center}
\end{figure}

\vspace{-.2cm}
\begin{block}{}
\scriptsize \textbf{Figure}: Reconstruction efficiency by the $\chi^{2}$ algorithm of the Higgs boson, \W~boson, top quark and \Tp, as the ratio of the number of events where the particle was correctly reconstructed to the number of events where jets could be matched to partons.
\end{block}
\end{column}

\begin{column}{.50\textwidth}

\begin{figure}[!Hhtbp]
  \begin{center}
    \includegraphics[width=1.0\textwidth]{../figs/Ana/HundresdsMassChi2Tp.png}
    %\includegraphics[width=0.45\textwidth]{figs/Ana/FiftiesMassChi2Tp.png}
    %\caption{Reconstructed \Tp~mass for all mass points from the $\chi^{2}$ sorting algorithm after basic selection. Each mass point is normalized to luminosity and its corresponding cross section. A gaussian fit of these distributions will be presented afterward in section~\ref{sec:finalsel}, accompanied with a discussion about the resolution on the reconstruction of the \Tp.}
    %\label{fig:RecT}
  \end{center}
\end{figure}

\vspace{-.2cm}
\begin{block}{}
\scriptsize \textbf{Figure}: Reconstructed \Tp~mass for some mass points from the $\chi^{2}$ sorting algorithm after basic selection. Each mass point is normalized to luminosity and its corresponding cross section.
\end{block}
\end{column}

\end{columns}
\end{frame}

\begin{frame}{Selection based on reconstructed objects}
\vspace{-.2cm}
\scriptsize

\begin{columns}
\begin{column}{.50\textwidth}
\vspace{-.2cm}
\begin{block}{}
\tiny
\begin{itemize}
\item Selection optimized via a multidimensional scan of variables 
\item Signal discrimination evaluated by $S/B$, using as signal the $M=700$\GeVcc~mass point, and as background the \ttbar~and QCD\_HT-500To1000
\item Selection has been adjusted to keep at least 10 signal events, for the 700~\GeVcc~mass point, after the full selection
\item \textbf{Data/MC plots are shown for illustration, but final background estimation is derived from data}
\end{itemize}
\end{block}
\vspace{-.2cm}
\begin{block}{}
\scriptsize \textbf{1st criterion}: $\chi^{2}<8$
\end{block}
\vspace{-.2cm}
\begin{block}{}
\scriptsize \textbf{Figure}: Distribution of the $\chi^{2}$ variable for data and MC samples. The signal sample used has a \Tp~mass of 700 \GeVcc. Backgrounds present higher values than the signal. The sum of MC is normalized to the integrated luminosity.
\end{block}
\end{column}

\begin{column}{.50\textwidth}
\begin{figure}[!Hhtbp]
  \begin{center}
    \includegraphics[width=1.0\textwidth]{../figs/Ana/chi2Nm1.png}
    %\caption{Distribution of the $\chi^{2}$ variable for data and MC samples. The signal sample used has a \Tp~mass of 700 \GeVcc. Backgrounds present higher values than the signal. The sum of MC is normalized to the integrated luminosity. }
    %\label{fig:chi2}
  \end{center}
\end{figure}
\end{column}

\end{columns}

\end{frame}

\begin{frame}{}
\vspace{-.2cm}
    \begin{block}{}\scriptsize
      \textbf{2nd criterion}: $\Delta R(bb)<1.2$\\
      \textbf{3rd criterion}: Higgs candidate mass between 105 and 145 \GeVcc
    \end{block}

\vspace{-.5cm}
\begin{columns}
\begin{column}{.50\textwidth}
\begin{figure}[!Hhtbp]
  \begin{center}
    \includegraphics[width=0.9\textwidth]{../figs/Ana/DRbbNm1.png}
  \end{center}
\end{figure}

\vspace{-.7cm}
    \begin{block}{}\tiny
      $\Delta R$ of the 2 b-tagged jets used to reconstruct the Higgs candidate after $\chi^{2}$ cut.
    \end{block}
\end{column}

\begin{column}{.50\textwidth}
\begin{figure}[!Hhtbp]
  \begin{center}
    \includegraphics[width=0.9\textwidth]{../figs/Ana/HMNm1.png}
  \end{center}
\end{figure}

\vspace{-.7cm}
    \begin{block}{}\tiny
      Distribution for $M(H_{cand})$ for data and the sum of Monte Carlo samples. All others selection criteria are applied up to this one.
    \end{block}
\end{column}

\end{columns}

\end{frame}


\begin{frame}{}
\vspace{-.2cm}
    \begin{block}{}\scriptsize
      \textbf{4th criterion}: $(M(top^{2nd})+M(W^{2nd}))/M(H)>6.8$\\
      \textbf{5th criterion}: $\Delta R (T' j^{6})>4.8$\\
      \textbf{6th criterion}: $H_{T}>0.67$
    \end{block}

\vspace{-.5cm}
\begin{columns}
\begin{column}{.50\textwidth}
\begin{figure}[!Hhtbp]
  \begin{center}
    \includegraphics[width=0.8\textwidth]{../figs/Ana/M2HPNm1.png}
  \end{center}
\end{figure}

\vspace{-.7cm}
    \begin{block}{}\tiny
      Distribution of $(M(top^{2nd})+M(W^{2nd}))/M(H)$ for data and the sum of the Monte Carlo samples. Selection criteria are applied up to Higgs mass cut. The low statistics in the multijet (QCD) MC sample is visible at this stage.
    \end{block}
\end{column}

\begin{column}{.50\textwidth}
\begin{figure}[!Hhtbp]
  \begin{center}
    \includegraphics[width=0.8\textwidth]{../figs/Ana/DRTp6JNm1.png}
  \end{center}
\end{figure}

\vspace{-.7cm}
    \begin{block}{}\tiny
      Distributions for $\Delta R (T' j^{6})$  for data and the sum of Monte Carlo samples. All others criteria are applied up to this one. The low statistics in the multijet (QCD) MC sample is visible at this stage.
    \end{block}
\end{column}

\end{columns}

\end{frame}

\begin{frame}{Selection optimization}
\vspace{-.2cm}

\begin{table}[htbH]
\begin{center}
\resizebox{\textwidth}{!}{
\begin{tabular}{|c|c|c|}
\hline 
Cut & $S/B$ & $S/\sqrt{S+B}$ \\
\hline
$\chi^{2}<8$ & $3.4\ex{-2} \pm 2.85\ex{-3}$  & $ 0.96 \pm 0.05$  \\
$\Delta R(bb)<1.2$ & $4.76\ex{-2} \pm 4.52\ex{-3}$ & $1.10 \pm 0.07$  \\
%$1.6<\Delta R (W_{cand} H_{cand})<4.0$ & $4.82\ex{-2} \pm 4.59\ex{-3}$  & $1.10 \pm 0.07$  \\
$105$ \GeVcc~$< M(H_{cand}) < 145$ \GeVcc~& $6.37\ex{-2} \pm 6.74\ex{-3}$  & $1.22 \pm 0.08$  \\
$(M(top^{2nd})+M(W^{2nd}))/M(H_{cand})>6.8$ & $0.15 \pm 0.03$  & $1.45 \pm 0.16$  \\
$\Delta R (T j^{6})>4.8$ & $0.42 \pm 0.19$  & $1.67 \pm 0.32$  \\
Relative $H_{T}>0.67$ & $1.16 \pm 0.17$  & $2.13 \pm 0.17$  \\
\hline
\end{tabular}
%\caption{$S/B$ and $S/\sqrt{S+B}$ from MC samples for each step of the selection after reconstruction of resonances with the $\chi^{2}$ sorting algorithm. Only $M=700$ \GeVcc~signal, \ttbar~and QCD\_HT-500To1000 MC samples were used. \label{tab:Estimators}}
}
\end{center}
\end{table}

\vspace{-.2cm}
    \begin{block}{}\scriptsize
      $S/B$ and $S/\sqrt{S+B}$ from MC samples for each step of the selection after reconstruction of resonances with the $\chi^{2}$ sorting algorithm. Only $M=700$ \GeVcc~signal, \ttbar~and QCD\_HT-500To1000 MC samples were used.
    \end{block}

\end{frame}


%%%%%%%%%%%%%%%%%%%%%%%%%%%%%%%%%%%%%%%%%%%%%%%%%%%%%%%%
%%%%%%%%%%%%%%%%%%%%%%%%%%%%%%%%%%%%%%%%%%%%%%%%%%%%%%%%
\section[Conclusion]{Conclusions, perspectives}
\subsection{Conclusions}
\begin{frame}{Conclusions}
\vspace{-.5cm}
\begin{block}{}
\begin{itemize}\scriptsize
  \item Feasibility study for a search of a \Tp~at LHC at 8 TeV
  %\begin{itemize}\scriptsize
  %  \item Design of an strategy for a search of a \Tp~in the single production mode in the full hadronic final state at 8 TeV LHC run.
    \begin{itemize}\tiny
      \item Les Houches 2013: Physics at TeV Colliders: New Physics Working Group Report \\arXiv:1405.1617
      \item Fully hadronic decays of a singly produced vectorlike top partner at the LHC \\Phys.Rev. D90 (2014) 11, 115008
    \end{itemize}
  %\end{itemize}
  \item MC simulations studies
    \begin{itemize}\scriptsize
      \item MadGraph
        \begin{itemize}\tiny
          \item Validation of releases to enter the CMS MC simulation central production
          \item Several scripts written and updated for collaboration usage
          \item Users support
        \end{itemize}
      \item Pythia 8
        \begin{itemize}\tiny
          \item Comparison of MadGraph + Pythia 8 simulation with other generators and data using RIVET
        \end{itemize}
      \item MadSpin        
        \begin{itemize}\tiny
          \item Inclusion of MadSpin for the gridpack generation of \ttbar~in CMS central production
        \end{itemize}
    \end{itemize}
  \item Search for a single produced \Tp~decaying into top and Higgs in the full hadronic final state
\end{itemize}
\end{block}
\end{frame}

\subsection{Perspectives}
\begin{frame}{Perspectives}
\vspace{-.3cm}
\begin{block}{}
\begin{itemize}\scriptsize
  \item Étude des désintégrations radiatives $Z^0\to\mu\mu\gamma$
  \begin{itemize}\scriptsize
    \item Échelle d'énergie pour 
    \item Finaliser l'étude des incertitudes systématiques
  \end{itemize}
  \item Recherche du boson de Higgs du Modèle Standard dans le canal $H\to\gamma\gamma$
  \begin{itemize}\scriptsize
    \item Réoptimisation de la classification utilisée
    \item Observation d'un excès dans le canal diphoton, excès observé dans d'autres canaux (et dans ATLAS) : découverte d'un nouveau boson, compatible avec un boson de Higgs du Modèle Standard
    \item Propriétés de l'excès (spin, couplages, ...) ? Découverte dans le canal $H\to\gamma\gamma$ seul ?
  \end{itemize}
  \item Production associée d'un boson de Higgs dans le canal $h\to\gamma\gamma$ dans le cadre d'un modèle de quarks vecteurs
  \begin{itemize}\scriptsize
    \item Étude prospective en simulation complète
    \item Étude sur les données récoltées à 
  \end{itemize}
\end{itemize}
\end{block}
\vspace{-.1cm}

\begin{block}{}
\begin{itemize}\scriptsize
\item L'analyse porte sur les données 2011 et 2012 et a été rendue publique le 4 juillet dernier. La quantité de données a doublé depuis... et le LHC continue à accumuler des données !
\item Longue période d'arrêt du LHC pour 2013-2014... avant reprise des collisions à 
% \item Caractérisation de la nouvelle particule
\end{itemize}
\end{block}
\end{frame}

\begin{frame}
\vspace{-.5cm}
\begin{center}
Observation of a new boson at a mass of 125 GeV\\with the CMS experiment at the LHC
\\\tiny
PLB 716 (2012) 30-61 - CMS-PAS-HIG-12-020 - arXiv:1207.7235
\end{center}
%\begin{center}
%\includegraphics[width=.4\textwidth]{plots/Higgs/HIG-12-028/fig16.pdf}
%\includegraphics[width=.4\textwidth]{plots/Higgs/HIG-12-028/fig18.pdf}
%\\
%\includegraphics[width=.4\textwidth]{plots/Higgs/HIG-12-028/fig17.pdf}
%\includegraphics[width=.4\textwidth]{plots/Higgs/HIG-12-028/fig19.pdf}
%\end{center}
\vspace{-.8cm}
\begin{center}
Merci de votre attention !
\end{center}
\end{frame}

%%%%%%%%%%%%%%%%%%%%%%%%%%%%%%%%%%%%%%%%%%%%%%%%%%%%%%%%
%%%%%%% BACKUP
%%%%%%%%%%%%%%%%%%%%%%%%%%%%%%%%%%%%%%%%%%%%%%%%%%%%%%%%
\appendix
\section{BACKUP}
\begin{frame}
\begin{center}
\LARGE
BACKUP
\end{center}
\end{frame}

\begin{frame}
\tableofcontents
\end{frame}

\subsection{SM and beyond}

\begin{frame}{\Tp~pair production}
\vspace{-.2cm}
\begin{figure}[!Hhtbp]
  \begin{center}
    \includegraphics[width=0.3\textwidth]{../figs/Gluon_fusion_T_pair.jpg}
    \includegraphics[width=0.3\textwidth]{../figs/Quarks_schannel_T_pair.jpg}
    \includegraphics[width=0.3\textwidth]{../figs/Gluon_tchannel_T_pair.jpg}
    %\caption{Feynman diagrams of \Tp~production in pairs.}
    %\label{fig:ProdDiagPair}
  \end{center}
\end{figure}

\vspace{-.2cm}
    \begin{block}{}
      \tiny \centering Feynman diagrams of \Tp~production in pairs.
    \end{block}

\end{frame}

\begin{frame}{\Tp~single production}
\vspace{-.2cm}
\begin{figure}[!Hhtbp]
  \begin{center}
    \includegraphics[width=0.45\textwidth]{../figs/Tchannel_T_single.jpg}
    \includegraphics[width=0.45\textwidth]{../figs/QuarkGluonFusion_SingleT.jpg}
    %\caption{Single \Tp~production Feynman diagrams.}
    %\label{fig:ProdDiagSingle}
  \end{center}
\end{figure}

\vspace{-.2cm}
    \begin{block}{}
      \tiny \centering Single \Tp~production Feynman diagrams.
    \end{block}

\end{frame}

\begin{frame}{\Tp~branching ratios}
\vspace{-.2cm}
\begin{figure}[!Hhtbp]
  \begin{center}
    \includegraphics[width=0.6\textwidth]{../figs/pheno_br_tp.png}
    %\caption{\Tp~branching ratios as a function of its mass~\cite{Cacciapaglia:2011fx}.}
    %\label{fig:TBRs}
  \end{center}
\end{figure}

\vspace{-.2cm}
    \begin{block}{}
      \tiny \centering \Tp~branching ratios as a function of its mass.
    \end{block}

\end{frame}

\begin{frame}{Top quark pair production}
\vspace{-.2cm}
\begin{figure}[!Hhtbp]
  \begin{center}
    \includegraphics[width=0.32\textwidth]{../figs/Gluon_fusion_top_pair.jpg}
    \includegraphics[width=0.32\textwidth]{../figs/Gluon_tchannel_top_pair.jpg}
    \includegraphics[width=0.32\textwidth]{../figs/Quarks_schannel_top_pair.jpg}
    %\caption{Top pair production processes Feynman diagrams for proton-proton collisions, via gluon fusion [left], gluon t-channel [middle] and quark-antiquark annihilation [right].}
    %\label{fig:PairProductionFD}
  \end{center}
\end{figure}

\vspace{-.2cm}
    \begin{block}{}
      \tiny \centering Top pair production processes Feynman diagrams for proton-proton collisions, via gluon fusion [left], gluon t-channel [middle] and quark-antiquark annihilation [right].
    \end{block}

\end{frame}

\begin{frame}{Top quark single production}
\vspace{-.2cm}
\begin{figure}[!Hhtbp]
  \begin{center}
    \includegraphics[width=0.32\textwidth]{../figs/Schannel_top_single.jpg}
    \includegraphics[width=0.32\textwidth]{../figs/Tchannel_top_single.jpg}
    \includegraphics[width=0.32\textwidth]{../figs/TWchannel_top_single.jpg}
    %\caption{Feynman diagrams of single top production processes of proton-proton collisions, from left to right s-channel, t-channel and associated \W~production.}
    %\label{fig:SingleProductionFD}
  \end{center}
\end{figure}

\vspace{-.2cm}
    \begin{block}{}
      \tiny \centering Feynman diagrams of single top production processes of proton-proton collisions, from left to right s-channel, t-channel and associated \W~production.
    \end{block}

\end{frame}

\begin{frame}{}
\vspace{-.2cm}
\begin{figure}[!Hhtbp]
  \begin{center}
    \includegraphics[width=0.9\textwidth]{../figs/singletop_allchanvsroots.png}
    %\caption{Single top production cross section as a function of the center of mass energy in proton-proton collisions compared to theoretical predictions for each production channel by ATLAS and CMS collaborations~\cite{TOPLHCWG}.}
    %\label{fig:SingleProduction}
  \end{center}
\end{figure}

\vspace{-.2cm}
    \begin{block}{}
      \tiny \centering Single top production cross section as a function of the center of mass energy in proton-proton collisions compared to theoretical predictions for each production channel by ATLAS and CMS collaborations.
    \end{block}

\end{frame}

\begin{frame}{Top Decay}
\vspace{-.2cm}
\begin{figure}[!Hhtbp]
  \begin{center}
    \includegraphics[width=0.4\textwidth]{../figs/Top_H_Decay.png}
    \includegraphics[width=0.4\textwidth]{../figs/Top_L_Decay.png}
    %\caption{Feynman diagrams for top decay channels with respective branching ratios.}
    %\label{fig:BRratiosandDecayChannels}
  \end{center}
\end{figure}

\vspace{-.2cm}
    \begin{block}{}
      \tiny \centering Feynman diagrams for top decay channels with respective branching ratios.
    \end{block}

\end{frame}

\begin{frame}{Top mass}
\vspace{-.2cm}
\begin{figure}[!Hhtbp]
  \begin{center}
    \includegraphics[width=0.6\textwidth]{../figs/LHC_topmass_May2015.png}
    %\caption{Top mass measurements from ATLAS and CMS collaborations and world combination including Tevatron results~\cite{TOPLHCWG}.}
    %\label{fig:TopMass}
  \end{center}
\end{figure}


\vspace{-.2cm}
    \begin{block}{}
      \tiny \centering Top mass measurements from ATLAS and CMS collaborations and world combination including Tevatron results.
    \end{block}

\end{frame}

\begin{frame}{Higgs production}
\vspace{-.2cm}
\begin{figure}[!Hhtbp]
  \begin{center}
    \includegraphics[width=0.35\textwidth, height=3.cm]{../figs/GluonFusion_H.png}
    \includegraphics[width=0.35\textwidth, height=3.cm]{../figs/Higgstrahlung.png}\\
    \includegraphics[scale=0.3]{../figs/VBF_H.png}
    \includegraphics[scale=0.3]{../figs/QuarkF_H.png}
    %\caption{Higgs boson production Feynman diagrams for proton-proton collisions: gluon fusion [left-up], Higgsstrahlung [right-up], vector boson fusion [left-down] and quark fusion [right-down].}
    %\label{fig:HiggsProd}
  \end{center}
\end{figure}

\vspace{-.2cm}
    \begin{block}{}
      \tiny \centering Higgs boson production Feynman diagrams for proton-proton collisions: gluon fusion [left-up], Higgsstrahlung [right-up], vector boson fusion [left-down] and quark fusion [right-down].
    \end{block}

\end{frame}

\begin{frame}{}
\vspace{-.2cm}
\begin{figure}[!Hhtbp]
  \begin{center}
    \includegraphics[width=0.6\textwidth]{../figs/7-14_Higgs_xsec.jpg}
    %\caption{Higgs boson theoretical production cross section as a function of center of mass energy, for a Higgs boson mass of 125 GeV/$c^{2}$~\cite{HIGGSXSWG}.}
    %\label{fig:HiggsProdXS}
  \end{center}
\end{figure}

\vspace{-.2cm}
    \begin{block}{}
      \tiny \centering Higgs boson theoretical production cross section as a function of center of mass energy, for a Higgs boson mass of 125 GeV/$c^{2}$.
    \end{block}

\end{frame}

\begin{frame}{Higgs decay}
\vspace{-.2cm}
\begin{figure}[!Hhtbp]
  \begin{center}
    \includegraphics[width=0.6\textwidth]{../figs/Higgs_BR_120-130.jpg}
    %\caption{Higgs boson decay branching ratios as a function of its mass~\cite{Dittmaier:2011ti, Dittmaier:2012vm, Heinemeyer:2013tqa, HIGGSXSWG}.}
    %\label{fig:HiggsBrs}
  \end{center}
\end{figure}

\vspace{-.2cm}
    \begin{block}{}
      \tiny \centering Higgs boson decay branching ratios as a function of its mass.
    \end{block}

\end{frame}

\begin{frame}{}
\vspace{-.2cm}
\begin{figure}[!Hhtbp]
  \begin{center}
    \includegraphics[width=0.3\textwidth]{../figs/BB_H.png}
    \includegraphics[width=0.3\textwidth]{../figs/Diphoton_H.png}
    \includegraphics[width=0.3\textwidth]{../figs/Golden_H.png}
    %\caption{Feynman diagrams of Higgs boson decay: $b\bar{b}$ [left], diphoton [center] and golden channels [right].}
    %\label{fig:HiggsDecays}
  \end{center}
\end{figure}

\vspace{-.2cm}
    \begin{block}{}
      \tiny \centering Feynman diagrams of Higgs boson decay: $b\bar{b}$ [left], diphoton [center] and golden channels [right].
    \end{block}

\end{frame}

\begin{frame}{Higgs mass}
\vspace{-.2cm}
\begin{figure}[!Hhtbp]
  \begin{center}
    \includegraphics[trim=10cm 7cm 1cm 1cm, clip=true, width=0.8\textwidth]{../figs/LHC_combined_obs_unblind_summary_a1_final.png}
    %\caption{ATLAS and CMS combination of Higgs boson mass measurement [top] and $\sigma/\sigma_{SM}$ (measured cross section over theoretical SM cross section) for searches performed by ATLAS and CMS in different Higgs boson decay channels [bottom]~\cite{Aad:2015zhl,CMS:2014ega,ATLAS-CONF-2015-007}.}
    %\label{fig:HiggsMass}
  \end{center}
\end{figure}

\vspace{-.2cm}
    \begin{block}{}
      \tiny \centering ATLAS and CMS combination of Higgs boson mass measurement.
    \end{block}

\end{frame}

\begin{frame}{}
\vspace{-.2cm}
\begin{figure}[!Hhtbp]
  \begin{center}
    \includegraphics[width=0.5\textwidth]{../figs/sqr_mlz_ccc_mH125.png}
    \includegraphics[width=0.4\textwidth]{../figs/ATLAS_HIGGS_mu_Summary.png}
    %\caption{ATLAS and CMS combination of Higgs boson mass measurement [top] and $\sigma/\sigma_{SM}$ (measured cross section over theoretical SM cross section) for searches performed by ATLAS and CMS in different Higgs boson decay channels [bottom]~\cite{Aad:2015zhl,CMS:2014ega,ATLAS-CONF-2015-007}.}
    %\label{fig:HiggsMass}
  \end{center}
\end{figure}

\vspace{-.2cm}
    \begin{block}{}
      \tiny \centering $\sigma/\sigma_{SM}$ (measured cross section over theoretical SM cross section) for searches performed by ATLAS and CMS in different Higgs boson decay channels.
    \end{block}

\end{frame}

\begin{frame}{Higgs width}
\vspace{-.2cm}
\begin{figure}[!Hhtbp]
  \begin{center}
    \includegraphics[width=0.42\textwidth]{../figs/u0g5o.png}
    \includegraphics[width=0.42\textwidth]{../figs/AllFitPaper_30_04_14_MeV.png}
    %\caption{Higgs boson width as a function of its mass~\cite{Dittmaier:2011ti, Dittmaier:2012vm, Heinemeyer:2013tqa, HIGGSXSWG} [left] and current limits from CMS measurement~\cite{Khachatryan:2014iha} [right].}
    %\label{fig:WidthHiggs}
  \end{center}
\end{figure}

\vspace{-.2cm}
    \begin{block}{}
      \tiny \centering Higgs boson width as a function of its mass [left] and current limits from CMS measurement [right].
    \end{block}

\end{frame}

%\begin{frame}{}
%\vspace{-.2cm}
%
%\vspace{-.2cm}
%    \begin{block}{}
%      \tiny \centering 
%    \end{block}
%
%\end{frame}

\include{CMSBU}

\subsection{Feasibility study}

\begin{frame}{Jets \pt}
\vspace{-.4cm}
\begin{figure}[!Hhtbp]
  \begin{center}
    \includegraphics[width=0.55\textwidth]{../figs/Pheno/JetPt.png}
    %\caption{$p_{T}$  of the six leading jets for backgrounds (stacked) and signal (over--imposed) normalized to 20 $fb^{-1}$ luminosity. QCD background is on top of the stack of backgrounds.}
    %\label{fig:Var1}
  \end{center}
\end{figure}
\vspace{-.4cm}
    \begin{block}{}
      \tiny \centering $p_{T}$  of the six leading jets for backgrounds (stacked) and signal (over--imposed) normalized to 20 $fb^{-1}$ luminosity. QCD background is on top of the stack of backgrounds.
    \end{block}

\end{frame}

\begin{frame}{B-jet multiplicity requirement}
\vspace{-.2cm}
\begin{figure}[!Hhtbp]
  \begin{center}
    \includegraphics[width=0.45\textwidth]{../figs/Pheno/Nb.png}
    %\caption{B-tagged jet multiplicity for backgrounds (stacked) and signal (over--imposed) normalized to 20~$fb^{-1}$ luminosity. The signal has as mean value 3 b-tagged jets.}
    %\label{fig:Nbs}
  \end{center}
\end{figure}

\vspace{-.2cm}
    \begin{block}{}
      \tiny B-tagged jet multiplicity for backgrounds (stacked) and signal (over--imposed) normalized to 20~$fb^{-1}$ luminosity. The signal has as mean value 3 b-tagged jets. The following method was used to emulate the performance of b-jet identification algorithms:
  \begin{enumerate}\tiny
  \item The CMS results for CSVM were used: $\epsilon^{b-tag}_{b}=0.9$, $\epsilon^{b-tag}_{c}=0.6$ and $\epsilon^{b-tag}_{l}=0.1$.
  \item Throw a random number $r$ between 0 and 1 for each event.
  \item Loop over all the jets from an event and, depending on their flavor and the random number from last step, declare each jet to be or not to be b-tagged. A jet is b-tagged if: it is coming from a b-quark and $r\leq\epsilon^{b-tag}_{b}$, or it is coming from a c-quark and $r\leq\epsilon^{b-tag}_{c}$, or it is coming from a light-quark and $r\leq\epsilon^{b-tag}_{l}$.
  \end{enumerate}
    \end{block}

\end{frame}

\begin{frame}{}
\vspace{-.2cm}

\vspace{-.2cm}
    \begin{block}{}
      \tiny \centering 
    \end{block}

\end{frame}

\begin{frame}{}
\vspace{-.2cm}

\vspace{-.2cm}
    \begin{block}{}
      \tiny \centering 
    \end{block}

\end{frame}

\begin{frame}{}
\vspace{-.2cm}

\vspace{-.2cm}
    \begin{block}{}
      \tiny \centering 
    \end{block}

\end{frame}

\begin{frame}{}
\vspace{-.2cm}

\vspace{-.2cm}
    \begin{block}{}
      \tiny \centering 
    \end{block}

\end{frame}

\begin{frame}{}
\vspace{-.2cm}

\vspace{-.2cm}
    \begin{block}{}
      \tiny \centering 
    \end{block}

\end{frame}

\include{MCBU}

\subsection{Analysis}

\begin{frame}{Datasets - Backgrounds}
\vspace{-.2cm}

%\begin{table*}[htbH]
\begin{center}
\resizebox{\textwidth}{!}{
\begin{tabular}{|c|c|c|}
\hline 
Samples & Cross-Section (pb) & Number of events\\
\hline
QCD\_Pt-120to170\_TuneZ2star\_8TeV\_pythia6 & 16\(\times 10^4\) & 5.9M\\
QCD\_Pt-170to300\_TuneZ2star\_8TeV\_pythia6 & 34\(\times 10^3\) & 5.8M\\
QCD\_Pt-300to470\_TuneZ2star\_8TeV\_pythia6 & 18\(\times 10^2\) & 5.9M\\ 
QCD\_Pt-470to600\_TuneZ2star\_8TeV\_pythia6 & 114 & 3.9M\\
QCD\_Pt-600to800\_TuneZ2star\_8TeV\_pythia6 & 27 & 3.9M\\
QCD\_Pt-800to1000\_TuneZ2star\_8TeV\_pythia6 & 3.5 & 3.9M\\
QCD\_HT-500To1000\_TuneZ2star\_8TeV-madgraph-pythia6 & 84\(\times 10^2\) & 30M\\ 
QCD\_HT-1000ToInf\_TuneZ2star\_8TeV-madgraph-pythia6 & 2\(\times 10^2\) & 14M\\ 
DYToCC\_M\_50\_TuneZ2star\_8TeV\_pythia6 & 31\(\times 10^2\) & 2M\\
DYToBB\_M\_50\_TuneZ2star\_8TeV\_pythia6 & 38\(\times 10^2\) & 2M\\
TTJets\_MSDecays\_central\_TuneZ2star\_8TeV-madgraph-tauola & 247.7 [NNLO] & 62M\\
%TT\_CT10\_TuneZ2star\_8TeV-powheg-tauola & 247.7 [NNLO] & 22M\\
T\_tW-channel-DR\_TuneZ2star\_8TeV-powheg-tauola & 11.1 [NNLO] &497k\\
T\_s-channel\_TuneZ2star\_8TeV-powheg-tauola & 3.79 [NNLO] & 260k\\
T\_t-channel\_TuneZ2star\_8TeV-powheg-tauola & 54.9 [NNLO] & 3.7M\\
Tbar\_tW-channel-DR\_TuneZ2star\_8TeV-powheg-tauola & 11.1 [NNLO] & 493k\\
Tbar\_s-channel\_TuneZ2star\_8TeV-powheg-tauola & 1.76 [NNLO] & 140k\\
Tbar\_t-channel\_TuneZ2star\_8TeV-powheg-tauola & 29.7 [NNLO] & 1.9M\\
WZ\_TuneZ2star\_8TeV\_pythia6\_tauola & 33.6 [NLO] & 10M\\
ZZ\_TuneZ2star\_8TeV\_pythia6\_tauola & 7.6 [NLO] & 9.8M\\
WW\_TuneZ2star\_8TeV\_pythia6\_tauola & 56 [NLO] & 10M\\
TTH\_Inclusive\_M-125\_8TeV\_pythia6 & 0.13 [NLO] & 100K\\
\hline
\end{tabular}
}
%\caption{List of Monte-Carlo background samples used in the analysis, their corresponding cross-section and their number of events.\label{tab:MCbkg}}
\end{center}
%\end{table*}

\end{frame}

\begin{frame}{Datasets - Signal}
\vspace{-.2cm}

%\begin{table*}[htbH]
\begin{center}
\resizebox{\textwidth}{!}{
\begin{tabular}{|c|c|c|c|}
\hline 
Sample & \Tp~Mass & Cross-Section & Number of events\\
            & (GeV$/c^{2}$) &  (fb) & \\
\hline
TprimeJetToTH\_M-600\_TuneZ2star\_8TeV-madgraph\_tauola & 600 & 215.4 & 95K \\
TprimeJetToTH\_M-650\_TuneZ2star\_8TeV-madgraph\_tauola & 650 & 177.8 & 99K \\
TprimeJetToTH\_M-700\_TuneZ2star\_8TeV-madgraph\_tauola & 700 & 143.7 & 99K \\
TprimeJetToTH\_M-750\_TuneZ2star\_8TeV-madgraph\_tauola & 750 & 118.6 & 99K \\
TprimeJetToTH\_M-800\_TuneZ2star\_8TeV-madgraph\_tauola & 800 & 100 & 96K \\
TprimeJetToTH\_M-850\_TuneZ2star\_8TeV-madgraph\_tauola & 850 & 84.3 & 99K \\
TprimeJetToTH\_M-900\_TuneZ2star\_8TeV-madgraph\_tauola & 900 & 72.6 & 99K \\
TprimeJetToTH\_M-950\_TuneZ2star\_8TeV-madgraph\_tauola & 950 & 62.6 & 96K \\
TprimeJetToTH\_M-1000\_TuneZ2star\_8TeV-madgraph\_tauola & 1000 & 53.9 & 99K \\
\hline
\end{tabular}
}
%\caption{List of Monte-Carlo signal samples used in the analysis, their corresponding cross-section and \Tp~mass.\label{tab:MCsig}}
\end{center}
%\end{table*}

\end{frame}

\begin{frame}{PU corrections}
\vspace{-.2cm}

\begin{figure}[!Hhtbp]
  \begin{center}
    \includegraphics[width=0.49\textwidth]{../figs/Ana/DataPU40.png}
    \includegraphics[width=0.49\textwidth]{../figs/Ana/MCPU40.png}\\
    \includegraphics[width=0.5\textwidth]{../figs/Ana/WeightPU40.png}
    \begin{block}{}
      \tiny \centering Pileup for data [up-left], MC S10 [up-right] and ratio between them [bottom].
    \end{block}
    %\caption{Pileup for data [up-left], MC S10 [up-right] and ratio between them [bottom].}
    %\label{fig:PU_distros}
  \end{center}
\end{figure}

\end{frame}

\begin{frame}{Six leading jets \pt}
\vspace{-.2cm}
\begin{figure}[!Hhtbp]
  \begin{center}
    \includegraphics[width=0.35\textwidth, height=0.65\textheight]{../figs/Ana/jet1pt.png}
    \includegraphics[width=0.35\textwidth, height=0.65\textheight]{../figs/Ana/jet2pt.png}
    \includegraphics[width=0.35\textwidth, height=0.65\textheight]{../figs/Ana/jet3pt.png}
    %\caption{Distribution of transverse momentum of the 6 leading jets. The gray band represents the statistical uncertainties from the sum of the MC background. Reasonable agreement is observed, with the multijet process as the dominant process at this stage. Normalization of samples was done to the 19.7~fb$^{-1}$.}
    %\label{fig:6jpt}
  \end{center}
\end{figure}

\vspace{-.2cm}
    \begin{block}{}
      \tiny \centering Distribution of transverse momentum of the 3 leading jets. The gray band represents the statistical uncertainties from the sum of the MC background. Reasonable agreement is observed, with the multijet process as the dominant process at this stage. Normalization of samples was done to the 19.7~fb$^{-1}$.
    \end{block}

\end{frame}

\begin{frame}{}
\vspace{-.2cm}
\begin{figure}[!Hhtbp]
  \begin{center}
    \includegraphics[width=0.35\textwidth, height=0.65\textheight]{../figs/Ana/jet4pt.png}
    \includegraphics[width=0.35\textwidth, height=0.65\textheight]{../figs/Ana/jet5pt.png}
    \includegraphics[width=0.35\textwidth, height=0.65\textheight]{../figs/Ana/jet6pt.png}
    %\caption{Distribution of transverse momentum of the 6 leading jets. The gray band represents the statistical uncertainties from the sum of the MC background. Reasonable agreement is observed, with the multijet process as the dominant process at this stage. Normalization of samples was done to the 19.7~fb$^{-1}$.}
    %\label{fig:6jpt}
  \end{center}
\end{figure}

\vspace{-.2cm}
    \begin{block}{}
      \tiny \centering Distribution of transverse momentum of the 4th, 5th and 6th leading jets. The gray band represents the statistical uncertainties from the sum of the MC background. Reasonable agreement is observed, with the multijet process as the dominant process at this stage. Normalization of samples was done to the 19.7~fb$^{-1}$.
    \end{block}

\end{frame}

\begin{frame}{Six leading jets $\eta$}
\vspace{-.2cm}
\begin{figure}[!Hhtbp]
  \begin{center}
    \includegraphics[width=0.35\textwidth, height=0.65\textheight]{../figs/Ana/jet1eta.png}
    \includegraphics[width=0.35\textwidth, height=0.65\textheight]{../figs/Ana/jet2eta.png}
    \includegraphics[width=0.35\textwidth, height=0.65\textheight]{../figs/Ana/jet3eta.png}
    %\caption{Distribution of $\eta$ of the 6 leading jets. The gray band represents the statistical uncertainties from the sum of the MC background. Reasonable agreement is observed, with the multijet process as the dominant process at this stage. Normalization of samples was done to luminosity.}
    %\label{fig:6jeta}
  \end{center}
\end{figure}

\vspace{-.2cm}
    \begin{block}{}
      \tiny \centering Distribution of $\eta$ of the 3 leading jets. The gray band represents the statistical uncertainties from the sum of the MC background. Reasonable agreement is observed, with the multijet process as the dominant process at this stage. Normalization of samples was done to luminosity.
    \end{block}

\end{frame}

\begin{frame}{}
\vspace{-.2cm}
\begin{figure}[!Hhtbp]
  \begin{center}
    \includegraphics[width=0.35\textwidth, height=0.65\textheight]{../figs/Ana/jet4eta.png}
    \includegraphics[width=0.35\textwidth, height=0.65\textheight]{../figs/Ana/jet5eta.png}
    \includegraphics[width=0.35\textwidth, height=0.65\textheight]{../figs/Ana/jet6eta.png}
    %\caption{Distribution of $\eta$ of the 6 leading jets. The gray band represents the statistical uncertainties from the sum of the MC background. Reasonable agreement is observed, with the multijet process as the dominant process at this stage. Normalization of samples was done to luminosity.}
    %\label{fig:6jeta}
  \end{center}
\end{figure}

\vspace{-.2cm}
    \begin{block}{}
      \tiny \centering Distribution of $\eta$ of the 4th, 5th and 6th leading jets. The gray band represents the statistical uncertainties from the sum of the MC background. Reasonable agreement is observed, with the multijet process as the dominant process at this stage. Normalization of samples was done to luminosity.
    \end{block}

\end{frame}

\end{document}
