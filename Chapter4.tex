\chapter[MC event generation]{Understanding theory predictions via Monte-Carlo event generation}
\label{chap:MC}

However we have nowadays a very elegant and complete theoretical description of particle physics, is not always evident how to translate this theory in actual predictions, in actual measurements. Moreover, on the case of hadronic colliders, as the LHC, it's even more difficult due to the particularities of strong interaction. On this subject, a set of tools and approaches have been developed in order to be able to make accurate predictions from theory that could be directly researched for on the experiments, as CMS or ATLAS for example. In the present chapter, we describe such tools and formalisms and a set of studies comparing the well behavior of this tools with data. 

\section{Mote-Carlo formalism}
\label{sec:MC}

\subsection{Partonic step}
\label{sec:parton}

\subsection{Hadronic step}
\label{sec:hadron}

\subsection{Detector step}
\label{sec:detector}


\section{Tools}
\label{sec:tools}

\subsection{Matrix-element generators}
\label{sec:ME}

\subsection{Hadron generators}
\label{sec:Had}

\subsection{Detector simulation}
\label{sec:det}

\section{Validation on data}
\label{sec:val}

