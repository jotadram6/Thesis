\chapter[VLQ models]{Vector Like Quarks: Generic model}
\label{chap:VLQ}

From chapter~\ref{chap:SM} we have seen how there are some parts in the SM that does not work very well. From such internal issues some further models/theories have been developed. All this theories are commonly grouped under the term Beyond Standard Model or simply BSM. One of the most famous BSM theory is supersymmetry (SUSY). This theory postulates a symmetry that does not distinguish between fermions and bosons. This idea have given birth to a plethora of model realizations and physics predictions. So far, nothing of the new consequences of this theory have been confirmed but the experiments have an enormous investment on their search. But not only SUSY have seen the day light, there is on the market an astonishing amount of BSM theories addressing different issues of the SM. Extra dimensions, fourth families, composite Higgs are a few of them.

In this chapter we will describe a bunch of models that introduce additional heavy quarks, heavier than the top, in order to solve the hierarchy problem, described on section~\ref{sec:hier}. 

\section{Motivation}
\label{sec:motiv}

\section{Generic Formulation}
\label{sec:form}

\section{Fesability study for a search of a $T$ at LHC at 8 TeV}
\label{sec:pheno}

\subsection{Production modes}
\label{sec:prod}

\subsection{Decay modes}
\label{sec:decay}

\subsection{Stragey for the full hadronic final state}
\label{sec:Pstra}

\subsection{Event selection}
\label{sec:Psel}

\subsection{Results}
\label{sec:Pres}

