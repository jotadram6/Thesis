\chapter[VLQ models]{Vector Like Quarks: Generic model}
\label{chap:VLQ}

From chapter~\ref{chap:SM} we have seen how there are some parts in the SM that does not work very well. From such internal issues some further models/theories have been developed. All this theories are commonly grouped under the term Beyond Standard Model or simply BSM. One of the most famous BSM theory is supersymmetry (SUSY). This theory postulates a symmetry that does not distinguish between fermions and bosons. This idea have given birth to a plethora of model realizations and physics predictions. So far, nothing of the new consequences of this theory have been confirmed but the experiments have an enormous investment on their search. But not only SUSY have seen the day light, there is on the market an astonishing amount of BSM theories addressing different issues of the SM. Extra dimensions, fourth families, composite Higgs are a few of them.

In this chapter we will describe a bunch of models that introduce additional heavy quarks, heavier than the top, in order to solve the hierarchy problem, described on section~\ref{sec:hier}. 

\section{Motivation}
\label{sec:motiv}

Why to introduce extra quarks.

Plausible solution of hierarchy problem.

Reference to models with extra quarks. (?)

\section{Generic Formulation}
\label{sec:form}

Formalism: Generic Langranian. Description of mixings with SM quarks.

\begin{table}[htbH]
\label{tab:VLQRepre}
\begin{center}
\begin{tabular}{|c|c|c|}
%\hline 
xxxxxxx & xxxxxxx & xxxxxxx
%\hline
\end{tabular}
\caption{Possible VLQ representations and correponding $SU(2)_{L}\times U(1)$ charges.}
\end{center}
\end{table}
%\begin{TOINCLUDE}Table with different possible representations and assigned charges\end{TOINCLUDE}

\subsection{Production modes}
\label{sec:prod}

Description of pair and single production. Parallel to production modes of top. Comparison between cross sections for pair and single production.

\begin{figure}[!Hhtbp]
  \begin{center}
    \includegraphics[width=0.3\textwidth]{figs/CMSlogo.png}
    \caption{Feynman diagrams of $T$ production in pairs [left] and single [right]}
    \label{fig:ProdDiag}
  \end{center}
\end{figure}

\begin{figure}[!Hhtbp]
  \begin{center}
    \includegraphics[width=0.3\textwidth]{figs/CMSlogo.png}
    \caption{$T$ production cross section for pair and single case as function of $T$ mass for different center of mass energy in proton-proton collisions.}
    \label{fig:TProdXS}
  \end{center}
\end{figure}
%\begin{TOINCLUDE}Plot of production cross section at LHC for different energies as function of the mass. Feynman diagrams of the production processes.\end{TOINCLUDE}

\subsection{Decay modes}
\label{sec:decay}

Description of possible decay channels of $T$. Relative importance of decay channels depending on the mass. 

\begin{figure}[!Hhtbp]
  \begin{center}
    \includegraphics[width=0.3\textwidth]{figs/CMSlogo.png}
    \caption{$T$ branching ratios as function of its mass.}
    \label{fig:TBRs}
  \end{center}
\end{figure}
%\begin{TOINCLUDE}Plot of branching ratio as function of the mass for different scenarios (representations)\end{TOINCLUDE}

\section{Fesability study for a search of a $T$ at LHC at 8 TeV}
\label{sec:pheno}

Discussion on selection of full hadronic channel.

\subsection{Stragey for the full hadronic final state}
\label{sec:Pstra}

General discussion of strategy used for selection. 

\begin{table}[htbH]
\label{tab:BKGS}
\begin{center}
\begin{tabular}{|c|c|c|}
%\hline 
xxxxxxx & xxxxxxx & xxxxxxx
%\hline
\end{tabular}
\caption{Cross sections and expected number of events for background processes and signal.}
\end{center}
\end{table}

\begin{figure}[!Hhtbp]
  \begin{center}
    \includegraphics[width=0.3\textwidth]{figs/CMSlogo.png}
    \caption{Forward jet produced in association with T.}
    \label{fig:ForwJ}
  \end{center}
\end{figure}
%\begin{TOINCLUDE}Table with cross sections and expected events for each background process and signal. Plot of eta of jet produced with T\end{TOINCLUDE}

\subsection{Event selection}
\label{sec:Psel}

Description of selection. Paragraph per variable.

\begin{figure}[!Hhtbp]
  \begin{center}
    \includegraphics[width=0.3\textwidth]{figs/CMSlogo.png}
    \caption{$p_{T}$  of the six leading jets for backgrounds (stacked) and signal (over--imposed)}
    \label{fig:Var1}
  \end{center}
\end{figure}

\begin{figure}[!Hhtbp]
  \begin{center}
    \includegraphics[width=0.3\textwidth]{figs/CMSlogo.png}
    \caption{Total hadronic energy for backgrounds (stacked) and signal (over--imposed)}
    \label{fig:Var2}
  \end{center}
\end{figure}

\begin{figure}[!Hhtbp]
  \begin{center}
    \includegraphics[width=0.3\textwidth]{figs/CMSlogo.png}
    \caption{$\Delta R$ between the reconstructed Higgs and W for backgrounds (stacked) and signal (over--imposed)}
    \label{fig:Var3}
  \end{center}
\end{figure}

\begin{figure}[!Hhtbp]
  \begin{center}
    \includegraphics[width=0.3\textwidth]{figs/CMSlogo.png}
    \caption{Mass of the reconstructed Higgs for backgrounds (stacked) and signal (over--imposed)}
    \label{fig:Var4}
  \end{center}
\end{figure}

\begin{figure}[!Hhtbp]
  \begin{center}
    \includegraphics[width=0.3\textwidth]{figs/CMSlogo.png}
    \caption{Relative total hadronic energy for backgrounds (stacked) and signal (over--imposed)}
    \label{fig:Var5}
  \end{center}
\end{figure}

\begin{table}[htbH]
\label{tab:SelEff}
\begin{center}
\begin{tabular}{|c|c|c|}
%\hline 
xxxxxxx & xxxxxxx & xxxxxxx
%\hline
\end{tabular}
\caption{Number of events for signal and backgrounds after the first selection cut (Cut 0), and efficiencies of each stage of the cutting procedure. The errors indicated are statistical only, based on the number of events.}
\end{center}
\end{table}
%\begin{TOINCLUDE}$N-1$ plots for selection variables. Efficiency table.\end{TOINCLUDE}

\subsection{Results}
\label{sec:Pres}

\begin{table}[tb]
\begin{center}
\begin{tabular}{l|c|c|c|c}
 & \multicolumn{2}{c|}{unweighted events}  & \multirow{2}{*}{weight} & weighted  \\
 & after Cut 10 & in mass window & & events \\
 \hline
 Signal & $8601$ & $3780$ & $0.03$ &$113 \pm 2$ \\
 \hline
   $t \bar{t}$ & $409$ & $57$ & $7.7$ & $437 \pm 58$ \\
 $W$+jets & $24$ & $3$ & $132$ & $395 \pm 228$ \\
 $QCD$ & $235$ & $34$ & $6.48$ & $220 \pm 38$ \\
 $tW$ & $18$ & $3$ & $11.3$ & $34 \pm 20$ \\
 $t$+ jet & $75$ & $7$ & $3.55$ & $25 \pm 9$ \\
  \hline
  total background & & & & $1112 \pm 352$ \\
\end{tabular}
\caption{Number of signal and background events from our simulation: in the first column the simulated events that pass all kinematic cuts, in the second column the events that fall in the mass window $710 < M_{jjjjj} < 750$~GeV, finally in the fourth column the number of weighted events in the mass window normalized to the physical cross section (the applied weight is listed in the third column). All the errors are statistical only. For the total background, we conservatively consider linear sum of errors.} \label{tab:events} \end{center}
\end{table}

\begin{figure}[!Hhtbp]
  \begin{center}
    \includegraphics[width=0.3\textwidth]{figs/CMSlogo.png}
    \caption{Reconstructed T mass after all cuts for backgrounds and signal (stacked)}
    \label{fig:M5J}
  \end{center}
\end{figure}
%\begin{TOINCLUDE}Expected yields table. Final M5J plot.\end{TOINCLUDE}