\chapter[The CMS experiment at LHC]{The CMS experiment at LHC}

The CMS experiment is one of the biggest particle physics experiments on the world. It is located at the ring of the LHC that is the main experience managed by CERN, the European Organization for Nuclear Research or Centre Europ\'{e}enne pour la Recherche Nucl\'{e}aire by its name on french. This center constitutes the biggest center for research on particle physics all over the world. All along its 60 years of existence, from 1954, 21 member states have been joining it, but an overall of 113 countries participate in different ways on this center. 

On the present chapter we discuss in detail different aspects of the LHC accelerator and the CMS experiment. In particular we make some emphasis in the CMS sub-detectors related to jets, objects that play the main role on the search that is the main subject of the present work. We also discuss the present state of both machines, their achievements and the challenges that were came through. Finally, also the expectations and goals for the upcoming run II are mentioned.  

\section{The Large Hadron Collider}
\label{sec:LHC}

The Large Hadron Collider, or LHC~\cite{Bruning:782076}, is a machine that accelerates and collides protons and heavy ions. This machine is the biggest particle collider nowadays with a circumference of 27 km. It also achieves the greatest energy by a collider up to present, planned to be 14 TeV at the center of mass of the collision. On the first run of the machine only 8 TeV were achieved, and next run is planned to start with 13 TeV. It's located in French-Swiss border near to Geneva. The tunnel for the machine was carved around 100 m under the ground, 45 m under the Jura mountains and 170 m under the L\'{e}man lake with an inclination of around 1.4\%, sloping down towards the lake . This machine has used as much as possible old LEP buildings and sites, that was an electron-positron collider built between 1984 and 1989. 

The protons and heavy ions accelerated by the machine are collided in different points where dedicated experiments are located to detect and study the product from the collisions. The four main experiments located on the LHC ring are CMS~\cite{Bayatian:922757,Bayatian:942733}, ATLAS~\cite{ATLAS:1999}, LHCb~\cite{Alves:2008zz} and ALICE~\cite{Cortese:879894}. The first two are experiments of generic purpose where searches for new physics and also precision measurements are performed. LHCb is dedicated to the physics of the b-quark, and ALICE focuses on the study of the quark-gluon plasma produced from heavy ions collisions. However one of the principal objectives of the construction of the LHC was the discovery of the Higgs boson, generic searches on new physics have been conducted from the very beginning of the first data taking on 2009. Moreover, after the Higgs discovery on 2012 there is a growing effort on the searches for new physics and precision measurement on the properties of the Higgs.

The LHC is a complex machine composed of several parts. The two principal parts are the injector chain and the main ring. A diagram of the whole CERN complex can be seen in figure~\ref{fig:Complex}. The injector chain has different stages that pre-accelerate protons and heavy ions to be injected into the main ring of LHC. In the main ring the protons and heavy ions are fully accelerated and collided in four different points over the ring.

\begin{figure}[!Hhtbp]
  \begin{center}
    \includegraphics[trim=4.5cm 0cm 0cm 0cm, clip=true, width=1.15\textwidth]{figs/cern-lhc-4.jpg}
    \caption{Organization of CERN complex}
    \label{fig:Complex}
  \end{center}
\end{figure}

\subsection{Injector chain}
\label{sec:injector}

The injector chain begins with the proton source. Protons are extracted via ionization of Hydrogen gas in the Duoplasmatron Proton Ion Source. Such extraction is pulsed, what makes up the first bunch structure. The extracted protons are then accelerated to 50 MeV in the linear accelerator, Linac2, that dates from 1978. After this first stage several steps are followed:
\begin{enumerate}
\item Linac2 injects proton bunches in the Proton Synchrotron Booster (PSB) where are accelerated to 1.4 GeV. 
\item From PSB, the protons are delivered to the Proton Synchrotron (PS) where they reach an energy of 25 GeV. In the PS the bunches are also split from 6 initial bunches to 72 spaced by 25 ns.
\item Finally, the pre-acceleration chain is finished by the SPS, Super Proton Synchrotron. There the bunches are accelerated up to 450 GeV right before being inserted to the main LHC ring. 
\end{enumerate}

The whole pre-acceleration chain has been optimized to obtain the best possible performance on the final acceleration in the LHC main ring. All parameters are carefully controlled, for example the number of bunches, the separation between bunches, the separation between trains of bunches or the injection energy to each subsystem. It's also remarkable to notice the level of control achieved in the bunches manipulation, from old subsystems as the PS from 1959 or the newest, the SPS that dates from 1976. 

Some recent plans for future accelerator have been theorized using the LHC main ring as injector for a bigger accelerator, for example the so called Very Large Hadron Collider or VLHC.  

\subsection{Main ring}
\label{sec:ring}

The main ring is composed of two rings that accelerate the proton bunches in opposite directions, clock-wise and counter clock-wise. An schematic view of the design of the main ring can be seen in figure~\ref{fig:schematic}. The rings crosses in different points in order to collide the protons and they are divided in eight straight sections and eight arcs. In each octant bunches are controlled by dipole magnets. These magnets, in figure~\ref{fig:dipole}, need to produce a very strong magnetic field in order to be able to bend a 7 TeV beam of protons. This intense magnetic field, 8.33 T, is produced by electrical currents that are only achievable by means of superconductivity. All the 1232 dipoles operate at a temperature of 1.9 K, under cooling by liquid helium. They also operate under ultra-high-vacuum. The beam lines with a pressure less than $10^{-9}$ mbar and the whole dipole system with $10^{-6}$ mbar, that serves also as insulating system from the surroundings. In addition, the LHC main ring has other magnets that focus and correct different characteristics of the beam: 520 quadrupoles, 2464 sextupoles, 1232 octupoles. 

\begin{figure}[!Hhtbp]
  \begin{center}
    \includegraphics[width=0.8\textwidth]{figs/lhc-schematic.jpg}
    \caption{Schematic of the LHC main ring design.}
    \label{fig:schematic}
  \end{center}
\end{figure}

\begin{figure}[!Hhtbp]
  \begin{center}
    \includegraphics[width=0.6\textwidth]{figs/cryodipole.jpg}
    \caption{Design of LHC cryodipole that bends the beam in the main ring.}
    \label{fig:dipole}
  \end{center}
\end{figure}

\subsubsection{Luminosity}
\label{sec:lumi}

On collider physics, such as the LHC, the figure of merit is the luminosity, given in equation~\ref{eq:lumiC}. The luminosity is proportional to the number of events per second, and in this terms is the quantity to be maximized by the design and operation of the accelerator. In terms of the collider characteristics depend on the number of bunches in the ring $k_{b}$, the number of protons per bunch $N_{b}$, the revolution frequency $f_{rev}$, the relativistic gamma factor $\gamma$, the normalized rms transverse beam emittance $\epsilon_{n}$ and the beta function at the interaction point $\beta^{*}$. The denominator on~\ref{eq:lumiC} can also be understood in terms of the horizontal and vertical width of the bunches at the crossing. In table~\ref{tab:LHCparams} can be found the LHC beam parameters at injection and collision.  

\begin{equation}
  \label{eq:lumiC}
  L=\frac{k_{b}N_{b}^{2}f_{rev}\gamma}{4\pi\epsilon_{n}\beta^{*}}
\end{equation}

\begin{table}[htbH]
\label{tab:LHCparams}
\begin{center}
%\topcaption{LHC proton beam parameters.\label{tab:LHCparams}}
%\resizebox{\textwidth}{!}{
\begin{tabular}{|c|c c|}
\hline 
Parameter/units & Injection & Collision \\
\hline
Energy [GeV]& 450 & 7000 \\ 
Luminosity [$\text{cm}^{-2}\text{s}^{-1}$] & & $10^{34}$ \\
$k_{b}$ Number of bunches & \multicolumn{2}{c|}{2808} \\
Bunch spacing [ns] & \multicolumn{2}{c|}{24.95} \\
$N_{b}$ intensity per bunch [protons/bunch] & \multicolumn{2}{c|}{$1.15\times 10^{11}$} \\
Beam current [A] & \multicolumn{2}{c|}{0.58} \\
$\epsilon_{n}$ normalized rms transverse beam emittance [$\mu$m] & 3.5 & 3.75 \\ 
$f_{rev}$ revolution frequency [kHz] & \multicolumn{2}{c|}{11.25} \\
\hline
\end{tabular}
%}
\end{center}
\end{table}

At the crossing points, the number of events coming from collisions and produced via an specific process, is directly proportional to the luminosity provided by the collider, as in equation~\ref{eq:lumiC}.

\begin{equation}
  \label{eq:lumiN}
  N_{events}=L\sigma_{process}
\end{equation} where $\sigma_{process}$ is the cross section of the process. 

The total cross section of a proton-proton collision from the crossing of two bunches at 14 TeV is 100-110 mb~\cite{Augier:1993ta}, from three different scattering processes: elastic, diffractive and inelastic. In the elastic scattering the protons only exchange momenta but their structure remain unchanged, that is the case for the majority of collisions. In diffractive scattering momenta is exchanged and also new particles are produced in addition to the two final protons. Finally, in inelastic scattering, the constituents of the protons, the partons, interchange a big amount of momentum and produce a large quantity of particles. The inelastic processes contribute less than diffraction to the total cross section. While inelastic collisions produce particles in the central rapidity (defined on~\ref{sec:Csys}) region, diffractive and elastic final products have a large rapidity. Only in the hard interactions, inelastic scattering, color is exchanged, being such the reason to fill up the central rapidity region. 

From the crossing of two bunches not only one proton-proton interaction is expected. Around 25 interactions are expected for each crossing. Form them, only one is coming from an inelastic collision, that is the type of process of more interest for detectors as CMS or ATLAS. This fact puts an additional difficulty to the detectors in order to extract the hard interaction from all the elastic and diffractive collisions happening at same time. Such phenomena is known as Pile-Up, an illustration of a collision with high pile-up can be found on figure~\ref{fig:pileup} as seen by CMS detector.

\begin{figure}[!Hhtbp]
  \begin{center}
    \includegraphics[width=0.7\textwidth]{figs/pileup.png}
    \caption{High pile-up event (78 interactions) seen by CMS detector. Image credit: Andre Holzner \copyright CERN}
    \label{fig:pileup}
  \end{center}
\end{figure}

\subsection{Run 1}
\label{sec:run1}

On February 10 of 2013 the first stable run of the LHC reached an end. This run, now called Run 1, started on November 20 of 2009. The original LHC start was planned for 2008, an incident on one of the electric connections of one of the magnets forced to stop the 19th of September of the same year. From the restart on 2009, the energy was augmented from 450 GeV to 4 TeV per beam. The 23th of September 2009 the first collisions were detected by the experiments. One week after the achieved center of mass energy was $\sqrt{s}=2.36$ TeV, already more than Tevatron (1.98 TeV). 

On 2010, from 30th March to 6th December 3.5 TeV per beam were reached delivering near 50 $\text{pb}^{-1}$. With the same energy, approximately 6 $\text{fb}^{-1}$ were delivered on 2011. 

On 2012, the center of mass energy reached one additional TeV, $\sqrt{s}=8$ TeV, and around 20 $\text{fb}^{-1}$ of integrated luminosity were delivered between April and December. On figure~\ref{fig:CMSlumi} can be seen the progress of the recorded luminosity by CMS for 2010-2012 period. The first six weeks of 2013 were devoted to proton-lead collisions.

After this very successful run, the LHC has been stopped for a year for repair and maintenance of different systems in the experiments and in the LHC itself. After this period, known as Long Shutdown  or LS1, the LHC is planned to restart a new run on the early spring of 2015.

\begin{figure}[!Hhtbp]
  \begin{center}
    \includegraphics[width=0.7\textwidth]{figs/cms-int-10to12.jpg}
    \caption{CMS integrated luminosity for proton-proton collisions delivered by LHC. \copyright CERN}
    \label{fig:CMSlumi}
  \end{center}
\end{figure}

\section{The Compact Muon Solenoid experiment}
\label{sec:CMS}

\subsection{Coordinate system}
\label{sec:Csys}

\subsection{Tracker system}
\label{sec:tracker}

\subsection{Electromagnetic calorimeter}
\label{sec:ecal}

\subsection{Hadronic Calorimeter}
\label{sec:hcal}

\subsection{Muon chambers}
\label{sec:muons}

\subsection{Trigger}
\label{sec:trigger}


%  LocalWords:  emittance
