\chapter*{Outline}

The present work has been structured in five chapters.

In the first chapter I describe the Standard Model and its predictions. I also describe in detail the top quark and the Higgs boson, their properties, how they have been measured and their production according to the Standard Model. Afterward, I proceed to describe an effective extension of the model and one of the new particles predicted by it. 

In the second chapter the experimental set up is detailed. In first instance, the Large Hadron Collider is described and the experiments based on it. Then, I pay special attention to the description of CMS experiment. This chapter is closed by the description of the techniques used in CMS to reconstruct particles from proton-proton collision events. 

The third chapter is devoted to cast the phenomenological study I performed during my first year of thesis. I describe in detail the feasibility study done for a possible search in data of a Vector Like top partner. This chapter serve as theoretical motivation to the data analysis describe after in chapter 5.

During the first two years of my thesis I have performed different tasks associated to the Monte-Carlo simulations of proton-proton collisions. This work is drawn up in the fourth chapter. I discuss there, what are Monte-Carlo simulations, how they are used in particle physics and the different packages in the market to produce them. I also show the specific task I have performed for CMS collaboration on the matter.

The fifth chapter is the main product of my work during my thesis. It contains the data analysis I performed using CMS data for proton-proton collisions provided the LHC at 8 TeV. I present in detail all the techniques used in the analysis and its results.