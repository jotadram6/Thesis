\chapter[The Standard Model]{The Standard Model}
\label{chap:SM}
Since the Greeks, different theories about the composition and structure of the world have been formulated. At ancient Greece this theories were elaborated from a philosophical point of view. Nowadays, we count with a very sophisticated  set of tools and concepts that allowed us to build up a general vision of nature, its components and structure. Moreover, on the subject of the constituents, or elemental constituents, we have developed a theory that is capable to describe the majority of known phenomena. This theory is the Standard Model (SM) of particle physics. 

This SM relies in two of the more fancy constructs of modern physics and mathematics. From physics side, the quantum field theory; from mathematics, group theory. Quantum field theory has born from the understanding of processes that take place at very small spatial scales but in a regime where special relativity play an important role. To describe such, a major part of the most brilliant minds of the 20th century dedicated their life, Paul Dirac, Richard Feynman, Enrico Fermi among them. The theory of quantum fields has set in a common place two extraordinary achievements of physics: special relativity and quantum mechanics. With it we have been capable to describe many phenomena: $\beta$ decay and $\alpha$ decay, solid state, with many other.

From the mathematics side, group theory has become one of the most powerful tools for particle physicist. However, their development began quite early, with Galois at XXXX, and was used in other parts of physics, it's with Lie algebras and the possibility of describing continuous symmetries that the most important step will be given. Also, this would have not been possible with the amazing connection found by Emmy Noether in XXXX. Her finding connected symmetries and physics in a form never known before. She found that for every conserved quantity in a system there is a symmetry followed by it. As group theory can be seen, in grosso modo, a way to mathematically describe symmetries, group theory became the tool to describe systems with conserved quantities. 

In this chapter, we are going to present the basics of the SM. We describe its seminal ideas, it structure and content and it's ultimate consequences. Finally, we close with its limitations.

\section{Symmetires and interactions}
\label{sec:symm}

\section{Fields and particles}
\label{sec:fields}

\subsection{The mass problem}
\label{sec:mass}

\subsection{Spontaneous Symmetry Breaking}
\label{sec:SSB}

\subsection{Higgs mechanism}
\label{sec:higgs}

\section{Hierarchy problem and other limitations}
\label{sec:hier}

