\chapter[The Standard Model]{The Standard Model}
\label{chap:SM}
Since the Greeks, different theories about the composition and structure of the world have been formulated. At ancient Greece this theories were elaborated from a philosophical point of view. Nowadays, we count with a very sophisticated  set of tools and concepts that allowed us to build up a general vision of nature, its components and structure. Moreover, on the subject of the constituents, or elemental constituents, we have developed a theory that is capable to describe the majority of known phenomena. This theory is the Standard Model (SM) of particle physics. 

This SM relies in two of the more fancy constructs of modern physics and mathematics. From physics side, the quantum field theory; from mathematics, group theory. Quantum field theory has born from the understanding of processes that take place at very small spatial scales but in a regime where special relativity play an important role. To describe such, a major part of the most brilliant minds of the 20th century dedicated their life, Paul Dirac, Richard Feynman, Enrico Fermi among them. The theory of quantum fields has set in a common place two extraordinary achievements of physics: special relativity and quantum mechanics. With it we have been capable to describe many phenomena: $\beta$ decay and $\alpha$ decay, solid state, with many other.

From the mathematics side, group theory has become one of the most powerful tools for particle physicist. However, their development began quite early, with Galois around 1830, and was used in other parts of physics, it's with Lie algebras and the possibility of describing continuous symmetries that the most important step will be given. Also, this would have not been possible with the amazing connection found by Emmy Noether in 1918. Her finding connected symmetries and physics in a form never known before. She found that for every conserved quantity in a system there is a symmetry followed by it. As group theory can be seen, in grosso modo, a way to mathematically describe symmetries, group theory became the tool to describe systems with conserved quantities. 

In this chapter, we are going to present the basics of the SM. We describe its seminal ideas, it structure and content and it's ultimate consequences. Finally, we close with its limitations.

\section{Symmetries and interactions}
\label{sec:symm}

From the very beginning of physics, one of the most fundamental questions has been how does bodies interact, and with it what exactly and interaction is. On the first type of interaction ever studied by physics, gravity, Newton proposed the concept of distant interaction, the idea that bodies could interact without being in direct contact. But the question on how exactly that distant action was performed remained unanswered. 

During the 19th and 20th century new phenomena were discovered pointing to brand new interactions, electricity, magnetism, radioactivity and nuclear structure inside atoms. The very precise and complete description of electromagnetism developed by Gauss, Faraday, Amp\`{e}re and finished by Maxwell arrived to the describe electricity and magnetism under the formalism of only one interaction within the mathematical formalism of classical fields. 

The definition of a classical field is an assignment of a quantity to every point in space and time. For physics the quantity that is attributed it's a physical quantity such as mass, electrical charge or probability. This quantity can be scalar or vector, giving rise to the notion of scalar or vector field. As an example, a fluid can be described in terms of fields, being the velocity of the fluid a vector field and its pressure a scalar field. Generic classical electromagnetic interactions can be described with the help of one vector field $\vec{A}(x)$, the vector potential, and one scalar field $\phi(x)$, the scalar potential. In the formalism of four-vectors from relativistic dynamics one can organize this two quantities in the four-potential $A_{\mu}=(-\phi,\vec{A})$. This can be used to define the strength field tensor $F_{\mu\nu}=\partial_{\mu}A_{\nu}-\partial_{\nu}A_{\mu}$, where $\partial_{\mu}$ is the covariant derivative. From the tensor is possible to obtain in a very generic and elegant way the equations of motion of the free field using the Lagrangian formalism, as in equation~\ref{eq:electromotion}. With the Lagrangian density defined in equation~\ref{eq:electrolagran}.

\begin{equation}
  \label{eq:electromotion}
  \partial_{\mu}\left( \frac{\partial \mathcal{L}}{\partial (\partial_{\mu}A_{\nu})} \right) -\frac{\partial \mathcal{L}}{\partial A_{\nu}}=0
\end{equation}

\begin{equation}
  \label{eq:electrolagran}
  \mathcal{L}=-\frac{1}{4}F^{\mu\nu}F_{\mu\nu}
\end{equation}

\section{Quantum fields and particles}
\label{sec:fields}

Classical fields, introduced and described in last section~\ref{sec:symm}, can be extended to a quantum theory. Such procedure is known as the quantization of fields and allow to unify special relativity and quantum mechanics in one theory, Quantum Field Theory (QFT), to describe the dynamics of systems in such regimes: rapidity close to the speed of light on the atomic or smaller scales. 

\subsection{The mass problem}
\label{sec:mass}

\subsection{Spontaneous Symmetry Breaking}
\label{sec:SSB}

\subsection{Higgs mechanism}
\label{sec:higgs}

\section{Hierarchy problem and other limitations}
\label{sec:hier}

